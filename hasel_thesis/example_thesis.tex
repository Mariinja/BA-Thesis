\documentclass{hasel_thesis}

\thesisType{Bachelor}
\date{\today}
\title{Motivating effective breaks for knowledge workers with Break  Scheduler }
\subtitle{Bachelor Thesis}
\author{Marinja Principe}
\home{Niederhasli} % Geburtsort
\country{Switzerland}
\legi{18-740-910}
\prof{Prof. Dr. Thomas Fritz}
\assistent{André Meyer}
\email{marinja.principe@uzh.ch}
\url{<url if available>}
\begindate{19.09.2022}
\enddate{19.03.2023}


\begin{document}
\cite{KIM198871}
\maketitle

\frontmatter

\begin{acknowledgements}
\end{acknowledgements}

\begin{abstract}
This is an abstract.
\end{abstract}

\begin{Summary}
This is the Summary.
\end{Summary}
    

\tableofcontents
\listoffigures
\listoftables
\lstlistoflistings

\mainmatter
\chapter{Introduction}
\section{Importance of breaks}
- why do we need breaks
- How does it influence knowledge workers
- Examples of papers which have shown breaks are important
\section{Good / Effective of breaks}
 - Definition of a break
 - what is a good break what is a bad break?
 - How can we influence breaks (timing, duration and activities)
    - show paper with different activities and show that it is important how to spend break time

 - what is already done in relation to breaks?
    - speak about google tool, activity study etc.

 - show there is not tool incorporating all 3 aspects
 - Explain the goal and approach of this thesis
 

\chapter{Related work}
\section{Timing of breaks}
- what have other studies already shown
- what is their approach
- what can I learn in finding good time slots for breaks
- Are there different types of people with different timing breaks
- What are the factors influencing the timing? (input variables?)

\section{Duration of the break}
- what have other studies already shown
- what is their approach
- what can I learn in finding good time slots for breaks
- Are there different types of people with different timing breaks
- What are the factors influencing the duration? (input variables?)
- Is it correlated with the timing?

\section{Activity during the break}
- what have other studies already shown
- what is their approach
- Are there overall good activities? or is it personal
- Are there different types of people with different needs for activities?
- What are the factors influencing the activities? (input variables?)

\chapter{Personalized Break Planner}
\section{Definition of the 3 Key Aspects}
\subsubsection{Timing of breaks}
\subsubsection{Duration of the break}
\subsubsection{Activity during the break}
\section{Rule-based method for adapting personal needs}
\subsubsection{Initial Break plan}
- How do i generate the initial break plan
- What different people do we have
- How can the user adjust this plan
- What is the theory behind it
\subsubsection{Adapting to user's needs}
- How is the system working?
- what are trigger values?
- what is the weight of the trigger values?
- Why did i choose this concept? (Show theory)

\section{Implementation}
\subsubsection{Frontend}
\subsubsection{Backend}



% %
% \subsubsection{Subsubsection}
% \fig[.5\textwidth]{logos/logo_hasel}{Our logo}{logo}

% \subsection{Subsection}
% %
% \paragraph{Paragraph.} Always with a point.

% \begin{lstlisting}[caption=An example code snippet]
% /**
%  * Javadoc comment
%  */
% public class Foo {
% 	// line comment
% 	public void bar(int number) {
% 		if (number < 0) {
% 			return; /* block comment */
% 		}
% 	}
% }
% \end{lstlisting}

\appendix
\chapter{First Appendix}
\chapter{Second Appendix}

\backmatter
\bibliographystyle{alpha}
\bibliography{hasel_thesis/text}



\end{document}
