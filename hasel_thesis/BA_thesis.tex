\documentclass{hasel_thesis}

\thesisType{Bachelor}
\date{\today}
\title{Motivating effective breaks for knowledge workers with Break  Scheduler }
\subtitle{Bachelor Thesis}
\author{Marinja Principe}
\home{Niederhasli} % Geburtsort
\country{Switzerland}
\legi{18-740-910}
\prof{Prof. Dr. Thomas Fritz}
\assistent{Dr. André N. Meyer}
\email{marinja.principe@uzh.ch}
\url{<url if available>}
\begindate{19.09.2022}
\enddate{19.03.2023}


\begin{document}

\maketitle

\frontmatter

\begin{acknowledgements}
\end{acknowledgements}

\begin{abstract}
This is an abstract.
\end{abstract}

\begin{Summary}
This is the Summary.
\end{Summary}
    

\tableofcontents
\listoffigures
\listoftables
\lstlistoflistings

\mainmatter
\chapter{Introduction}
%Introduction: genau, hier kommt die Motivation und Zusammenfassung vergleichbarer Forschung, um den Gap zu zeigen, anschliessend eine Zusammenfassung deiner Arbeit, Evaluation und Erkenntnisse, sowie Zusammenfassung der Contributions 

%- Importance of breaks
% %- why do we need breaks
% - How does it influence knowledge workers
% - Examples of papers which have shown breaks are important
%Why do we need breaks?
%What is a break? (vacatation, ..


\section{Motivation}
- Background and context of the research problem
- Research problem and objectives
- Research Methodology





Most knowledge workers face a highly fragmented workload and many commitments, which can cause stress and dissatisfaction. This can lead to burnout or problems for the organization \cite{Elkin.1990}. Therefore, it is important to learn more about the recovery process of employees. Various studies \cite{Largo-Wight.2017},\cite{KimS.ParkY.&Niu.2017} have shown that regular breaks can significantly reduce work-related stress.  In addition, regular breaks can reduce physical discomfort \cite{Waongenngarm.2018}, but also help people stay focused \cite{Ariga.2011} \cite{Bloom.2014} and thus keep productivity high throughout the workday. As discussed by \cite{Trougakos.2009}, there are different types of breaks. They are differentiated into vacation breaks, breaks between work days (such as weekends), and breaks within work days. While the effect of vacation breaks are usually short-lived, short, regular breaks within the workday have greater value for resting a worker's personal and physiological resources.
Some studies investigated how to find the right time for a break to increase productivity and well-being. Kaur et al. \cite{Kaur.2020} explored the finding of opportune moments for knowledge workers to take a break, using affect, workstation activity and task data. Their study demonstrated a 86\% accuracy of the approach for predicting opportune moments. In addition, several commercial approaches and tools that support reminders for taking breaks exist and are already widely used \cite{Alghamdi.2020}, for example the Pomodoro technique \cite{Cirillo.2006} or the 20-20-20 rule \cite{Min.2019}. Packer et al. \cite{Packer.2021} showed that micro-breaks have a restorative effect on well-being and focus.  However, finding the timing of the micro-break is essential as shown by Kaur et al. \cite{Kaur.2020}. Nevertheless, different studies \cite{KimS.ParkY.&Niu.2017} \cite{Berman.2007} have demonstrated that there is no one "perfect" break schedule that suits everyone. While some people prefer and benefit more from having regular short breaks to the coffee machine, others prefer to go for a walk during their lunch break. Therefore, finding opportune moments for a good break and knowing how best to spend them remains a challenge. 

Since each person has only a limited amount of personal resources, it is necessary to understand how to use them up and, more importantly, how to recover them. The theory of resource depletion \cite{BaumeisterR.F.BratslavskyE.MuravenM.&TiceD.M..1998} states that each person has only a limited amount of "personal resources" that can be consumed, but also recovered, depending on the activity. Activities that consume resources are so-called chores. On the other hand, activities that restore resources are called respite breaks \cite{Trougakos.2009}. Whether an activity restores or consumes resources varies from person to person. This indicates that break activities are key to a good break. But when, how long, and with what activities should a good and effective break be spent? 


%One of the most effective ways to promote physical and mental health is to incorporate small positive habits  into everyday life \cite{Taylor.2005}. Since many adults spend a large portion of their day at work, it can be valuable to consider the time they spend at work for incorporating small positive habits. Most knowledge workers face a highly fragmented workload and many commitments, which can lead to stress and dissatisfaction. Different studies \cite{Largo-Wight.2017},\cite{KimS.ParkY.&Niu.2017} have shown that regular breaks can significantly reduce work-related stress.  In addition, regular breaks may reduce physical discomfort \cite{Waongenngarm.2018} but also help to stay concentrated \cite{Ariga.2011} \cite{Bloom.2014} and, thus, keep productivity high throughout the workday. 
%But when, how long and with which activities should a good and effective break be spent? 




%- Good / Effective breaks
 % - Definition of a break
 % - what is a good break what is a bad break?
 % - How can we influence breaks (timing, duration and activities)
 %    - show paper with different activities and show that it is important how to spend break time

 % - what is already done in relation to breaks?
 %    - speak about google tool, activity study etc.

 % - show there is not tool incorporating all 3 aspects
 % - Explain the goal and approach of this thesis

Before we look into the definition of a \textbf{good break} it is important to define the term break. \cite{Trougakos.2009} defines a break as a time in which work-related tasks are not expected nor required to be done. In other words, a break is a time when the owner can decide how to spend it, in order to achieve a good break. In this work, we are using the definition of a good break by \cite{Trougakos.2009}. A "good break" is a break that recovers the personal resources of an employee. On the other hand a "bad break" will drain the anyway limited resources. If an activity recovers or drains resources is depending on two main aspects: The amount of effort, which is used to fulfill this activity and whether it is enjoyable for the person. Good ways of spending breaks are to spend time in nature, relax or perform physical activity \cite{Bloom.2014} \cite{Largo-Wight.2017}.  As shown by Kim et al \cite{KimS.ParkY.&Niu.2017}, there are different types of breaks. They are divided into relaxation, nutritional, social, and cognitive micro-breaks. Each of these groups has a different goal and therefore a different use. For example, stretching exercises and naps can contribute to relaxation, while short conversations with work colleagues about non-work related topics improve social support. These activities also affect the duration of a break. But it's not just the activity that affects the duration of a break. Different techniques, such as the Pomodoro technique or the 20-20-20 technique, recommend different break duration.

Overall, existing approaches considered static break schedules and didn’t let users personalize the break timing to individual preferences and needs (e.g. \cite{Henning.1997} \cite{Cooley.2013}), or didn’t support users in identifying how to best spend their breaks, to facilitate recharging and recovering (e.g. \cite{Kaur.2020}). 
The goal of this work is to develop a personalized break planner that supports users in finding the opportune time for an activity and deciding on an activity in order to develop good/effective break habits. 



The following questions want to be answered during a User study to evaluate the suggested solution:

-	[Awareness] RQ1: How can self-reflection and nudging increase knowledge workers' awareness about their existing break habits and their personal resources, such as energy, attention levels or physical well-being?

Awareness is an important aspect of change. First, e person needs to be aware of their action before they can evaluate whether their actions need change or not. Therefore the Break Scheduler wants to be able to raise awareness of its user's personal resources, such as energy, attention and physical well-being. Input data for raising awareness of personal resources are limited to self-reports. The rule-based system and nudging should help the user to evaluate their behaviour and suggest better alternatives. The goal is also to find specific features of the solution which can increase knowledge workers' awareness of break habits.


-	[Impact] RQ2: Can a personalizable break scheduler support knowledge workers' identification of good break activities and what is the impact on their break habits and ability to recharge personal resources during breaks?

When awareness is raised the user can act on it and generate an impact. For each break, the Break scheduler suggests a timing, duration and activity which was evaluated by the rule-based system. The impact of the rule-based system and the different analysis features in to be evaluated in the user study. To evaluate the impact, a pre-questionnaire will be compared with the user data of the tool, as well as the post-questionnaire. As the Break Scheduler should enable users to find good breaks, how it can enable the user to find good breaks and if there exist activities beneficial for different people.


-	[Tool] RQ3: Which aspects of the personalizable break scheduler prototype are most helpful, and are there design improvement opportunities?
In the end, the implementation of the Break Scheduler will be discussed. Potential advantages and disadvantages will be discussed, in order to create a personalizable tool which helps the user to identify good break habits and raise the awareness of their resources. The fisability of it incorporation into the daily life and andling will also be discussed.


\section{Description of Work}
- Structure of the thesis


%
 %Factore for a "good break" \cite{Bloom.2014}
 %- Recovery process
 %- health (physical and mental)
 %- well-being
 %- job performance
 %- creativity

 \chapter{Background}
 \begin{comment} 
 - Definition of a Break
 - Personal Resources and Physiological Capacities
 - Self-Reporting /self-Experimenting

 
\end{comment}

 This section will establish a common understanding of the theory used for the presented solution. First, the background of personal resources and physiological capacities will be looked at, followed by an introduction to work recovery. At the end, the definition of a break used for this thesis and an overview of different break types is shown.

\section{Personal Resources and Physiological Capacities}

\begin{comment} 
 - what are personal reouces, what are physiological capacities (done)
 - why is it important (done)

 
\end{comment}
\subsection{Definition of Personal Resources and Physiological Capacities}
In the study by 
Baumeister et al. \cite{BaumeisterR.F.BratslavskyE.MuravenM.&TiceD.M..1998}, they define \textbf{personal resources} as a limited set of sources that enable a person to complete various activities during the day. The amount of resources varies from person to person. This concept has been applied in multiple areas of psychology and organizational behaviour. While some studies focus on specific factors such as energy or stress, personal resources encompass various aspects. These include (but are not limited to) factors such as attention, energy, well-being, creativity, decision-making, and social contacts. Each person has a limited amount of personal resources available daily, so it is essential to know how to recharge them. 

\textbf{Physiological Capacity}, on the other hand, is a limited resource beyond which a person is physically unable to continue without rest. An example would be sleep or food. Each person has a limit to how long they can physically go without sleep or food. To restore physiological Capacity, each person must sleep, eat, or drink. Exhaustion of physiological resources usually leads to serious health consequences. At this point, it is essential to mention that the two resources (personal and physiological) are not entirely independent.
Some studies, such as the following, do not separate personal resources from physiological Capacity. These theories use the definition of limited resources, which is crucial for the next chapter on recovery.

Meijman and Mulder's \cite{Meijman.1998} focus on the effort-recovery model, which deals with workers' efforts and their work management. The central concept is that a worker's work effort leads to work overload and decreased personal resources. According to Meijan and Mulders, recovery can only begin by taking time away from the overall work demand. This shows the importance of the recovery process and awareness of personal resources.

Hobfoll's \cite{Hobfoll.1989, Hobfoll.1998} conservation of resource theory compares stress and well-being and investigates the relationship between them. They could show that the loss of resources is more noticeable and salient than its gaining. This enhances the point that it is suggested to prevent a resource deficit proactively.

Both studies include the concept of work demands that require effort or resources to be completed. The following section discusses the concepts of job demand and job control.



\subsection{Job demand and Job Control}
The term \textbf{Job demand} refers to the features (physical, psychological, social etc.) of the job that requires effort and, therefore, specific resources cost to complete them \cite{Trougakos.2009}. Examples of job demands are a heavy workload or time pressure. Previous studies could show that a high work demand affects the well-being \cite{Sonnentag.2003} of the person as well as their needed recovery \cite{Sonnentag.2006}. A high job demand often leads to the feeling of time pressure. On the one hand, this leads to a depletion of resources as the increased demand requires more effort to complete all the tasks. As the time pressure is already high, the time for breaks and recovery will be relatively short or cut altogether, leading to less or no time for recovery while using more resources than available. This is also why many people with a high job demand engage in more chore activities such as work or running errands.
Compared with people with low job demands, short breaks for people with high demand may not be sufficient to recover from the defecate resources created over time. In such cases, a more significant period, for example, some days off, can only restore their resources effectively. However, when the resources are refilled, it is essential to pro-activity engage in activities which restore the personal resources to avoid a deficit again.


The job demands-resources model \cite{BakkerA.B.DemeroutiE.DeBoerE.andSchaufeliW.B..2003b, DemeroutiE.BakkerA.B.NachreinerF.andSchaufeliW.B..2001a} also includes a similar notion of demand. Job demands refer to aspects of work concerning physical, psychological, social or organizational factors. Each task uses a particular effort and skills, which leads to a patch of demand costs, e.g. physical or psychological costs. When the employee can not recover from this job demand, the demand can turn into job stressors \cite{Hobfoll.1998}. Such stressors can lead to physiological (e.g. cortisol level increase) or/and psychological (e.g. fatigue) symptoms \cite{Sonnentag.2022}. Resources are essential to balance each person's needs and work demands. As defined in the job demands-resources model, Job resources refer to physical and psychological aspects required to finish tasks, reduce job demand and achieve personal growth. Hence, not only are resources necessary as a balance for job demand but also for the development of each person. Job resources have the potential for motivation, leading to high engagement and performance. It, therefore, also has an intrinsic role \cite{Bakker.2007}.

If this balance between demand and resources is off, the work demand can become a stressor. In addition, people with fewer job or personal resources are more prone to resource depletion because abundance also consumes resources \cite{Trougakos.2009}. 

\textbf{Job control}, on the other hand, refers to the opportunity of each employee to influence their job activities \cite{Trougakos.2009} and enable them to take decisions. Many studies could already show that high job control benefits the employee's well-being \cite{Daniels.1994, Jackson.1983}. High job control has an advantage in two ways. First, employees with high job control are flexible to take breaks or switch activities when tired, which helps them recharge their resources flexibly to their needs. Secondly, each employee can structure their day to fit their needs, for example, by joining a yoga course during the lunch break. To summarize, employees with high job control are a) more likely to take a break when it is needed and b) more flexible in the selection of the activity, as they can vary their break durations.
On the contrary, studies have shown that employees with high job control and no formally scheduled breaks often do not take any breaks and overwork themselves \cite{McLean.2001}. They fail to take a break when needed, so they only take breaks when their resources are already meagre. If they then engage in breaks activities, it is not enough anymore to take short breaks but need an extended break. Especially employees with less repetitive work (e.g. knowledge workers) can benefit from high job control \cite{Trougakos.2009}. Job control also affects the perception of certain activities. Let us consider the example of working during lunchtime. A person with a high job demand decides to work during lunchtime, knowing that she /he has the control to stop whenever needed. Compared to a person with low job control, whose employer asks to finish a task before leaving, the perception of the person with high job control will likely not be as draining as the other.

What happens when the equilibrium between an individual's personal resources and the demands and control within their job becomes imbalanced?

\subsection{Consequences of Personal Resources levels}
The consequences of different levels of personal resources will be discussed in more detail to underline the importance of personal resources. 
Other research has found a relationship between stress and depletion of resources \cite{Sonnentag.2001}. Additionally, Trougakos and Hideg \cite{Trougakos.2009} suspect that low resources lead to high emotional exhaustion. Their explanation is the following. When personal resources are low, people are more prone to stress and less able to deal with it. This theory will also be covered by one of the most influential stress theories by Lazarus and Folkman \cite{Lazarus.1984}, which state that personal and social resources play an important role when dealing with stress. According to Lazarus and Folkman, specific tasks will be perceived as a stressor when coping with stress, whereas when resources are high, they will be more likely to be perceived as a challenge.
Furthermore, Lazarus and Folkman \cite{Lazarus.1984} differ between two coping mechanisms: Problem- or emotion-focused strategies. People with low resources prefer to distance themselves from a problem as they cannot deal with it now. On the other hand, people with high resources are more likely to tackle the issues and try to solve them, which will change the current situation, leading to a likely better position.
Last but not least, one argues that the performance during the day depends on the assigned focus, which depletes the resources during the day. It is necessary to restore personal resources to keep the performance during the day high. When this is not done, the performance can degrade\cite{Trougakos.2009}, as these resources are limited.

As stated above, the consequences of personal resources are significant for different factors, including the employee's well-being and job performance. Therefore it is crucial to understand how to recover personal resources from the daily work demand.
 

 \section{Work Recovery Research}
  \begin{comment} 
 - why are we looking at recovery for work --> many knowledge workers are stressed and recovery is key
  - How can we recharge this resources
 - Different ways of recovery: 
    - organisational
    individual -->breaks

 
\end{comment}
As seen in the last section, personal and physiological resources are limited and differ from person to person. A deficit of these resources can lead to depletion of work or/and decrease the mental or/and physical well-being. The recovery of these resources is, therefore, a fundamental aspect. Sonntag et al. define recovery as an “unwinding and restoration process during which a person’s strain level has increased as a reaction to a stressor, or any other demand returns to its prestressor level” \cite[p.366]{Sonnentag.2017}. Usually, such stressors are generated by work demands, including negative state characteristics. This negative state can be characterized by high arousal as anger or anxiety, but it can also be characterized by low arousal as exhaustion, depression or fatigue \cite{Sonnentag.2022}. As restoration is done by eliminating the demand for a certain period of time \cite{Trougakos.2009}, the recovery can only start when the job demand has ended, including the psychological detachment from work. 
However, can this demand not be decreased, as the person is mentally still present, or demand remains due to other stressors; no recovery can take place, and the load reactions accumulate over time. This leads to a higher work demand and a high resource depletion since abundance also consumes resources \cite{Trougakos.2009}.


The good thing is that \textbf{motivation} can be a counteract of depleted resources \cite{Bandura.1986}. Work motivation is defined as the will to achieve a certain goal \cite{Locke.2004}. Locke \cite{Locke2000} has shown that job performance depends not only on one's skills or abilities but also on motivation. As every job task leads to a certain goal, the perception of the goal is crucial. The Goal-Setting Theory by Latham and Locke \cite{Latham.1991} states that when a goal is challenging and more advanced, it will lead to a greater outcome. More challenging goals often lead to bigger rewards, such as pay or a higher reputation. Therefore, employees often feel that even if their resources are low, it will be beneficial for them to continue and achieve the goal, increasing their focus and effort. The motivation helps them strive for their goal and temporarily compensate for low resources. On the other hand, a goal will not be perceived as "worthy"; more energy is needed to complete the task or will not even be complete at all \cite{Trougakos.2009}. Muraven and Slessareva \cite{Muraven.2003} conducted a study which could show that a depleted person can continue working on tasks which are perceived as essential or beneficial for them or others. They could show that depleted people continue working on self-control tasks which they believe are helpful to others, whereas people who don't believe the tasks will help others do not continue. Additionally, people who thought they could personally benefit from the task performed better than people who didn't. In summary, Muraven and Slessareva \cite{Muraven.2003} state that depleted people are motivated for a task when the benefits of the outcome are higher than the associated cost of their resource depletion. On the other hand, a clear direction and goals can have an impact on well-being and stress levels when resources are low. Previous research could show that role ambiguity is positively related which stress and even emotional burnout \cite{Posig.2003}. Role ambiguity will be defined as the lack of clear goals and/or directions. A depleted person needs clear goals and direction to know where to invest their remaining energy to achieve the goal. Additionally, when resources are already low, formulating and defining new clear goals might be challenging. When speaking about motivation, it is essential to differentiate between intrinsic and extrinsic motivation. Intrinsic motivation refers to having a choice, for example, when a person autonomously and volitionally defines a goal. This motivation inherits interest and is enjoyable. On the other hand, extrinsic motivation is forced and controlled. They lead to external validation as rewards or avoiding punishments \cite{Ryan.2000}. Intrinsic motivation employees will perform better as the tasks are more enjoyable and, therefore, easier to complete.

Another important definition when looking into work recovery is the definition of \textbf{chore and respite activities} \cite{Trougakos.2009}. According to their definition of resource depletion \cite{BaumeisterR.F.BratslavskyE.MuravenM.&TiceD.M..1998}, each activity can consume or restore personal resources, depending on different factors and preferences. 

Activities that consume more sources than they can recover are called \textbf{chores}. They continue draining resources and fail to restore personal resources. Such activities can negatively affect work performance or well-being. Sonntag \cite{Sonnentag.2001} could show that such activities primarily represent continuing with work in different ways, for example, switching tasks, domestic tasks or running errands and generally require an increased regulation of behaviour. These activities are mostly not enjoyed by the person, resulting in resource depletion. It is essential to state that the perception of if an activity is a chore depends on the person. An example would be cooking, which some people enjoy leading to the storage of personal resources, while others don't, leading to a depletion of their resources.


On the other hand, activities that restore these sources are called \textbf{respites}. To recover, activities need to be low effort, preferred choice, or enjoyable \cite{Trougakos.2009}. As shown in the sections above, high resources positively affect job performance and well-being. Therefore respite activities are essential for job performance and well-being too. Trougakos and Hidge specify two characteristics for a respite activity: the amount of effort and the degree on how much it was their preferred choice. A low effort helps the person to reduce the demand and uses fewer resources to do the activity, resulting in a higher resource outcome. High-effort activities mostly increase demand, which is disadvantageous for the demand reduction necessary for recovery. A high effort can be balanced by the degree of preferred choice. These activities can aid in the recovery of depleted resources as they are enjoyable or energize the user. They generate a positive feeling and well-being. Exercise is an example of a high-effort but preferred choice activity, as the activity uses a lot of energy but leaves the person with a positive feeling and a higher mental and physical well-being. These two factors define if an activity is perceived as a respite or chore activity. However, as seen in the given an example, the balance between them is crucial and differs from person to person. However, it is suggested that respites are selected for break activities, as they restore personal resources, whereas chores are advised to be avoided. This shows that break activities are essential factors for a beneficial break.

As a common understanding of personal resources and how to recover resources is established, the definition of a break can be looked at in more detail.
 


\section{Definition of a Break}

A work break is defined by \cite{Trougakos.2009} as a period of time when no work-related tasks are expected or need to be completed. It is suggested that breaks engage in activities with low demand on resources, enabling the user to recover them. Moreover, the recovery ability of the break depends on how the break time is spent. The activity of a break is, therefore, key. This includes various types of breaks such as vacations, weekends, breaks at the end of the day, and breaks during work, including micro-breaks. In order to understand the advantage and disadvantages of these types of breaks, it will be further elaborated.


Lounsbury and Hoopes \cite{Lounsbury.1986} define vacation as a period of time "when a person is not actively engaged in his or her work. It is a time when a person is free to pursue other interests, and therefore a time when the work situation may become less important compared to other areas of experience such as family and personal leisure" \cite[p. 393]{Lounsbury.1986}. The goal is to regain energy to be more productive during work time. It has been shown by \cite{Westman.1997, Westman.2001} that vacations reduce stress and lower burnout levels. Additionally, employees are more likely to engage in respite activities, which are relaxing and enjoyable, leading to positive emotions and recharging personal resources \cite{Fritz.2006}. Unfortunately, these effects wear off within days or weeks after returning to work again. This means that vacations have only a short-term positive impact on stress levels \cite{Fritz.2006}.


Weekends and end-of-day are mini-vacations between work days that allow workers to relax and engage in other non-work related activities. Unfortunately, there are not yet as many studies that specifically look at weekends and their benefits to workers, but the studies that do exist show that the lack of time off can lead to lower well-being or burnout \cite{Fritz.2005}. Weekends promote rest and well-being through restorative and stimulating activities suggested with low demand. Additionally, engaging in favourable and low-demand activities at the end of the day leads to greater recovery. Weekends and end-of-day can be used to recharge personal resources, but only when invested in respite activities. Resources are depleted if they are used for chore activities such as longer working hours or running errands.


The first study of breaks during the working day goes back to Mayo in 1933 \cite{Mayo.1933}, in which a very general form was studied. The best-known types of breaks during the day are coffee or lunch breaks. A particular type of with-in-day breaks are microbreaks, defined as really short breaks, e.g., less than 10 minutes \cite{BennettAndrewA.GabrielAllisonS.CalderwoodCharles.2020}. Several studies have already examined the benefits of such breaks and how they can contribute to the recovery of personal resources. For example, just 5 minutes of stretching can help combat muscle fatigue that typically occurs after 40 minutes. Such research mainly focuses on the interval of the breaks, their timing, and their duration. Only a few studies specifically focus on the activity during those breaks. However, it was found that break activities during the workday affect emotions and their recovery process, showing that enjoyable and restful activities provide greater recovery \cite{Trougakos2008}. 

When considering further research, one can find a relatively clear commonality: To understand the recovery process, one needs to consider the activity conducted during a break, not depending on the break type. Generally, as already shown in the recovery research, activities with low demand tend to have greater recovery and activities which fail to reduce the demand tend to have negative consequences. Especially for with-in-day breaks, it is often the case that employees can't emotionally detach themselves from work and therefore struggle to recover during breaks. In the case of vacations or weekends, it is often easier for employees.



\chapter{Related Work}
%-	Das «theoretical background» chapter nennen wir meist “Related Work”, aber solche Details können wir dann später anschauen, wenn auch klarer ist, ob es noch einen theoretischen Hintergrund braucht. Hier wird dann sicher auch ein Kapitel wichtig sein zum Thema wie Wissensarbeiter:innen arbeiten, und weshalb Pausen wichtig/gut sind, und weshalb es schwierig ist, diese zu Timen, Managen, und geeignete Aktivitäten zu finden. Dh. Du kannst dann in den Subkapiteln (detaillierter als in der Intro) aufzeigen, was es schon gibt, und wo dein Ansatz anknüpft bzw. was neu ist
\begin{comment} 
- Overview of previous research on knowledge worker motivation and breaks
- Relevant theories and concepts related to knowledge worker motivation and breaks
- Previous studies and research works on the impact of breaks on productivity and well-being
\end{comment}
To understand existing work in the area of work, the next chapter will examine the importance of breaks for knowledge workers and the promotion of breaks in their work environment. Additionally, already existing studies on the characteristics of effective breaks will be discussed.

\section{The Importance of Breaks for Knowledge Workers}
\begin{comment} 
Preventive breaks can help to reduce phisical disconfort, but most of the time knowledge worker only take breaks when it is alreaddy to late. cite Frequent short breaks...

Also add something to work cultur
\end{comment}

Knowledge work has become increasingly important in today's world. As automated machines take over routine and manual labour, the work of knowledge workers has become a critical factor in the economy \cite{Oskarsdottir.2022}. As noted by Oskarsdottir et al. \cite{Oskarsdottir.2022}, knowledge workers' well-being and personal resources are critical factors in their productivity. They divide the related factors into two groups: Internal factors, which include well-being and personal resources, and external factors, such as salary, job design, organization and culture.
In this thesis, we are only concerned with internal factors, specifically knowledge workers' well-being and personal resources.

One of the most effective ways to promote physical and mental health is to incorporate small positive habits into daily life \cite{Taylor.2005}. Since many knowledge workers spend a large portion of their day at work, it can be helpful to use the time they spend at work to establish small positive habits. Most knowledge workers face a highly fragmented workload and many commitments, which can lead to stress and dissatisfaction. Several studies \cite{Largo-Wight.2017, KimS.ParkY.&Niu.2017} have shown that regular breaks can significantly reduce work-related stress. In addition, frequent breaks can reduce physical discomfort \cite{Waongenngarm.2018} and help stay focused \cite{Packer.2021}, thus keeping productivity high throughout the workday.
As work requires energy and effort, it drains the energy level of knowledge workers and can influence their physical and mental well-being. After a certain period, it is necessary to recharge the resources used for accomplishing the work tasks \cite{Sonnentag.2006}. For most knowledge workers, this recovery process only starts each day after working hours. As shown in the background, weekends and end-day recovery are essential for the well-being of each person. Still, with-in-work breaks can help knowledge workers proactively keep their resources high and prevent the depletion state, which needs even longer breaks to recharge. They are also beneficial for job performance as they can increase the assigned focus to each task \cite{Trougakos.2009}. 


\subsection{Promoting breaks in workspaces}
To incorporate with-in-day breaks, knowledge workers need to have a workspace which enables their employees to take breaks. One key factor for a beneficial break environment in the workspace is high job control. Especially in the area of knowledge workers, which are involved in several projects with different agendas and deadlines, it is essential to have flexible schedule breaks, and restorative phases \cite{Trougakos.2009}. As discussed in the background, knowledge workers mostly have high job control, leading to the flexibility to take breaks and the risk of overworking. Having a tool supporting knowledge workers in maintaining their flexibility while engaging in healthy break routines is essential. Some studies, such as that of Kaur et al. \cite{Kaur.2020}, use effect, job activity, and task data to examine which moments are opportune for knowledge workers to take a break. They can predict the right time with 86\% accuracy. However, other, more commercial tools support break reminders and are already widely used \cite{Alghamdi.2020}, for example, the Pomodoro technique \cite{Cirillo.2006} or the 20-20-20 rule \cite{Min.2019}. Some of the already existing solutions and studies will be discussed in more detail in the next section.
 


\begin{comment} 
Desired benefits were shared, but subjective
When asked to explain what makes a break good and/or
successful, 112 respondents (76%) referred to a desired mental
state upon returning to work. Examples include breaks that
“cleared my mind”, where they returned “refreshed”,
“refocused”, “relaxed”, or “recharged”, or that the break
“lowered my stress.” An overall theme was that workers return
from a successful break feeling ready to work. 

Kaur et al. \cite{Kaur.2020} explored the finding of opportune moments for knowledge workers to take a break, using affect, workstation activity and task data. Their study demonstrated a 86\% accuracy of the approach for predicting opportune moments. In addition, several commercial approaches and tools that support reminders for taking breaks exist and are already widely used \cite{Alghamdi.2020}, for example the Pomodoro technique \cite{Cirillo.2006} or the 20-20-20 rule \cite{Min.2019}. Packer et al. \cite{Packer.2021} showed that micro-breaks have a restorative effect on well-being and focus.  However, finding the timing of the micro-break is essential as shown by Kaur et al. \cite{Kaur.2020}. Nevertheless, different studies \cite{KimS.ParkY.&Niu.2017} \cite{Berman.2007} have demonstrated that there is no one "perfect" break schedule that suits everyone. While some people prefer and benefit more from having regular short breaks to the coffee machine, others prefer to go for a walk during their lunch break. Therefore, finding opportune moments for a good break and knowing how best to spend them remains a challenge. 
\end{comment}




\section{Existing Studies on the Characteristics of Effective Breaks}
Important research was already done by many studies in the area of breaks. The most relevant for this thesis will be discussed to get a common understanding on different approaches.

\subsection{Timing of breaks}
% - what have other studies already shown
% - what is their approach
% - what can I learn in finding good time slots for breaks
% - Are there different types of people with different timing breaks
% - What are the factors influencing the timing? (input variables?)
Most of the studies done are looking into the break timing and duration of a break. A solution to find opportune moments for breaks at work was done by Kaur et al. \cite{Kaur.2020}. Using effect, workstation data and task information to conduct whether it is an opportune moment for a break or a task switch. Their accuracy timing is up to 86\%. However, this study only looks at the time when a break should be started but not the duration nor the activity even though many studies have shown that the activities are key for a effective break. They also differ between taking a break and switching tasks. When a person is stuck at a problem but the attention level is still high enough, they propose a task switch instead of a break.
BreakSense
TimeForBreak

\subsection{Duration of the break}
- what have other studies already shown
- what is their approach
- what can I learn in finding good time slots for breaks
- Are there different types of people with different timing breaks
- What are the factors influencing the duration? (input variables?)
- Is it correlated with the timing?

\subsection{Activity during the break}
- what have other studies already shown
- what is their approach
- Are there overall good activities? or is it personal
- Are there different types of people with different needs for activities?
- What are the factors influencing the activities? (input variables?

Although there are already quite some studies on the timing and duration of a break there is surprisingly less work on activities during a break. The study of Sonntag et al. \cite{Sonnentag.2001} they are looking at two approaches to study the recovery process.

One approach focuses on the \textbf{activity during a break}. As defined by Trougakos and Hideg \cite{Trougakos.2009}, activities that consume personal sources are called chores, leaving the person with fewer personal resources than before the activity was practised. On the other hand, activities that restore these sources are called respite breaks \cite{Trougakos.2009}. Whether an activity restores or consumes resources depends on the person. \textbf{Individual differences} influence the experience and appraisal of the activities. One individual factor is the extraversion of a person. The definition of extraversion by McCrae and John \cite{McCrae.1992} states that it is the degree to which a person is talkative, assertive and energetic. Highly extroverted individuals may recharge their personal resources while engaging in social activities, such as eating lunch with other people. While introverted individuals like to label such an activity as a chore activity. Such activities drain more energy from them than they can recharge. Introverted people may benefit more from having lunch by themselves, enabling them to recharge their social energy and therefore restore their personal resources. Another interesting point is that previous research could find that people with a higher extraversion compared to people with a lower extraversion have a more positive appraisal tendency \cite{Gallagher.1990, Hemenover.1996}, leading to a higher coping ability \cite{Penley.2002}. Additionally, they experience more happy emotions which helps them to daily better with daily stressors \cite{Gallagher.1990}. Although it is eventually more difficult for introverted people to regain their personal resources, past research could show that individuals mostly stay in occupations and organisations which fit their personalities and therefore enable introverted people to recharge their personal resources \cite{Johansson.1970, Schaubroeck.1998}.
Thus, individual differences play a significant role when looking into the activities.
An other important role is the enjoyability of the activity. Enganing in activities that the person does enjoy could lead to positive experience. On the other hand, needing alot of motivation to engange in a activity which is not so enjoyable causes a decrease of the resources. Therefore finding enjoyable activities has a high value, as there is barly resources needed to do them, but they recharge a lot of resources.


The other approach focuses on \textbf{psychological experiences}.


The study of Bloom et al. \cite{Bloom.2014} studied different activities and their benefit for knowledge workers. They specifically look at the lunch break as a with-in work break and studied the difference between the lunch break activities: exposure to nature, relaxation and their normal lunch break. This study conduct that .....(han nüt gfunde im paper..)

Epstein et al. \cite{epstein.2016t} conducted a study by building a desktop application which gives insights into the break habits of the user in relation to their productivity using self-reports. They could show that the pre-break productivity had a significant influence on the selected activity. When productivity was rated low, participants tent to take digital breaks as checking mails, whereas when productivity was high they tend to take necessary breaks as going to the bathroom. Their data shows that nessesairy break length are around 5min whereas the break duration for going outside is around 10min is. The duration is therefore in relation with the activity

\chapter{Personalized Break Scheduler}
\section{Approach}

\begin{comment} 
Contrary, studies have shown that employees which have a high job control and no formally scheduled breaks, often do not take any breaks and overwork themselves \cite{McLean.2001}. They fail to take a break when needed, so they only take breaks when their resources are already really low. If they then engage in breaks activities, mostly it is not enough anymore to take short breaks but need an extended break.
- notification is good so that people do not forget to take breaks (take positiv of high job demand but elimit the bad of not having scheduled breaks)
\end{comment}

As the goal of this thesis is to raise awareness of their break habits as well as enable the user to find beneficial activities, the solution uses self-reported input data. 

\section{Requirements}
\section{Design Choices}
\section{Design of Architectur}
%Anschliessend ist es wichtig, nicht gleich mit dem Tool/Implementierung zu beginnen, sondern deinen Approach zu bzw. die Grundkonzepte erklären. D.h. 3.1 und 3.2 könnten wahrscheinlich hier reinpassen; die genauen Konzepte können wir dann aber auch noch besprechen.
The goal of this work is to develop a personalized break planner that supports users in finding the opportune time with a focus on the activities, to develop good/effective break habits. The emphasis is to support the user in analysing his break preferences and not on dictating the "best schedule" and should leave room for flexibility during the day.

\section{Definition of the 3 Key Aspects}
To achieve this goal, 3 key aspects need to be considered:

\subsubsection{Timing of breaks}
As shown through several studies \cite{Largo-Wight.2017} \cite{KimS.ParkY.&Niu.2017}, it is important to schedule regular breaks, but the preferred break interval can vary from user to user. Thus, our approach will initially start with a static break interval, which can be defined and adjusted by the user when using the tool for the first time but also afterwards. This defined interval will be adjusted and further personalized over time, based on the user's preferences. In order to adjust the break interval time, a simple rule-based calculation was defined, which will be explained in the implementation chapter.

It is essential that breaks are not scheduled during meetings. Therefore the time of breaks must be adjusted to the personal calendar of the user, otherwise, he can not participate in any breaks. This feature will be achieved using a synchronisation between the personal calendar and the Break Scheduler. Additionally all planned breaks will be saved in the personal calendar, which helps the user to remember his planned breaks when planning the rest of his work. A positive side effect is also that other members of the company may have access to your calendar and could try not to schedule short-notice meetings into your planned breaks or motivate them to join some of the breaks.


\subsubsection{Duration of the break}
Different techniques such as the Pomodoro technique and the 20-20-20 approach suggest different break durations. However, depending on the time of day and activity, the duration of the break may be different, e.g., a lunch break compared to a coffee break at 9 am. Based on users’ self-reports, the approach shall find more optimal durations for each individual user.  

\subsubsection{Activity during the break}
The activities pursued during the break significantly impact the recreational benefits of the break, as shown in the study by Kim et al \cite{KimS.ParkY.&Niu.2017}. User Preferences and self-reports on their experience with the system will be incorporated to personalise break activity suggestions. 



\section{Rule-based method}
These three aspects will be integrated into an application that uses a rule-based method to decide when and how long breaks should be taken and to suggest activities that can contribute to higher storage on well-being. The system will suggest timely breaks that can be adjusted throughout the day depending on the data related to the three key aspects mentioned above. The system's predictions and suggestions will be validated by a brief self-report from the user before and after each break, as well as a detailed report at the end of the day.
\subsubsection{Timing and duration of breaks}
The rule-based system starts with a static rule, such as the 50-10 rule, which will be defined by the user before the start of the first Break Schedule. Additionally, which it is adjusted depending on the time of day, for example, there should be no breaks at the end of the day, since many users find them inconvenient \cite{KimS.ParkY.&Niu.2017}. The break schedule adapts to the personal calendar, which ensures that no breaks are scheduled during meetings. 
Each morning the user will be asked to rate their current energy rate, which will be a measure of the personal resources available for the rest of the day.  At the end of the day, each user will be asked how much energy he used during this day, which should be a measure to visualise the used personal resources. The term energy was used instead of personal resources as many users find it difficult to rate their personal resources and easier to rate their current energy on a scale from 0 to 100. These morning and evening reflections should enable users to find specific patterns between their resources and well-being. For example: If you use more energy during the day than in the morning, you could get sick after some days as you continuously overstep your personal boundaries. This data is additional self-reflection material which is not yet included in the rule-based system of the break scheduler but will be shown in the dashboard of the tool to enable the user to overview their personal resources and used resources during the day.


In addition to the used energy, the evening report will also ask about the break duration and the break interval. For both, the break duration and the break interval,the user has three selections each: Good, too long or too short. When a break duration or interval got the rating "good", the time will not change. If it was too long or too short the time will be adjusted by a percentage of the existing time. More is explained in the implementation of the rule-based system.


\subsubsection{Activity during the break}
In order to detect "good" break activities, each activity has an experience value. This value should sum up, how the activity was rated in the past. Activities with a higher experience value should be chosen for future breaks. In order to calculate the experience value, the rule-based system  focuses on FOUR main aspects when rating the personal resources of a user:
- energy
- attention
- physical well-being
- activity rating

The energy was chosen because it encompasses different aspects of resources, such as how tired a person is or how motivated.
Attention was chosen because it is a hot topic in many papers, as .... paper...... adds, and is crucial for a working day as a knowledge worker. 
On the other hand, the mind cannot function completely without a healthy body .... paper...... . Therefore, a value for physical well-being had to be added. If an activity is perceived as positive and enjoyable, it is an asset for a good break, as suggested by ..... (Add Paper). Therefore, we also included an emoticon rating of the activity in the final report. These four values are intended to simplify and quantify personal resources. But how to evaluate the energy or attention?

(Emoticon image of activity rating).
Important for enjoyability of activity!


Since each person has an individual and subjective perception and evaluation of these 3 values, the rule-based system does not look at the values per se, but at the difference between the values before and after the break. This removes the individual definition of the scale and focuses only on the change before and after the break. 

Additionally, the experience value of the activity defined by the rule-based system depends only on the difference between before and after the break, the error rate is reduced. All factors that occur during a break should be included in the experience value of that break. This allows the system to cover different areas of a break by only asking about energy, attention, and physical well-being.

This experience value is crucial when selecting activities in the creation of the break schedule. The rule-based system will first select all active activities which are not yet tested and do not have an experience value. After that, he will select always the highest experience value.

Each of these activities has assigned categories. The following categories were defined according to ... add paper....:
- Relaxation
- Body Movement
- Nutrition-intake
- Beverages-intake
- Social
- Cognitive
- Exposure to nature

This will help us with the analysis of the break schedule, as activities can be grouped in different aspects such as body movement or nutrition intake. These categories will also be used by the rule-based system itself. Each of these categories has four assigned values:
- an overall sum of experience value
- a morning sum of experience value
- a lunch sum of experience value
- an afternoon of experience value

When selecting an activity the rule-based system first checks the categories and selects the category with the highest sum for this daytime. Afterwards, he will select the activity with the highest experience value of the activities in this category. Same as with the activity, before the rule-based system does that, he will check for unknown categories which have do not yet have a value. The explicit selection process is described in the implementation.
 
To evaluate the system described above, it will be used by multiple users during one work week. During this preliminary evaluation, the goal will be to learn about the user’s experience of using the system, as well as learn about their self-reported impact on their overall well-being, attention span and feasibility to use such a system in real-world work.

\subsubsection{Initial Break plan}
- How do i generate the initial break plan
- What different people do we have
- How can the user adjust this plan
- What is the theory behind it

The break Scheduler starts with an initial break plan. To achieve this plan, the user will be asked to define his key values, such as the preferred break interval and break duration. The user can also decide to go with the default which is 50-10 as this is recommended by various papers. (add paper here!) He will also be asked to fill out certain settings questions as if he would like to include a lunch break (different duration) or has a sleeping and cooking option. Additionally, he will be shown a list full of predefined break activities from which he can choose what to incorporate and what not. This will help to already cancel outbreak activities which are not feasible or preferred. After this first initial questionnaire for the first time, the break schedule will be defined. All activities are same likely to be chosen for the breaks as there is no data yet. 
\subsubsection{Adapting to user's needs}
- How is the system working?
- what are trigger values?
- what is the weight of the trigger values?
- Why did i choose this concept? (Show theory)

The generated break schedule is synchronized with your chosen calendar interface (Outlook or Google Calendar). This synchronization allows the Break Scheduler to ensure that no breaks are scheduled during meetings. Additionally, all generated breaks will be added to your calendar and a reminder will be displayed for morning and evening reports. A crucial point for the Break Scheduler is the integration of the tool into the user's work environment and the integration with tools already in use, so it does not require much maintenance in the tool itself. For this reason, all adjustments to breaks are made in the personal Outlook or Google calendar, which is synchronized with the Break Scheduler. The user can adjust the start time and duration, or delete the break if new appointments come up during the day.
The goal is to help the user find their best fit and allow manual adjustments to stay flexible throughout the day by working with a single tool, their personal calendar.

Before and after each break, there is a brief self-reflection in which the user assesses their energy and attention levels, as well as their physical well-being. These insights, along with an assessment of the activity, yield the experience value of that break. This value is added to the activity as well as the overall and daily category value. These values will help the planner find good and effective breaks in the next iterations. For more information on the calculation, see the chapter on implementation.



\chapter{Implementation}
%Nun sollte ein (separates) Implementation-Kapitel folgen (ich glaube das hast du in 4.0.1 vorgesehen). Hier kannst du beschreiben, wie du die Key-Konzepte effektiv implementiert hast.
- Overview of the Break Scheduler software
- Description of the software's features and functionalities
- Explanation of the software's design and development process

For the implementation of the Break Planner, it was important that it will run on the user's laptop and that all data can be saved locally on the laptop itself, so only the user has full control over his/her data and data privacy issues can reduce. In order to realise the Break Planner, a Desktop application was preferred that is able to run on different operating systems without a lot of extra effort. Therefore an Electron app was built using typescript. The database of the Break Scheduler is a simple sqlite3 database which will constantly be updated during the use of the application. 

\section{Procedure}
One of the first steps before starting with the implementation was to define the concept of the Break Planner. Supporting the definition of the goals of this tool, the first step was to define the main objectives. During the related work phase, it becomes apparent, that there are a lot of studies focusing on the break time and the duration, but only a few studies are looking at the break activity. Additionally,  whereas there are a lot of different tools to plan static breaks which only consider the timing and duration, no tool could be found, which helps a user to find good breaks beneficial to their energy, attention and physical well-being. Therefore the following 3 Key objectives were defined for the Break Planner:

•	Improve awareness of users on their existing break habits and their effectiveness
•	Enable users to self-reflect on their personal resources, such as attention, energy and physical well-being.
•	Empower users to identify good break activities that restore their personal resources

After the high-level concept was done, the requirements were defined and the customer flow was analyzed (Add pictures here).

One important point for the customer flow was to incorporate the tool into the work environment so that the user does not need to adjust his way of working too much. That is also the reason, why all breaks will be added to the personal calendar of the user and can only be adjusted in this calendar. The user should be able to work with the tools he is anyway using in order to reduce the number of tools used at the same time, which should increase focus. (Add paper here)

After the main customer flow was defined, it was important to get a draft for the rule-based system. Key was to identify important factors for a break schedule using different already existing studies. As already mentioned there are a lot of studies and tools looking at the time and duration of breaks. However, this was not the focus of this thesis, therefore it was decided to use a manually adjustable break interval and duration which be will slightly adjust over the days. The focus of the rule-bases System is the selection of effective breaks. As discussed above a good or effective break is defined as a break which recharges the personal resources of the user. The rule-based system  focus on 3 main aspects when rating the personal resources of a user:
- energy
- attention
- physical well-being




.....

\section{Desktop application}
Dashboard CSS template: https://themewagon.com/themes/star-admin-2-free-bootstrap-4-html5-admin-dashboard-template/ open source

\subsubsection{Rule-Based-System}
As described in the Personalizable Break Scheduler chapter, the rule-based method assigns experience values depending on the ratings of the user before and after the break. For the calculation of the experience value therefore a straightforward sum was used:

... add picture of calculation...

In addition it will be rounded in order to goupe activities with similar experience values. Activities with value 53 should be same likly to be selected as activities with 56. The grouping will take place in a 5er steps.

\chapter{Methodology}
%Dann folgt ein Preliminary Evaluation Chapter: hier beschreibst du zunächst die Methode und anschliessend Resultate
This methodology chapter describes the method and procedure used in this user study to answer the effectiveness of the Break Scheduler application in raising awareness of personal resources and enabling users to learn more about their break habits. This section provides a detailed explanation of the sample selection, study design, data collection procedures, and data analysis methods used to answer the research questions stated in the introduction, enabling the reader to reproduce the findings when needed. The following subchapters provide more information on each aspect of the methodology.

\section{Participants}
\begin{comment} 
- Description of the sample size, criteria for inclusion/exclusion, and demographics of the participants
\end{comment}
 The study includes a sample of 10-15 knowledge workers with different backgrounds (students and employees) working in various industries, ranging from IT, chemistry, and mathematics to education, aged 20-40, living and working in the German part of Switzerland. The recruiting was done by word-of-mouth approach. Each participant was asked to participate with his/her main work laptop/computer as the application needed to be incorporated into their daily work. As the Break Scheduler is a desktop application, the participants needed to be able to install this application on their device, which already pre-eliminated people who were not allowed by their employers to download specific applications on their work machines. Within this sample group, all three main operating systems ( macOS, Windows and Linux) were represented, eliminating the exclusion risks of certain groups of knowledge workers.

\section{Study Design}
\begin{comment} 
- Explanation of the experimental design, including control and intervention groups, pre- and post-tests, and other relevant details
\end{comment}

This experimental study tests the Break Scheduler application for its effectiveness. To achieve this, the control conditions of each participant were gathered by a pre-intervention Questionnaire, which asked about demographic data as well as their awareness regarding their personal resources and daily break habits. This with-in-study design ensures each user acts as their own control, enabling a pre-and post-intervention evaluation to answer the research questions. The selection of participants was done by a word-of-mouth approach in combination with a snowball sampling method and eliminated participants who could not install the application on their work machine. The participants in this study were voluntary and could be stopped at any time. Other than insights into their results, no monetary or non-cash deposit was offered.

This study conducts three phases: Pre-Intervention, Intervention and Post-Intervention phase. 

The Pre-Intervention phase provides insights into the participant's demographical data and their awareness of their personal resources and break habits. These phases refer to the control conditions to define the pre-intervention state of each participant. The participants are asked to complete the \textbf{pre-questionnaire} in the form of an online survey with questions about their existing break habits and awareness of personal resources. This questionnaire should represent the baseline of the user's awareness of personal resources and break habits with the influence of the Break Scheduler. The survey also includes demographic questions (e.g., age, gender, occupation, work background) to control for potential confounding variables. This pre-intervention questionnaire is attached to this thesis.

After completing the pre-intervention questionnaire, the intervention phases start, where the researcher will guide the participants through installing the \textbf{break scheduler application}. Subsequently, they will be asked to use the break scheduler for five consecutive workdays during their regular work. The break scheduler app includes one morning and evening self-reflection questionnaire and a short pre-and post-break self-reflection questionnaire. Some self-reflections are part of the application, while others are used to collect qualitative data for the preliminary user evaluation. All collected data is stored in the app's local database. All data produced during the intervention will only be stored locally on the user's machine and can only be accessed by them. However, as part of the consent form, all participants will sign that they will grant access to their database file after the intervention for study purposes.

After completing the intervention phase, participants will be asked to answer the \textbf{post-questionnaire}, where they are asked about their experience with the break scheduler app, their learning, and potential ideas for improvements. They are also asked about the advantages and disadvantages of the implementation. In this post-intervention state, participants will rate their awareness of their resources and break habits again and will rate the impact of the Break Scheduler.

The pre-and post-intervention will be compared in an evaluation analysis to answer the research question.


\section{Procedure}
\begin{comment} 
- Step-by-step description of the procedure followed in conducting the study, including data collection, administration of the recharge breaks, and any other relevant details
\end{comment}
To enable the reader to replicate the conducted results, all step-by-step tasks are stated and further elaborated.

\textbf{Participants:} Participants were recruited through a word-of-mouth approach and a snowball sampling method by asking friends and acquaintances if they would be willing to participate in the study. A total of 18 knowledge workers agreed to participate. Out of these 18, two participants were used in a pre-evaluation testing phase which eliminated them from participating in the study. Four of them needed to cancel the study during the intervention phase due to personal reasons. It led to a total of 12 participants fulfilling all three mandatory stages.

\textbf{Pre-Questionnaire:} Before the intervention, all participants were sent an online survey to collect demographic data and information about their awareness of their personal resources and break habits. This study was filled out before the intervention started. All detailed questions of the Pre-Questionnaire can be found in the appendix.

\textbf{Onboarding:} At the start of the Intervention phase, a short onboarding session was held with each participant to install the Break Scheduler application and provide a brief introduction to its use. Participants were shown how to connect the Break Scheduler with the calendar for synchronisation and how to fill out the different self-reports and the applications' main features.

\textbf{Intervention:} Participants were asked to use the Break Scheduler application for a period of one to two weeks, during which time they were encouraged to log their personal resources with a morning and evening report as well as their break activity and their well-being before and after the break measured in an energy, attention capacity and physical well-being rating. They were also asked to rate the activity after each break. Each evening when filling out the evening report, the participant uses the application to schedule breaks for the new day. Participants were able to contact the researcher if they had any questions or problems with the application.

\textbf{Post-Questionnaire:} After the intervention phase, participants completed a post-questionnaire that included questions about their awareness level after the intervention, as well as the impact of the Break Scheduler and their implementation of the tool. All detailed questions of the Post-Questionnaire can be found in the appendix.

\section{Data Analysis}
\begin{comment} 
- Explanation of the statistical methods used to analyze the data, including descriptive statistics, inferential statistics, and any other relevant techniques

The data analysis section of your study is where you describe the methods you used to analyze the data you collected. This section should include the following elements:

Data Cleaning: Describe any steps you took to clean the data, such as removing missing values, checking for outliers, or transforming variables.

Descriptive Statistics: Provide summary statistics for the demographic and pre- and post-intervention data, including means, standard deviations, frequencies, and ranges.

Inferential Statistics: Describe the inferential statistical methods you used to analyze the data and test your hypotheses. For example, if you were comparing the mean level of personal resource recharge between the intervention and control groups, you might use a paired t-test or a repeated measures ANOVA.

Results: Present the results of your statistical analysis in a clear and concise manner, including tables, figures, and effect sizes. Interpret the results in the context of your study, including any significant findings and their implications.

Limitations: Discuss any limitations or limitations of your study, including sources of bias, limitations of the data, or limitations of the statistical methods used.

Conclusion: Summarize the main findings of the study and provide implications for future research.

This section should be written in a clear, concise manner that is accessible to readers who may not have a strong background in statistics. Any statistical tests should be accompanied by an explanation of the test and its assumptions, as well as the results of the test and their implications.
\end{comment}


5.	Measurements:
5.1.	Quantitative Data Collection:
Rule-based System:
The rule bases system's values (e.g., experience values, break interval and duration, etc.) will be saved in the Database file. This file will be manually sent back to us by the user for this study. These values will indicate which activities and categories will be ranked best. These activities and categories can be compared to the qualitative data the user is providing. 
Report files:
From all the reports (e.g., morning and evening reports, as well as before and after break reports) saved in the database, we know how many reports were filled out per day, indicating how actively a user used the tool and how many daily breaks they did. This number can be compared with the number of the pre-intervention questionnaire. 
Settings:
During the break schedule, the user will experiment with their break interval and duration and eventually find their preferred duration and interval timing. It will be interesting to compare the numbers of the break schedule with their guessed break interval and break duration in the pre-intervention. These numbers can also vary from person to person.

5.2.	Qualitative Data Collected
Daily Self-Reflections
Daily self-reflection should help the user to reflect on their day and compare their energy level to their used energy during the day. Each person has a limited amount of personal resources which will be used during the day (e.g. resources for attention, decision making or just social interactions). These morning and evening reflections should enable users to find specific patterns between their resources and well-being. For example: If you use more energy during the day than in the morning, you could get sick after some days as you continuously overstep your personal boundaries. This data is additional self-reflection material which is not yet included in the rule-based system of the break scheduler. The user quantifies his/her energy level in the morning and activities during the day (rating 0-100). The energy minus the activity value will show the difference and help users better understand and manage their personal resources. There are also free text boxes where the user can enter notes about their mental or physical well-being which should help them self-reflect more in-depth and give more insight into their day. This also helps to detect patterns for triggers for physical complaints, for example, headaches. This data will be visualized on the overview page for the user to understand their energy household better. Additionally, it will be used in this study to see if there is any energy change during the week. In the best case, the morning energy will rise, and the used energy will be smaller, which means the user has increased his personal resource and decreased the usage during the day. 
On the other hand, two questions in the evening report reflect on the break interval and the duration. The user can select between the ratings "good", "too long", and "too short". This rating will adjust the break interval and duration for the following break schedule.

Pre/Post-Break Self-Reflections
These short self-reflections should rate the break and are used by the rule-based system to determine the suggested activity. In the pre-break reflection, the user is asked about their energy, attention level, and physical well-being. The same three questions and an overall emoticon rating about the break activity will be shown in the post-break self-reflection. If the difference in the energy, attention and physical level is positive, the break can be considered "good" (resources are recharged). If the difference is negative, personal resources are drained, which is why this break is not considered a "good" one. These values will be added to the sum of the experience value of each activity and the respective categories of the activity. The categories and activities with the highest experience values will be selected for the next generation of the break schedule. In addition to the suggested activities, the break schedule also selects a random activity. Which activity is selected gives exciting insights for the user study and ensures that low-ranked activities can be chosen again and achieve higher experience values. The user quantities his/her energy, attention and physical well-being before and after the break (rating 0-100). The difference between the after and before will rate whether it was a "good" are a "bad" break.
Additionally, the activity will be rated by an emoticon rating, quantified (-5 to 5) and used by the rule-based system to quantify the break activity. There are also free text boxes where the user can enter notes about their mental or physical well-being which should help them self-reflect more in-depth and raise awareness about their well-being. This data is not used by the break scheduler and is only for self-reflection purposes.

Pre-Intervention questionnaire ()
In the Pre-Intervention Questionnaire, it is crucial to understand the demographical background of the user and their usual break habits. This data is needed to compare their break habits, including the break scheduler and to control for potential confounding variables. Below is a summary of the chapters, and the entire questionnaire is attached.
1.	Demographical data
2.	Awareness of personal resources
3.	Break habits
The questionnaire can be found at the end of this thesis.

Post-Intervention Questionnaire (no content yet, TBD)
The post-intervention questionnaire will give additional information to answer the RQs regarding the impact and the tool. The questions are categorized according to the three RQs : 
1.	Awareness
2.	Impact
3.	Tool
The questionnaire can be found at the end of this thesis.

	
Analysis
A lot of the analysis will be done by comparing the pre-and post-intervention questionnaires. For example, their awareness will be compared to the awareness at the end of the study. Additional qualitative data can enhance the user's answers as to which activity or break interval was preferred the most. In the best case, a correlation can be found between the energy during the day and the ratings of the breaks. 
 
Limitations 
This study has several potential limitations, including self-reflection bias (since participants will be completing the survey and reporting their own personal and physical resources and maybe want to present themselves better for the study) and selection bias due to the self-selection and word-to-mouth approach of the researcher. Additionally, the sample size is too small to make a conclusive statement for a wider group of people. However, some trends can give insight.

\chapter{Results}
\begin{comment} 
Summary of the findings from the study, including any significant effects of the recharge breaks on personal resource recharge
\end{comment}

\section{RQ1 Awareness}
\section{RQ2 Impact}
\section{RQ3 Tool}


  

\chapter{Conclusion}
%
- short summary
- conclusion
    - show limitations

\chapter{Futur work}

Another factor influencing the break interval is the user's sleep schedule. As shown in several studies \cite{Rosekind.2010} \cite{Gingerich.2017} \cite{Choi.2018}, a good night's sleep  can help to stay focused longer, which means taking fewer breaks. 
It is also important to ensure that the system does not suggest breaks to the user during periods of high concentration and focus when suggesting a break, to avoid interruptions and dissatisfaction with the system. Thus, various computer interaction trackers will be used to detect the state of the user and the personal calendar will be considered, to avoid scheduling breaks during meetings. 
To account for breaks that are less predictable, such as when a user is very stressed, the approach shall consider suggesting “emergency breaks” when sensing a higher stress state from an increased heart rate \cite{Hjortskov.2004}. 
Finally, spontaneous breaks that were not suggested by the system must also be detected to help adjust and personalize the timing of future breaks. Detecting such breaks could be accomplished through pedometer data, computer interaction trackers or self-reports. 



% 
% \subsubsection{Subsubsection}
% \fig[.5\textwidth]{logos/logo_hasel}{Our logo}{logo}

% \subsection{Subsection}
% %
% \paragraph{Paragraph.} Always with a point.

% \begin{lstlisting}[caption=An example code snippet]
% /**
%  * Javadoc comment
%  */
% public class Foo {
% 	// line comment
% 	public void bar(int number) {
% 		if (number < 0) {
% 			return; /* block comment */
% 		}
% 	}
% }
% \end{lstlisting}

\appendix
\chapter{First Appendix}
\chapter{Second Appendix}

\backmatter
\bibliographystyle{alpha}
\bibliography{hasel_thesis/references}



\end{document}
