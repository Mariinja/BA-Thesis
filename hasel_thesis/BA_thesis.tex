\documentclass{hasel_thesis}

\thesisType{Bachelor}
\date{\today}
\title{Motivating effective breaks for knowledge workers with Break  Scheduler }
\subtitle{Bachelor Thesis}
\author{Marinja Principe}
\home{Niederhasli} % Geburtsort
\country{Switzerland}
\legi{18-740-910}
\prof{Prof. Dr. Thomas Fritz}
\assistent{André Meyer}
\email{marinja.principe@uzh.ch}
\url{<url if available>}
\begindate{19.09.2022}
\enddate{19.03.2023}


\begin{document}

\maketitle

\frontmatter

\begin{acknowledgements}
\end{acknowledgements}

\begin{abstract}
This is an abstract.
\end{abstract}

\begin{Summary}
This is the Summary.
\end{Summary}
    

\tableofcontents
\listoffigures
\listoftables
\lstlistoflistings

\mainmatter
\chapter{Introduction}
One of the most effective ways to promote physical and mental health is to incorporate small positive habits  into everyday life \cite{Taylor.2005}. Since many adults spend a large portion of their day at work, it can be valuable to consider the time they spend at work for incorporating small positive habits. Most knowledge workers face a highly fragmented workload and many commitments, which can lead to stress and dissatisfaction. Different studies \cite{Largo-Wight.2017},\cite{KimS.ParkY.&Niu.2017} have shown that regular breaks can significantly reduce work-related stress.  In addition, regular breaks may reduce physical discomfort \cite{Waongenngarm.2018} but also help to stay concentrated \cite{Ariga.2011} \cite{Bloom.2014} and, thus, keep productivity high throughout the workday. 
But when, how long and with which activities should a good and effective break be spent? 

Definition of a Work break from \cite{Trougakos.2009}: Time period in which work-related tasks are not expected nor required to be done.
Different kind of work break: Vactions, Weekends, end of day break and during work break (Micro-breaks).
In the next chapter, the different break times will be evaluated.


- Importance of breaks
% %- why do we need breaks
% - How does it influence knowledge workers
% - Examples of papers which have shown breaks are important

Vacations \cite{Trougakos.2009}:
Reduce stress levels and lower burnout levels \cite{Westman.2001}. But it is shown that these effects fade-out within days or weeks, which indicates that actions have only short-lived positive effects on stress levels \cite{Fritz.2006}.

Weekends and end of day breaks
There are not yet a lot studies specific targeting weekends, but the few available indicate that the absents of non-work related time nor social activities can lead to lower well-being and even burnouts \cite{Fritz.2005}.
In generall they can help the recovery and well-being, when used for restful and stimulating activies with a low demand.

During work days break:
Goes back to \cite{Mayo.1933} which did a very general form  of during work days break study.


Some studies investigated how to find the right time for a break to increase productivity and well-being. Kaur et al. \cite{Kaur.2020} explored the finding of opportune moments for knowledge workers to take a break, using affect, workstation activity and task data. Their study demonstrated a 86\% accuracy of the approach for predicting opportune moments. In addition, several commercial approaches and tools that support reminders for taking breaks exist and are already widely used \cite{Alghamdi.2020}, for example the Pomodoro technique \cite{Cirillo.2006} or the 20-20-20 rule \cite{Min.2019}. Packer et al. \cite{Packer.2021} showed that micro-breaks have a restorative effect on well-being and focus and, in comparison to taking a vacation, micro-breaks were shown stronger effects on regeneration.  However, finding the timing of the micro-break is essential as shown by Kaur et al. \cite{Kaur.2020}. Nevertheless, different studies \cite{KimS.ParkY.&Niu.2017} \cite{Berman.2007} have demonstrated that there is no one "perfect" break schedule that suits everyone. While some people prefer and benefit more from having regular short breaks to the coffee machine, others prefer to go for a walk during their lunch break. Therefore, finding opportune moments for a good break and knowing how best to spend them remains a challenge. 

- Good / Effective breaks
 % - Definition of a break
 % - what is a good break what is a bad break?
 % - How can we influence breaks (timing, duration and activities)
 %    - show paper with different activities and show that it is important how to spend break time

 % - what is already done in relation to breaks?
 %    - speak about google tool, activity study etc.

 % - show there is not tool incorporating all 3 aspects
 % - Explain the goal and approach of this thesis

 Definition of Personal sources: "workers have a limited amount of ‘‘personal resources’’ that
allow them to complete the variety of taxing activities they engage in
throughout the day." (Baumeister et all 1998, from momentory recovery)
People with fewer sources are more vulnerable to resource loss as abundance of resources also drains resources.
Definition of recorvery: "Recovery occurs by temporarily
removing the demands the employees face"

 Besides the timing, it is further important to consider how the break is spent and how long the break should be. Good ways of spending breaks are to spend time in nature, relax or perform physical activity \cite{Bloom.2014} \cite{Largo-Wight.2017}.  As shown by Sooyeol et al \cite{KimS.ParkY.&Niu.2017}, there are different types of breaks. They are divided into relaxation, nutritional, social, and cognitive micro-breaks. Each of these groups has a different goal and therefore a different use. For example, stretching exercises and naps can contribute to relaxation, while short conversations with work colleagues about non-work related topics improve social support. These activities also affect the duration of a break. But it's not just the activity that affects the duration of a break. Different techniques, such as the Pomodoro technique or the 20-20-20 technique, recommend different break durations.

Overall, existing approaches considered static break schedules and didn’t let users personalize the break timing to individual preferences and needs (e.g. \cite{Henning.1997} \cite{Cooley.2013}), or didn’t support users in identifying how to best spend their breaks, to facilitate recharging and recovering (e.g. \cite{Kaur.2020}). 

 

\chapter{Theoretical background}
\section{Timing of breaks}
- what have other studies already shown
- what is their approach
- what can I learn in finding good time slots for breaks
- Are there different types of people with different timing breaks
- What are the factors influencing the timing? (input variables?)

\section{Duration of the break}
- what have other studies already shown
- what is their approach
- what can I learn in finding good time slots for breaks
- Are there different types of people with different timing breaks
- What are the factors influencing the duration? (input variables?)
- Is it correlated with the timing?

\section{Activity during the break}
- what have other studies already shown
- what is their approach
- Are there overall good activities? or is it personal
- Are there different types of people with different needs for activities?
- What are the factors influencing the activities? (input variables?)

\chapter{Personalized Break Planner}
The goal of this work is to develop a personalized break planner that supports users in finding the opportune   time for an activity and to decide on an activity in order to develop good/effective break habits. 

\section{Definition of the 3 Key Aspects}
To achieve this goal, 3 key aspects need to be considered:

\subsubsection{Timing of breaks}
As shown through several studies \cite{Largo-Wight.2017} \cite{KimS.ParkY.&Niu.2017}, it is important to schedule regular breaks, but different intervals should be considered depending on the time of day \cite{KimS.ParkY.&Niu.2017} and the individual. Thus, our approach will initially start with a static break interval that will be adjusted and further personalized over time, based on the user's preferences, calendar and the time of day. Another factor influencing the break interval is the user's sleep schedule. As shown in several studies \cite{Rosekind.2010} \cite{Gingerich.2017} \cite{Choi.2018}, a good night's sleep  can help to stay focused longer, which means taking fewer breaks. 
It is also important to ensure that the system does not suggest breaks to the user during periods of high concentration and focus when suggesting a break, to avoid interruptions and dissatisfaction with the system. Thus, various computer interaction trackers will be used to detect the state of the user and the personal calendar will be considered, to avoid scheduling breaks during meetings. 
To account for breaks that are less predictable, such as when a user is very stressed, the approach shall consider suggesting “emergency breaks” when sensing a higher stress state from an increased heart rate \cite{Hjortskov.2004}. 
Finally, spontaneous breaks that were not suggested by the system must also be detected to help adjust and personalize the timing of future breaks. Detecting such breaks could be accomplished through pedometer data, computer interaction trackers or self-reports. 

\subsubsection{Duration of the break}
Different techniques such as the Pomodoro technique and the 20-20-20 approach suggest different break durations. However, depending on the time of day and activity, the duration of the break may be different, e.g., a lunch break compared to a coffee break at 9 am. Based on users’ self-reports, the approach shall find more optimal durations for each individual user.  

\subsubsection{Activity during the break}
The activities pursued during the break significantly impact the recreational benefits of the break, as shown in the study by Kim et al \cite{KimS.ParkY.&Niu.2017}. User preferences and self-reports on their experience with the system will be incorporated to personalise break activity suggestions. 



\section{Rule-based method}
These three aspects will be integrated into a software system that uses a rule-based method to decide when and how long breaks should be taken, and to suggest activities that can contribute to a higher restorage on well-being. The system will suggest timely breaks that can be adjusted throughout the day depending on the data related to the three key aspects mentioned above. The system's predictions and suggestions will be validated by a brief self-report from the user before and after each break, as well as a detailed report at the end of the day.
The rule-based system starts with a static rule, such as the 50-10 rule, which is adjusted depending on the time of day to make an initial suggestion for the break schedule. For example, there should be no breaks at the end of the day, since many users find them inconvenient \cite{KimS.ParkY.&Niu.2017}. The break schedule is adjusted to the personal calendar, which ensures that no breaks are scheduled during meetings. As mentioned earlier, sleep schedule can also influence break intervals. In the short self-reports, the user will quantify the quality of the break. If the break is perceived as useless, for example, if it was too long, or too short after the last break. The rule-based system will then react accordingly and shorten the break duration or extend the break intervals. However, not only user input, but also data related to the three key concepts: Timing, Duration and Activity, have a significant impact on the rule-based system.
Compared to previous works, the system aims to dynamically adjust and personalize the break schedule, based on user-feedback and sensing, and suggests recreational activities for spending the break. 
To evaluate the system described above, it will be used in situ by 3 users during one work week. During this preliminary evaluation, the goal will be to learn about the user’s experience of using the system, as well as learn about their self-reported impact on their overall well-being, productivity and feasibility to use such a system in real-world work.

\subsubsection{Initial Break plan}
- How do i generate the initial break plan
- What different people do we have
- How can the user adjust this plan
- What is the theory behind it
\subsubsection{Adapting to user's needs}
- How is the system working?
- what are trigger values?
- what is the weight of the trigger values?
- Why did i choose this concept? (Show theory)

\section{Implementation}
\subsubsection{Frontend}
\subsubsection{Backend}
\chapter{Results}
\subsection{Break Planner}
\subsection{User Study}
\subsection{Evaluation}
\chapter{Conclusion}
- short summary
- conclusion
    - show limitations
- futur work



% 
% \subsubsection{Subsubsection}
% \fig[.5\textwidth]{logos/logo_hasel}{Our logo}{logo}

% \subsection{Subsection}
% %
% \paragraph{Paragraph.} Always with a point.

% \begin{lstlisting}[caption=An example code snippet]
% /**
%  * Javadoc comment
%  */
% public class Foo {
% 	// line comment
% 	public void bar(int number) {
% 		if (number < 0) {
% 			return; /* block comment */
% 		}
% 	}
% }
% \end{lstlisting}

\appendix
\chapter{First Appendix}
\chapter{Second Appendix}

\backmatter
\bibliographystyle{alpha}
\bibliography{hasel_thesis/references}



\end{document}
