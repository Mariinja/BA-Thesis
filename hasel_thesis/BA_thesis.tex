\documentclass{hasel_thesis}

\thesisType{Bachelor}
\date{\today}
\title{Motivating effective breaks for knowledge workers with Break  Scheduler }
\subtitle{Bachelor Thesis}
\author{Marinja Principe}
\home{Niederhasli} % Geburtsort
\country{Switzerland}
\legi{18-740-910}
\prof{Prof. Dr. Thomas Fritz}
\assistent{Dr. André N. Meyer}
\email{marinja.principe@uzh.ch}
\url{<url if available>}
\begindate{19.09.2022}
\enddate{19.03.2023}


\begin{document}

\maketitle

\frontmatter

\begin{acknowledgements}
\end{acknowledgements}

\begin{abstract}
This is an abstract.
\end{abstract}

\begin{Summary}
This is the Summary.
\end{Summary}
    

\tableofcontents
\listoffigures
\listoftables
\lstlistoflistings

\mainmatter
\chapter{Introduction}
%Introduction: genau, hier kommt die Motivation und Zusammenfassung vergleichbarer Forschung, um den Gap zu zeigen, anschliessend eine Zusammenfassung deiner Arbeit, Evaluation und Erkenntnisse, sowie Zusammenfassung der Contributions 

%- Importance of breaks
% %- why do we need breaks
% - How does it influence knowledge workers
% - Examples of papers which have shown breaks are important
%Why do we need breaks?
%What is a break? (vacatation, ..





Most knowledge workers face a highly fragmented workload and many commitments, which can lead to stress and dissatisfaction. This can lead to burnout or problems for the organization \cite{Elkin.1990}. Therefore, it is important to learn more about the recovery process of employees. Various studies \cite{Largo-Wight.2017},\cite{KimS.ParkY.&Niu.2017} have shown that regular breaks can significantly reduce work-related stress.  In addition, regular breaks can reduce physical discomfort \cite{Waongenngarm.2018}, but also help people stay focused \cite{Ariga.2011} \cite{Bloom.2014} and thus keep productivity high throughout the workday. As discussed by \cite{Trougakos.2009}, there are different types of breaks. They are differentiated into vacation breaks, breaks between work days (such as weekends), and breaks within work days. While the effect of vacation breaks are usually short-lived, short, regular breaks within the workday have greater value for resting a worker's personal and physiological resources.
Some studies investigated how to find the right time for a break to increase productivity and well-being. Kaur et al. \cite{Kaur.2020} explored the finding of opportune moments for knowledge workers to take a break, using affect, workstation activity and task data. Their study demonstrated a 86\% accuracy of the approach for predicting opportune moments. In addition, several commercial approaches and tools that support reminders for taking breaks exist and are already widely used \cite{Alghamdi.2020}, for example the Pomodoro technique \cite{Cirillo.2006} or the 20-20-20 rule \cite{Min.2019}. Packer et al. \cite{Packer.2021} showed that micro-breaks have a restorative effect on well-being and focus.  However, finding the timing of the micro-break is essential as shown by Kaur et al. \cite{Kaur.2020}. Nevertheless, different studies \cite{KimS.ParkY.&Niu.2017} \cite{Berman.2007} have demonstrated that there is no one "perfect" break schedule that suits everyone. While some people prefer and benefit more from having regular short breaks to the coffee machine, others prefer to go for a walk during their lunch break. Therefore, finding opportune moments for a good break and knowing how best to spend them remains a challenge. 

Since each person has only a limited amount of personal resources, it is necessary to understand how to use them up and, more importantly, how to recover them. The theory of resource depletion \cite{BaumeisterR.F.BratslavskyE.MuravenM.&TiceD.M..1998} states that each person has only a limited amount of "personal resources" that can be consumed, but also recovered, depending on the activity. Activities that consume resources are so-called chores. On the other hand, activities that restore resources are called respite breaks \cite{Trougakos.2009}. Whether an activity restores or consumes resources varies from person to person. This indicates that break activities are key to a good break. But when, how long, and with what activities should a good and effective break be spent? 


%One of the most effective ways to promote physical and mental health is to incorporate small positive habits  into everyday life \cite{Taylor.2005}. Since many adults spend a large portion of their day at work, it can be valuable to consider the time they spend at work for incorporating small positive habits. Most knowledge workers face a highly fragmented workload and many commitments, which can lead to stress and dissatisfaction. Different studies \cite{Largo-Wight.2017},\cite{KimS.ParkY.&Niu.2017} have shown that regular breaks can significantly reduce work-related stress.  In addition, regular breaks may reduce physical discomfort \cite{Waongenngarm.2018} but also help to stay concentrated \cite{Ariga.2011} \cite{Bloom.2014} and, thus, keep productivity high throughout the workday. 
%But when, how long and with which activities should a good and effective break be spent? 




%- Good / Effective breaks
 % - Definition of a break
 % - what is a good break what is a bad break?
 % - How can we influence breaks (timing, duration and activities)
 %    - show paper with different activities and show that it is important how to spend break time

 % - what is already done in relation to breaks?
 %    - speak about google tool, activity study etc.

 % - show there is not tool incorporating all 3 aspects
 % - Explain the goal and approach of this thesis

Before we look into the definition of a \textbf{good break} it is important to define the term break. \cite{Trougakos.2009} defines a break as a time in which work-related tasks are not expected nor required to be done. In other words, a break is a time when the owner can decide how to spend it, in order to achieve a good break. In this work, we are using the definition of a good break by \cite{Trougakos.2009}. A "good break" is a break that recovers the personal resources of an employee. On the other hand a "bad break" will drain the anyway limited resources. If an activity recovers or drains resources is depending on two main aspects: The amount of effort, which is used to fulfill this activity and whether it is enjoyable for the person. Good ways of spending breaks are to spend time in nature, relax or perform physical activity \cite{Bloom.2014} \cite{Largo-Wight.2017}.  As shown by Sooyeol et al \cite{KimS.ParkY.&Niu.2017}, there are different types of breaks. They are divided into relaxation, nutritional, social, and cognitive micro-breaks. Each of these groups has a different goal and therefore a different use. For example, stretching exercises and naps can contribute to relaxation, while short conversations with work colleagues about non-work related topics improve social support. These activities also affect the duration of a break. But it's not just the activity that affects the duration of a break. Different techniques, such as the Pomodoro technique or the 20-20-20 technique, recommend different break durations.

Overall, existing approaches considered static break schedules and didn’t let users personalize the break timing to individual preferences and needs (e.g. \cite{Henning.1997} \cite{Cooley.2013}), or didn’t support users in identifying how to best spend their breaks, to facilitate recharging and recovering (e.g. \cite{Kaur.2020}). 
The goal of this work is to develop a personalized break planner that supports users in finding the opportune time for an activity and to decide on an activity in order to develop good/effective break habits. 

%
 %Factore for a "good break" \cite{Bloom.2014}
 %- Recovery process
 %- health (physical and mental)
 %- well-being
 %- job performance
 %- creativity

\chapter{Related Work}
%-	Das «theoretical background» chapter nennen wir meist “Related Work”, aber solche Details können wir dann später anschauen, wenn auch klarer ist, ob es noch einen theoretischen Hintergrund braucht. Hier wird dann sicher auch ein Kapitel wichtig sein zum Thema wie Wissensarbeiter:innen arbeiten, und weshalb Pausen wichtig/gut sind, und weshalb es schwierig ist, diese zu Timen, Managen, und geeignete Aktivitäten zu finden. Dh. Du kannst dann in den Subkapiteln (detaillierter als in der Intro) aufzeigen, was es schon gibt, und wo dein Ansatz anknüpft bzw. was neu ist



\subsection{Definition of a break}

A work break is defined by \cite{Trougakos.2009} as a period of time when no work-related tasks are expected or need to be completed. This includes various types of breaks such as vacations, weekends, breaks at the end of the day, and breaks during work, including micro-breaks. Microbreaks are defined as really short breaks, e.g., less than 10 minutes \cite{BennettAndrewA.GabrielAllisonS.CalderwoodCharles.2020}. Several studies have already looked at the benefits of such breaks and how they can contribute to the recovery of personal resources.
The next chapter will examine the different types of breaks in more detail.



\subsubsection{Vacations}
\cite{Lounsbury.1986} define vacation as a period of time "when a person is not actively engaged in his or her work. It is a time when a person is free to pursue other interests, and therefore a time when the work situation may become less important compared to other areas of experience such as family and personal leisure" (p. 393\cite{Lounsbury.1986}). The goal is to regain energy to be more productive during work time. It has been shown by \cite{Westman.2001} that vacations reduce stress and lower burnout levels. Unfortunately, these effects wear off within days or weeks \cite{Westman.2001}. This means that vacations have only a short-term positive effect on stress levels \cite{Fritz.2006}.


\subsubsection{Weekends}
Weekends are mini-vacations between work days that allow workers to personally relax and engage in other non-work related activities. Unfortunately, there are not yet as many studies that specifically look at weekends and their benefits to workers, but the studies that do exist show that the lack of time off work can lead to lower well-being or burnout \cite{Fritz.2005}. Weekends can be used to promote rest and well-being through restorative and stimulating activities suggested with low demand.

\subsubsection{During work day breaks}
The first study of breaks during the working day goes back to \cite{Mayo.1933}, in which a very general form was studied. The best known types of breaks during the day are coffee or lunch breaks. There are more recent studies that look at specific short breaks, also called micro-breaks. \cite{BennettAndrewA.GabrielAllisonS.CalderwoodCharles.2020}. There are studies such as \cite{BennettAndrewA.GabrielAllisonS.CalderwoodCharles.2020} and \cite{Ding.2020} that demonstrate the benefits of such micro-breaks. For example, just 5 minutes of stretching can help combat muscle fatigue that typically occurs after 40 minutes. 

\subsection{Personal resources and Physiological capacities}
To define a "good break", it is important to understand the definition of personal resources and physiological capacities. In the study of \cite{BaumeisterR.F.BratslavskyE.MuravenM.&TiceD.M..1998} they define personal resources as a limited set of sources that enable a person to complete various activities during the day. The amount of resources varies from person to person. People with fewer resources are more prone to resource depletion because abundance also consumes resources. Physiological capacity, on the other hand, is a limited resource beyond which a person is physically unable to continue without recovery. An example would be sleep or food. Since both, personal or physiological sources, are resources which are limited, it is important to know how to restore them.

Restoration is done by eliminating the demand for a period of time. \cite{Trougakos.2009}. According to the definition of resource depletion \cite{BaumeisterR.F.BratslavskyE.MuravenM.&TiceD.M..1998}, activities can consume or restore personal resources. 
Activities that consume resources are called chores. On the other hand, activities that restore resources are called recreational breaks {Trougakos.2009}.
Whether an activity restores or consumes resources varies from person to person. 
This shows that break activities are important factors for a good break.

%  Definition of Personal sources: "workers have a limited amount of ‘‘personal resources’’ that
% allow them to complete the variety of taxing activities they engage in
% throughout the day." (Baumeister et all 1998, from momentory recovery)
% People with fewer sources are more vulnerable to resource loss as abundance of resources also drains resources.
% Definition of recorvery: "Recovery occurs by temporarily
% removing the demands the employees face"

% Definition of Physiological capacities: limit of physically ability: ex recovery by consuming food, resting or sleeping
% --> thats why sleep data is important!
% "ultimately everyone has a limit, beyond which point they are physically
% unable to continue without physically replenishing themselves. In this case,
% recovery occurs by consuming food energy, resting, and sleeping."


\section{Timing of breaks}
- what have other studies already shown
- what is their approach
- what can I learn in finding good time slots for breaks
- Are there different types of people with different timing breaks
- What are the factors influencing the timing? (input variables?)

\section{Duration of the break}
- what have other studies already shown
- what is their approach
- what can I learn in finding good time slots for breaks
- Are there different types of people with different timing breaks
- What are the factors influencing the duration? (input variables?)
- Is it correlated with the timing?

\section{Activity during the break}
- what have other studies already shown
- what is their approach
- Are there overall good activities? or is it personal
- Are there different types of people with different needs for activities?
- What are the factors influencing the activities? (input variables?)

\chapter{Personalized Break Planner}

%Anschliessend ist es wichtig, nicht gleich mit dem Tool/Implementierung zu beginnen, sondern deinen Approach zu bzw. die Grundkonzepte erklären. D.h. 3.1 und 3.2 könnten wahrscheinlich hier reinpassen; die genauen Konzepte können wir dann aber auch noch besprechen.
The goal of this work is to develop a personalized break planner that supports users in finding the opportune   time for an activity and to decide on an activity in order to develop good/effective break habits. 

\section{Definition of the 3 Key Aspects}
To achieve this goal, 3 key aspects need to be considered:

\subsubsection{Timing of breaks}
As shown through several studies \cite{Largo-Wight.2017} \cite{KimS.ParkY.&Niu.2017}, it is important to schedule regular breaks, but different intervals should be considered depending on the time of day \cite{KimS.ParkY.&Niu.2017} and the individual. Thus, our approach will initially start with a static break interval that will be adjusted and further personalized over time, based on the user's preferences, calendar and the time of day. Another factor influencing the break interval is the user's sleep schedule. As shown in several studies \cite{Rosekind.2010} \cite{Gingerich.2017} \cite{Choi.2018}, a good night's sleep  can help to stay focused longer, which means taking fewer breaks. 
It is also important to ensure that the system does not suggest breaks to the user during periods of high concentration and focus when suggesting a break, to avoid interruptions and dissatisfaction with the system. Thus, various computer interaction trackers will be used to detect the state of the user and the personal calendar will be considered, to avoid scheduling breaks during meetings. 
To account for breaks that are less predictable, such as when a user is very stressed, the approach shall consider suggesting “emergency breaks” when sensing a higher stress state from an increased heart rate \cite{Hjortskov.2004}. 
Finally, spontaneous breaks that were not suggested by the system must also be detected to help adjust and personalize the timing of future breaks. Detecting such breaks could be accomplished through pedometer data, computer interaction trackers or self-reports. 

\subsubsection{Duration of the break}
Different techniques such as the Pomodoro technique and the 20-20-20 approach suggest different break durations. However, depending on the time of day and activity, the duration of the break may be different, e.g., a lunch break compared to a coffee break at 9 am. Based on users’ self-reports, the approach shall find more optimal durations for each individual user.  

\subsubsection{Activity during the break}
The activities pursued during the break significantly impact the recreational benefits of the break, as shown in the study by Kim et al \cite{KimS.ParkY.&Niu.2017}. User preferences and self-reports on their experience with the system will be incorporated to personalise break activity suggestions. 



\section{Rule-based method}
These three aspects will be integrated into a software system that uses a rule-based method to decide when and how long breaks should be taken, and to suggest activities that can contribute to a higher restorage on well-being. The system will suggest timely breaks that can be adjusted throughout the day depending on the data related to the three key aspects mentioned above. The system's predictions and suggestions will be validated by a brief self-report from the user before and after each break, as well as a detailed report at the end of the day.
The rule-based system starts with a static rule, such as the 50-10 rule, which is adjusted depending on the time of day to make an initial suggestion for the break schedule. For example, there should be no breaks at the end of the day, since many users find them inconvenient \cite{KimS.ParkY.&Niu.2017}. The break schedule is adjusted to the personal calendar, which ensures that no breaks are scheduled during meetings. As mentioned earlier, sleep schedule can also influence break intervals. In the short self-reports, the user will quantify the quality of the break. If the break is perceived as useless, for example, if it was too long, or too short after the last break. The rule-based system will then react accordingly and shorten the break duration or extend the break intervals. However, not only user input, but also data related to the three key concepts: Timing, Duration and Activity, have a significant impact on the rule-based system.
Compared to previous works, the system aims to dynamically adjust and personalize the break schedule, based on user-feedback and sensing, and suggests recreational activities for spending the break. 
To evaluate the system described above, it will be used in situ by 3 users during one work week. During this preliminary evaluation, the goal will be to learn about the user’s experience of using the system, as well as learn about their self-reported impact on their overall well-being, productivity and feasibility to use such a system in real-world work.

\subsubsection{Initial Break plan}
- How do i generate the initial break plan
- What different people do we have
- How can the user adjust this plan
- What is the theory behind it
\subsubsection{Adapting to user's needs}
- How is the system working?
- what are trigger values?
- what is the weight of the trigger values?
- Why did i choose this concept? (Show theory)

\chapter{Implementation}
%Nun sollte ein (separates) Implementation-Kapitel folgen (ich glaube das hast du in 4.0.1 vorgesehen). Hier kannst du beschreiben, wie du die Key-Konzepte effektiv implementiert hast.
\section{Frontend}
\section{Backend}
\chapter{Results}
%Dann folgt ein Preliminary Evaluation Chapter: hier beschreibst du zunächst die Methode und anschliessend Resultate
\subsection{Break Planner}
\subsection{User Study}
\subsection{Evaluation}
\chapter{Conclusion}
%
- short summary
- conclusion
    - show limitations
- futur work



% 
% \subsubsection{Subsubsection}
% \fig[.5\textwidth]{logos/logo_hasel}{Our logo}{logo}

% \subsection{Subsection}
% %
% \paragraph{Paragraph.} Always with a point.

% \begin{lstlisting}[caption=An example code snippet]
% /**
%  * Javadoc comment
%  */
% public class Foo {
% 	// line comment
% 	public void bar(int number) {
% 		if (number < 0) {
% 			return; /* block comment */
% 		}
% 	}
% }
% \end{lstlisting}

\appendix
\chapter{First Appendix}
\chapter{Second Appendix}

\backmatter
\bibliographystyle{alpha}
\bibliography{hasel_thesis/references}



\end{document}
