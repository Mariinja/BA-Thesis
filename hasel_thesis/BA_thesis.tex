\documentclass{hasel_thesis}

\thesisType{Bachelor}
\date{\today}
\title{Motivating effective breaks for knowledge workers with Break Scheduler}
\subtitle{Bachelor Thesis}
\author{Marinja Principe}
\home{Niederhasli} % Geburtsort
\country{Switzerland}
\legi{18-740-910}
\prof{Prof. Dr Thomas Fritz}
\assistent{Dr André N. Meyer}
\email{marinja.principe@uzh.ch}

\begindate{19.09.2022}
\enddate{19.03.2023}
\graphicspath{{images/}}

\begin{document}

\maketitle

\frontmatter

\begin{acknowledgements}
I would like to express my deepest gratitude to Prof. Dr Thomas Fritz for having the opportunity to write this bachelor's thesis at the Human Aspects of Software Engineering Lab (HASEL) and for his support. A special thanks go to Dr André N. Meyer, who provided me with invaluable guidance and support throughout the entire process. The outcome of this thesis would not have been possible without his ideas and hints, which pushed me in the right direction. Additionally, I would like to thank Roy A. Rutishauser for his expertise and support in the development process of the Break Scheduler. Other big thanks go to all the participants investing time and effort into this study and provided interesting insights. On a more personal note, I would like to thank my friends and family for their support and motivation during the process.
\end{acknowledgements}

\begin{abstract}
Knowledge workers generally have only a limited amount of personal resources, including energy, attention and physical capacity, to achieve tasks throughout their day \cite{BaumeisterR.F.BratslavskyE.MuravenM.&TiceD.M..1998}. Are these resources depleted, the person can feel stressed and emotionally exhausted \cite{Sonnentag.2001, Trougakos.2009}. Knowing how and being able to recharge personal resources is, therefore, essential. As knowledge workers often spend a large part of their day at work, it can be helpful to use time spent at work to establish tiny positive habits, which help to recharge personal resources. Several studies \cite{Largo-Wight.2017, KimS.ParkY.&Niu.2017} demonstrated how that regular breaks can significantly reduce stress and physical discomfort. However, while many studies focus on identifying opportune moments to suggest breaks, they rarely consider the activities that knowledge workers pursue during work. By the definition of resource depletion \cite{BaumeisterR.F.BratslavskyE.MuravenM.&TiceD.M..1998}, each activity can recharge or deplete resources, depending on personal preferences, making break activities crucial for achieving beneficial breaks.

This thesis explores how the Break Scheduler approach may increase users' awareness on their personal resources and break habits and how it supports them in identifying beneficial break activities to improve their personal resources. This approach uses self-reporting and nudging to improve awareness. Additionally, a rule-based system suggests a break schedule which is personalized by the user and will be adjusted by the Break Scheduler over the use period based on the user's self-reports. The investigation included 13 participants who used the software over one to two weeks. A total of 154 breaks are reported, as well as 143 daily reports. Each participant also answered a pre- and post-intervention questionnaire, giving valuable insights into their demographics, previous break taking habits, and experience with the approach. Overall, the findings suggest that self-reporting and nudging, such as scheduling the breaks in advance and notifications, can improve the awareness of the participants personal resources and break habits. Additionally, the personalisation aspect of the Break Scheduler is crucial to help users to identify break activities that were successfully supporting them recharge their personal resources. The results of this thesis offer insights into the potential of the Break Scheduler approach in supporting knowledge workers to increase their awareness on their personal resources and break habits by self-reporting and nudging and in helping them find beneficial activities to improve their personal resources.
\end{abstract}

\begin{Zusammenfassung}
Wissensarbeiter verfügen in der Regel nur über eine begrenzte Menge an persönlichen Ressourcen, einschließlich Energie, Aufmerksamkeit und physischer Kapazität, um die Aufgaben des Tages zu bewältigen \cite{BaumeisterR.F.BratslavskyE.MuravenM.&TiceD.M..1998}. Sind diese Ressourcen erschöpft, kann sich die Person gestresst und emotional erschöpft fühlen \cite{Sonnentag.2001, Trougakos.2009}. Zu wissen, wie man persönliche Ressourcen wieder aufladen kann, ist daher von entscheidender Bedeutung. Da Wissensarbeiter oft einen großen Teil ihres Tages am Arbeitsplatz verbringen, kann es hilfreich sein, die am Arbeitsplatz verbrachte Zeit zu nutzen, um kleine positive Gewohnheiten zu etablieren, die dazu beitragen, persönliche Ressourcen wieder aufzuladen. Mehrere Studien \cite{Largo-Wight.2017, KimS.ParkY.&Niu.2017} haben gezeigt, dass regelmäßige Pausen Stress und körperliches Unwohlsein deutlich reduzieren können. Während sich viele Studien jedoch darauf konzentrieren, günstige Momente für Pausen zu identifizieren, berücksichtigen sie selten die Aktivitäten, denen Wissensarbeiter während der Arbeit nachgehen. Nach der Definition von Ressourcenerschöpfung \cite{BaumeisterR.F.BratslavskyE.MuravenM.&TiceD.M..1998} kann jede Aktivität je nach persönlichen Präferenzen Ressourcen wiederherstellen oder aufbrauchen, was Pausenaktivitäten zu einem entscheidenden Faktor für das Erreichen vorteilhafter Pausen macht.

In dieser Arbeit wird untersucht, wie der Ansatz des Pausenplaners das Bewusstsein der Nutzer für ihre persönlichen Ressourcen und Pausengewohnheiten schärfen kann und wie er sie bei der Identifizierung vorteilhafter Pausenaktivitäten zur Verbesserung ihrer persönlichen Ressourcen unterstützt. Dieser Ansatz nutzt Selbstauskünfte und Anstöße, um das Bewusstsein zu verbessern. Darüber hinaus schlägt eine regelbasierte Methode einen Pausenplan vor, der vom Benutzer personalisiert und vom Pausenplaner im Laufe der Nutzungszeit auf der Grundlage der Selbstberichte des Benutzers angepasst wird. Die Untersuchung umfasste 13 Teilnehmer, die die Software über einen Zeitraum von ein bis zwei Wochen nutzten. Es wurden insgesamt 154 Pausen und 143 Tagesberichte erfasst. Jeder Teilnehmer beantwortete außerdem einen Fragebogen vor und nach der Intervention, der wertvolle Einblicke in seine demografischen Daten, seine bisherigen Pausengewohnheiten und seine Erfahrungen mit dem Ansatz bot. Insgesamt deuten die Ergebnisse darauf hin, dass die Selbstberichterstattung und die Anreize, wie z. B. die Planung der Pausen im Voraus und Benachrichtigungen, das Bewusstsein für die persönlichen Ressourcen und Pausengewohnheiten der Teilnehmer verbessern können. Darüber hinaus ist der Personalisierungsaspekt des Pausenplaners von entscheidender Bedeutung, um den Nutzern zu helfen, Pausenaktivitäten zu identifizieren, die sie erfolgreich dabei unterstützen, ihre persönlichen Ressourcen wieder aufzuladen. Die Ergebnisse dieser Arbeit bieten Einblicke in das Potenzial des Pausenplaner-Ansatzes, Wissensarbeiter dabei zu unterstützen, ihr Bewusstsein für ihre persönlichen Ressourcen und Pausengewohnheiten durch Selbstauskunft und Nudging zu schärfen und ihnen dabei zu helfen, nützliche Aktivitäten zu finden, um ihre persönlichen Ressourcen zu verbessern.
\end{Zusammenfassung}
    

\tableofcontents
\listoffigures
\listoftables
%\lstlistoflistings

\mainmatter
\chapter{Introduction}
%Introduction: genau, hier kommt die Motivation und Zusammenfassung vergleichbarer Forschung, um den Gap zu zeigen, anschliessend eine Zusammenfassung deiner Arbeit, Evaluation und Erkenntnisse, sowie Zusammenfassung der Contributions 

%- Importance of breaks
% %- why do we need breaks
% - How does it influence knowledge workers
% - Examples of papers which have shown breaks are important
%Why do we need breaks?
%What is a break? (vacatation, ..
 \begin{comment} 
- Background and context of the research problem
- Research problem and object description ives
- Research Methodology


\textbf{RQ 1:} How can self-reflection and nudging increase knowledge workers' awareness about their existing break habits and their personal resources, such as energy, attention levels or physical well-being?



Awareness is an important aspect of change. First, e person needs to be aware of their action before they can evaluate whether their actions need change or not. Therefore the Break Scheduler should be able to raise awareness of its user's personal resources, such as energy, attention and physical well-being. Input data for raising awareness of personal resources are limited to self-reports. The rule-based system and nudging should help the user to evaluate their behaviour and suggest alternatives. The goal is also to find specific features of the solution which can increase knowledge workers' awareness of break habits.

\begin{tcolorbox}[colback=white!5!white,colframe=black!75!black]
[Impact] RQ2: Can a personalizable Break Scheduler support knowledge workers' identification of good break activities and what is the impact on their break habits and ability to recharge personal resources during breaks?
\end{tcolorbox}


When awareness is raised the user can act on it and generate an impact. For each break, the Break Scheduler suggests a timing, duration and activity which was evaluated by the rule-based system. The impact of the rule-based system and the different analysis features will be evaluated in the preliminary evaluation. To evaluate the impact, a pre-questionnaire will be compared with the user data of the tool, as well as the post-questionnaire. As the Break Scheduler should enable users to find beneficial breaks, it is important to evaluate which features help to personalize the Break scheduler to the specific needs of the users.

\begin{tcolorbox}[colback=white!5!white,colframe=black!75!black]
[Tool] RQ3: Which aspects of the personalizable break scheduler prototype are most helpful, and are there design improvement opportunities?
\end{tcolorbox}

In the end, the implementation of the Break Scheduler will be discussed. Potential advantages and disadvantages will be evaluated, in order to create a personalizable tool which helps the user to identify beneficial break habits and raise the awareness of their resources.


\end{comment}

In today's world, with automated machines taking over routine and manual labour, the work of knowledge workers has become a critical factor in the economy \cite{Oskarsdottir.2022}. An essential factor for the work of knowledge workers is their well-being and personal resources, as stated by Oskarsdottir et al. \cite{Oskarsdottir.2022}. Most of them face a highly fragmented workload and many commitments, which can cause stress and dissatisfaction. This can lead to burnout or organisational problems \cite{Elkin.1990}. Therefore, it is important to learn more about the recovery process of knowledge workers' personal resources. Various studies \cite{Largo-Wight.2017, KimS.ParkY.&Niu.2017} have shown that regular breaks can significantly reduce work-related stress. In addition, regular breaks can reduce physical discomfort \cite{Waongenngarm.2018}, but also help people stay focused longer \cite{Ariga.2011, Bloom.2014}, and thus keeping productivity high throughout the workday. Trougakos and Hideg \cite{Trougakos.2009} propose to differentiate breaks in a systematic manner into vacation breaks, breaks between work days (such as weekends), and breaks within work days. While the effects of vacation breaks are usually short-lived, short, regular breaks within the workday have greater value for recovering a worker's personal and physiological resources.
Some studies investigated how to find the right time for a break to increase productivity and well-being. For example, Kaur et al. \cite{Kaur.2020} explored the finding of opportune moments for knowledge workers to take a break, using workstation activity and task data. Their approach detects weather a knowledge worker is happy or productive by using contextual emotion and activity signals in addition to daily tasks and report. Their study demonstrated an accuracy of 86\% using their approach for predicting opportune moments. In addition, several commercial approaches and tools that support reminders for taking breaks exist and are already widely used \cite{Alghamdi.2020}, for example, the Pomodoro technique \cite{Cirillo.2006} or the 20-20-20 rule \cite{Min.2019}. Packer et al. \cite{Packer.2021} showed that micro-breaks have a restorative effect on well-being and focus. However, finding the timing of the micro-break is essential, as shown by Kaur et al. \cite{Kaur.2020}. Nevertheless, different studies \cite{KimS.ParkY.&Niu.2017, Berman.2007} have demonstrated that no "perfect" break schedule suits everyone. While some people prefer and benefit more from having regular short breaks at the coffee machine, others prefer walking during their lunch break. Therefore, finding opportune moments for a good break and knowing how best to spend them remains challenging. 

Since each person has only a limited amount of personal resources, it is necessary to understand how to spend them best and, more importantly, recover them. The theory of resource depletion \cite{BaumeisterR.F.BratslavskyE.MuravenM.&TiceD.M..1998} states that each person has only a limited amount of "personal resources" that can be consumed but also recovered, depending on the activity. Activities that consume resources are so-called chores \cite{Trougakos.2009}. Typical examples of chores are working on other work-related tasks or running errands. On the other hand, activities that recharge resources are called respite breaks \cite{Trougakos.2009}. Whether a certain activity recharges or consumes resources varies from person to person. This indicates that break activities are critical for a good break and that each activity can consume or recharge resources depending on the different factors and preferences \cite{BaumeisterR.F.BratslavskyE.MuravenM.&TiceD.M..1998}. Despite the importance of workday breaks, few resources could be found investigating prescriptive suggestions for break activities. Epstein et al. \cite{epstein.2016t} conducted an approach which recorded break activities in relation to the productivity of the user. They could show that the pre-break productivity influences the chosen break activity. On the other hand, one approach by Hunter and Wu \cite{hunter2016give} focuses on prescriptive suggestions for breaks and their activities to improve the workday. They could indicate a relation between break activities and resource recovery depending on the time of the day. However, no study could be found focusing on increasing the users' awareness of personal resources and break habits while supporting them by suggesting break activities to achieve beneficial breaks.

This thesis aims to develop a personalised approach, named the Break Scheduler, which supports users in creating awareness of their personal resources and break habits and provides them with suggestions for beneficial break activities to improve their personal resources. In contrast to existing work, this thesis focuses on the user's awareness of personal resources and break habits as a tool for them to find beneficial break habits and, therefore, beneficial break activities. The Break Scheduler approach was realised through a desktop application that uses different features to achieve these goals. Multiple self-reports, such as morning and evening reports and before and after break reports, helped the user reflect on their current state of personal resources and guide a rule-based system to suggest the break schedule for the next day. Proactively planning breaks helped to create awareness of break habits and incorporate or improve tiny habits during working hours. Notifications ensured that breaks will not be forgotten during the day. To study the effect of the approach, the following research questions were explored and answered through the preliminary evaluation:

\textbf{RQ1:} How can self-reflection and nudging increase knowledge workers' awareness about their existing break habits and their personal resources, such as energy, attention levels or physical well-being?

\textbf{RQ2:} Can a personalisable Break Scheduler support knowledge workers' identification of good break activities, and what is the impact on their break habits and ability to recharge personal resources during breaks?

\textbf{RQ3:} Which aspects of the personalisable Break Scheduler prototype are most helpful, and are there design improvement opportunities?

To answer these research questions, a preliminary evaluation with 13 participants was conducted. Participants were asked to answer a pre- and post-intervention questionnaire and participate in an intervention period of around one to two weeks.
A total of 154 breaks were reported during the intervention period, and around 62 evening and 81 morning reports. Therefore, all data consist of the pre- and post-intervention questionnaire data and data gathered by the Break Scheduler during the intervention. Descriptive analysis is used for the quantitative data, and thematic analysis for the qualitative data.
This thesis suggests the potential of the Break Scheduler approach to increase mindfulness of personal resources and break habits by self-reporting personal resources multiple times a day and planning the breaks ahead. Proactive planning in cooperation with break reminders helps the users to raise awareness of their break habits and remind them to take breaks. Furthermore, the findings indicate that the Break Scheduler approach can help users find beneficial break activities and that taking more beneficial breaks improved the personal resources levels of the participants. However, break habits are very personal and differ from person to person. A tool supporting the user to create beneficial break habits should enable personalisation.

The following structure is used in this thesis. In chapter \ref{background}, a common background will be established, reviewing the theory of personal resources, job demand and job control and looking at the definition of a break. In chapter \ref{related_work}, already existing related work in the area of break and resource recovery for knowledge workers will be discussed. Chapter \ref{design_approach} explains the primary design approach of the Break Scheduler, followed by a detailed description of the implementation in chapter \ref{implementation}. In chapter \ref{primary_evaluation}, the preliminary evaluation setup is examined, and in chapter \ref{results}, the findings will be presented. Chapter \ref{discussion} evaluates the findings and discusses future works. In this chapter, the three research questions will also be answered. Chapter \ref{conclusion} concludes the thesis.


 \chapter{Background} \label{background}
 \begin{comment} 
 - Definition of a Break
 - Personal Resources and Physiological Capacities
 - Self-Reporting /self-Experimenting

 
\end{comment}

This section will establish a common understanding of the theory used for the presented solution. First, the background of personal resources and physiological capacities will be examined, followed by an introduction to work recovery. Finally, the definition of a break used for this thesis and an overview of different break types are shown.

\section{Personal Resources and Physiological Capacities}

\begin{comment} 
 - what are personal resources, what are physiological capacities (done)
 - why is it important (done)

 
\end{comment}
Personal and physiological resources are a key concept of the presented approach , which is why they will be discussed in detail in this section. The counterpart of resources is job demand which is also influenced by job control. Both of these values go hand in hand, therefore both are important concepts to look into in more detail. Why resources and the balance of resources are essential will be shown in the last section of this chapter.

\subsection{Definition of Personal Resources and Physiological Capacities}
Baumeister et al. \cite{BaumeisterR.F.BratslavskyE.MuravenM.&TiceD.M..1998} define \textbf{personal resources} as a limited set of resources that enable a person to complete various activities during the day. The amount of resources varies from person to person. This concept has been applied in multiple areas of psychology and organizational behaviour. While some studies focus on specific factors such as energy or stress, personal resources encompass various aspects. These include (but are not limited to) factors such as attention, energy, well-being, creativity, decision-making, and social contacts. Each person has a limited amount of personal resources available daily, so it is essential to know how to recharge them. 

\textbf{Physiological Capacity}, on the other hand, is a limited resource beyond which a person is physically unable to continue without rest. An example would be sleep or food. Each person has a limit to how long they can physically go without sleep or food. To recharge physiological capacity, each person must sleep, eat, or drink. Exhaustion of physiological resources usually leads to serious health consequences \cite{Trougakos.2009}. At this point, it is essential to mention that the two resources (personal and physiological) are not entirely independent.
Some studies \cite{Meijman.1998,Hobfoll.1989, Hobfoll.1998} do not separate personal resources from physiological capacity. These theories use the definition of limited resources, which is crucial for the next chapter on recovery.

Meijman and Mulder \cite{Meijman.1998} focus on the effort-recovery model, which deals with workers' efforts and their work management. The central concept is that a worker's work effort leads to work overload and decreased personal resources. According to them, recovery can only begin by taking time away from the overall work demand. This shows the importance of the recovery process and awareness of personal resources.

Hobfoll's \cite{Hobfoll.1989, Hobfoll.1998} conservation of resource theory compares stress and well-being and investigates the relationship between them. They were able to show that the loss of resources is more noticeable than the recovery / gain of resources. This enhances the point that it is suggested to prevent a resource deficit proactively.

Both studies include the concept of work demands that require effort or resources to be completed. The following section discusses the concepts of job demand and job control.



\subsection{Job demand and Job Control} \label{job_control}
The term \textbf{Job demand} refers to the features (physical, psychological, social etc.) of a job that require effort and, therefore, have a specific resource cost to complete this demand \cite{Trougakos.2009}. Examples of job demands are a heavy workload or time pressure. Previous studies showed that a high work demand affects the well-being \cite{Sonnentag.2003} of the person as well as their needed recovery time \cite{Sonnentag.2006}. A high job demand often leads to the feeling of time pressure. On the other hand, this leads to a depletion of resources as the increased demand requires more effort to complete all the tasks. As the time pressure is already high, the time for breaks and recovery will be relatively short or cut altogether, leading to less or no time for recovery while using more resources than available. This is also why many people with a high job demand engage in more chore activities such as work or running errands.
Compared with people with low job demands, short breaks for people with high demand may not be sufficient to recover from the resource deficit created over time. In such cases, only a more significant period, for example, some days off, can recharge their resources effectively. However, when the resources are refilled, it is essential to pro-actively and continually engage in activities which recharge the personal resources to avoid a deficit again \cite{Trougakos.2009}.

The job demands-resources model \cite{BakkerA.B.DemeroutiE.DeBoerE.andSchaufeliW.B..2003b, DemeroutiE.BakkerA.B.NachreinerF.andSchaufeliW.B..2001a} also includes a similar notion of demand. According to their definition, job demands refer to aspects of work concerning physical, psychological, social or organizational factors. Each task uses a particular effort and skills, which leads to a sum of demand costs, e.g. physical or psychological costs. When the employee can not recover from this job demand, the demand can turn into job stressors \cite{Hobfoll.1998}. Such stressors can lead to physiological (e.g. cortisol level increase) or/and psychological symptoms (e.g. fatigue)\cite{Sonnentag.2022}. Resources are essential to balance each person's needs and work demands. Not only are resources necessary as a balance for job demand but also for the development of each person. Job resources have the potential for motivation, leading to high engagement and performance. Therefore, job resources also have an intrinsic and motivational role \cite{Bakker.2007}.


\textbf{Job control}, on the other hand, refers to the freedom in how to organize the job demand, what to work on and when \cite{Trougakos.2009} and enable them to take decisions. Many studies were able to show that high job control benefits the employee's well-being \cite{Daniels.1994, Jackson.1983}. High job control has an advantage in two ways. First, employees with high job control are flexible to take breaks or switch tasks when tired, which helps them recharge their resources flexibly to their needs. Secondly, each employee can structure their day to fit their needs, for example, by joining a yoga course during the lunch break. 
On the other hand, studies have shown that employees with high job control and no formally scheduled breaks often do not take any breaks and overwork themselves \cite{McLean.2001}. They fail to take a break when needed, so they only take breaks when their resources are already meager. If they then engage in breaks activities, it is often not enough anymore for them to take short breaks but they need an extended break. Especially employees with less repetitive work (e.g. knowledge workers) can benefit from high job control \cite{Trougakos.2009}. Job control also affects the perception of certain activities. Let us consider the example of working during lunchtime. A person with a high job control decides to work during lunchtime, knowing that they have the control to stop whenever needed. Compared to a person with low job control, whose employer asks to finish a task before leaving, the perception of the person with high job control will likely not be as draining as the other.

But what happens when the equilibrium between an individual's personal resources and the demands and control within their job becomes imbalanced? This question will be discussed in the next chapter.

\subsection{Consequences of Personal Resource Levels} \label{consequences_resources_levels}
When personal resources are deplete, it can lead to stress, emotional exhaustion and decreased performance.  
Other research has found a relationship between stress and depletion of resources \cite{Sonnentag.2001}. Additionally, Trougakos and Hideg \cite{Trougakos.2009} suspect that low resources lead to high emotional exhaustion. They propose that when personal resources are low, people are more prone to stress and less able to deal with it. This theory will also be covered by one of the most influential stress theories by Lazarus and Folkman \cite{Lazarus.1984}, which state that personal and social resources play an important role when dealing with stress. According to Lazarus and Folkman, specific tasks will be perceived as a stressor when already coping with stress, whereas when resources are high, they will be more likely to be perceived as a challenge.
Furthermore, Lazarus and Folkman \cite{Lazarus.1984} distinguish two types of coping mechanisms: Problem- or emotion-focused strategies. People with low resources prefer to distance themselves from a problem as they can not deal with it now. On the other hand, people with high resources are more likely to tackle the issues and try to solve them, which will change the current situation, leading to a likely better position.
Last but not least, one \cite{Trougakos.2009} argues that the performance during the day depends on the assigned focus, which depletes the resources during the day. It is necessary to recharge personal resources to keep the performance during the day high. When this is not done, the performance can degrade\cite{Trougakos.2009}, as these resources are limited.

As stated above, the consequences of personal resources are significant for different reasons, including the employee's well-being and job performance. Therefore it is crucial to understand how to recover personal resources despite the daily work demand.
 

 \section{Work Recovery}
  \begin{comment} 
 - why are we looking at recovery for work --> many knowledge workers are stressed and recovery is key
  - How can we recharge this resources
 - Different ways of recovery: 
    - organisational
    individual -->breaks
\end{comment}
As seen in the previous section, personal and physiological resources are limited and differ from person to person. A deficit of these resources can lead to depletion of work or/and decrease the mental or/and physical well-being. The recovery of these resources is, therefore, a fundamental concept for mental and physical well-being. Sonntag et al. define recovery as an “unwinding and restoration process during which a person’s strain level has increased as a reaction to a stressor, or any other demand returns to its prestressor level” \cite[p.366]{Sonnentag.2017}. Usually, such stressors are generated by work demands, including negative state characteristics. This negative state can be characterized by high arousal as anger or anxiety, but it can also be characterized by low arousal as exhaustion, depression or fatigue \cite{Sonnentag.2022}. As restoration is done by eliminating the demand for a certain period of time \cite{Trougakos.2009}, the recovery can only start when the job demand has ended, including the psychological detachment from work. 
However, if this demand can not be decreased, as the person is mentally still present, or demand remains due to other stressors; no recovery can take place, and the load reactions accumulate over time. This leads to a higher work demand and a high resource depletion since an abundance of resources also consumes resources \cite{Trougakos.2009}.


One potential counteract for depleted resources is \textbf{motivation} \cite{Bandura.1986}. Work motivation is defined as the will to achieve a certain goal \cite{Locke.2004}. Locke \cite{Locke2000} has shown that job performance depends not only on one's skills or abilities but also on motivation. As every job task leads to a certain goal, the perception of the goal is crucial. The Goal-Setting Theory by Latham and Locke \cite{Latham.1991} states that when a goal is challenging and more advanced, it will lead to a greater outcome. More challenging goals often lead to bigger rewards, such as pay or a higher reputation. Therefore, employees often feel that even if their resources are low, it will be beneficial for them to continue and achieve the goal, increasing their focus and effort. The motivation helps them strive towards their goal and temporarily compensate for low resources. On the other hand, if a goal is not perceived as "worthy", more energy is needed to complete the task, or it will not even be completed at all \cite{Trougakos.2009}. Muraven and Slessareva \cite{Muraven.2003} conducted a study which can show that a depleted person can continue working on tasks which are perceived as essential or beneficial for them or others. They can show that depleted people continue working on self-control tasks which they believe are helpful to others, whereas people who do not believe the tasks will help others do not continue. Additionally, people who thought they could personally benefit from the task performed better than people who did not. In summary, Muraven and Slessareva \cite{Muraven.2003} state that depleted people are motivated for a task when the benefits of the outcome are higher than the associated cost of their resource depletion. On the other hand, a clear direction and goals can have an impact on well-being and stress levels when resources are low. Previous research can show that role ambiguity is positively related to stress, and even emotional burnout \cite{Posig.2003}. Role ambiguity will be defined as the lack of clear goals and/or directions. A depleted person needs clear goals and direction to know where to invest their remaining energy to achieve the goal. Additionally, when resources are already low, formulating and defining new clear goals might be challenging. When speaking about motivation, it is essential to differentiate between intrinsic and extrinsic motivation. Intrinsic motivation refers to having a choice, for example, when a person autonomously and out of their own volition defines a goal. This motivation inherits interest and is enjoyable. On the other hand, extrinsic motivation is forced and controlled. They lead to external validation as rewards or avoiding punishments \cite{Ryan.2000}. Intrinsic motivation employees will perform better as the tasks are more enjoyable and, therefore, easier to complete.

Another important definition when considering work recovery is the definition of \textbf{chore and respite activities} \cite{Trougakos.2009}. According to their definition of resource depletion \cite{BaumeisterR.F.BratslavskyE.MuravenM.&TiceD.M..1998}, each activity can consume or recharge personal resources, depending on different factors and preferences. 

Activities that consume more resources than they can recover are called \textbf{chores} \cite{Trougakos.2009}. They continue draining resources and fail to recharge personal resources. Such activities can negatively affect work performance or well-being. Sonntag \cite{Sonnentag.2001} was able to show that such activities primarily represent continuing with work in different ways, for example, switching tasks, domestic tasks or running errands and generally require an increased regulation of behaviour. These activities are mostly not enjoyed by the person, resulting in resource depletion. It is essential to state that the perception of an activity as a chore is highly dependent on the subject's perception. An example would be cooking, which some people enjoy leading to the recovery of personal resources, while others do not, leading to a depletion of their resources.


On the other hand, activities that recharge these sources are called \textbf{respites} \cite{Trougakos.2009}. To recover, activities need to be low effort, preferred choice, or enjoyable \cite{Trougakos.2009}. As shown in the sections \ref{consequences_resources_levels}, high resources positively affect job performance and well-being. Therefore respite activities are essential for job performance and well-being too. Trougakos and Hidge \cite{Trougakos.2009} specify two characteristics for a respite activity: the amount of effort and the degree on how much it was their preferred choice. A low-effort activity helps the person to reduce the demand and uses fewer resources to do the activity, resulting in a higher resource outcome. High-effort activities mostly increase demand, which is disadvantageous for the demand reduction necessary for recovery. A high effort can be balanced by the degree of preferred choice. These activities can aid in the recovery of depleted resources as they are enjoyable or energize the user. They generate a positive feeling and well-being. Exercise can be an example of a high-effort but preferred choice activity, as the activity uses a lot of energy but leaves the person with a positive feeling and a higher mental and physical well-being. These two factors define if an activity is perceived as a respite or chore activity. However, as seen in the given example, both are crucial and differ from person to person. It is suggested that respites are selected for break activities, as they recharge personal resources, whereas chores are advised to be avoided. This shows that break activities are essential factors for a beneficial break.

As a common understanding of personal resources and how to recover resources is established, the definition of a break can be looked at in more detail.


\section{Definition of a Break}

A work break is defined by \cite{Trougakos.2009} as a period of time when no work-related tasks are expected or need to be completed. This includes various types of breaks such as vacations, weekends, breaks at the end of the day, and breaks during work, including micro-breaks. To understand the advantage and disadvantages of each of these types of breaks, they will be further elaborated.


Lounsbury and Hoopes \cite{Lounsbury.1986} define \textbf{vacation} as a period of time "when a person is not actively engaged in his or her work. It is a time when a person is free to pursue other interests, and therefore a time when the work situation may become less important compared to other areas of experience such as family and personal leisure" \cite[p. 393]{Lounsbury.1986}. The goal is to recharge energy to be more productive during work time. It has been shown by different studies \cite{Westman.1997, Westman.2001}, that vacations reduce stress and lower burnout levels. Additionally, employees are more likely to engage in respite activities, which are relaxing and enjoyable, leading to positive emotions and recharging of personal resources \cite{Fritz.2006}. Unfortunately, these effects wear off within days or weeks after returning to work again. This means that vacations have only a short-term positive impact on stress levels \cite{Fritz.2006}.


\textbf{Weekends and end-of-day} are mini-vacations between work days that allow workers to relax and engage in other non-work related activities. Unfortunately, there are not yet as many studies that specifically look at weekends and their benefits to workers, but the studies that do exist show that the lack of time off can lead to lower well-being or burnout \cite{Fritz.2005}. Weekends promote rest and well-being through restorative and stimulating activities, preferrable with low demand. Additionally, engaging in favourable and low-demand activities at the end of the day leads to greater recovery. Weekends and end-of-day can be used to recharge personal resources, but only when invested in respite activities. Resources are depleted if they are used for chore activities such as longer working hours or running errands.


The first study of \textbf{breaks during the working day} goes back to Mayo in 1933 \cite{Mayo.1933}, in which a very general form was studied. The best-known types of breaks during the day are coffee or lunch breaks. A particular type of within-day breaks are microbreaks, defined as really short breaks, e.g., less than 10 minutes \cite{BennettAndrewA.GabrielAllisonS.CalderwoodCharles.2020}. Several studies have already examined the benefits of such breaks and how they can contribute to the recovery of personal resources. For example, just 5 minutes of stretching can help combat muscle fatigue that typically occurs after 40 minutes. Such research mainly focuses on the interval of the breaks, their timing, and their duration. Only a few studies specifically focus on the activity during those breaks. However, it was found that break activities during the workday affect emotions and their recovery process, showing that enjoyable and restful activities provide greater recovery \cite{Trougakos2008}. 

When considering further research, one can find a relatively clear common message: To understand the recovery process, one needs to consider the activity conducted during a break, not depending on the break type. Generally, as already shown in the recovery research, activities with low demand tend to have greater recovery and activities which fail to reduce the demand tend to have negative consequences. Especially for within-day breaks, it is often the case that employees can not emotionally detach themselves from work and therefore struggle to recover during breaks. In the case of vacations or weekends, it is often easier for employees.



\chapter{Related Work} \label{related_work}
%-	Das «theoretical background» chapter nennen wir meist “Related Work”, aber solche Details können wir dann später anschauen, wenn auch klarer ist, ob es noch einen theoretischen Hintergrund braucht. Hier wird dann sicher auch ein Kapitel wichtig sein zum Thema wie Wissensarbeiter:innen arbeiten, und weshalb Pausen wichtig/gut sind, und weshalb es schwierig ist, diese zu Timen, Managen, und geeignete Aktivitäten zu finden. Dh. Du kannst dann in den Subkapiteln (detaillierter als in der Intro) aufzeigen, was es schon gibt, und wo dein Ansatz anknüpft bzw. was neu ist
\begin{comment} 
- Overview of previous research on knowledge worker motivation and breaks
- Relevant theories and concepts related to knowledge worker motivation and breaks
- Previous studies and research works on the impact of breaks on productivity and well-being
\end{comment}

To understand existing work in the area of this thesis, the next chapter will examine the importance of breaks for knowledge workers and the promotion of breaks in their work environment. Additionally, already existing studies on the characteristics of effective breaks will be discussed.

\section{The Importance of Breaks for Knowledge Workers}
\begin{comment} 
Preventive breaks can help to reduce phisical disconfort, but most of the time knowledge worker only take breaks when it is alreaddy to late. cite Frequent short breaks...

Also add something to work cultur
\end{comment}

Knowledge work has become increasingly important in today's world. As automation takes over routine and manual labour, the work of knowledge workers has become a critical factor in the economy \cite{Oskarsdottir.2022}. As noted by Oskarsdottir et al. \cite{Oskarsdottir.2022}, knowledge workers' well-being and personal resources are critical factors in their productivity. They divide the related factors into two groups: Internal factors, which include well-being and personal resources, and external factors, such as salary, job design, organization and culture.
In this thesis, we are only concerned with internal factors, specifically knowledge workers' well-being and personal resources.

\textbf{The Importance of Within-Day Breaks for Knowledge Workers.}
As work requires energy and effort, it drains the energy level of knowledge workers and can influence their physical and mental well-being. After a certain period, it is necessary to recharge the resources used for accomplishing the work tasks \cite{Sonnentag.2006}. For most knowledge workers, this recovery process only starts each day after working hours. As shown in the Background chapter \ref{background}, weekends and end-day recovery are essential for the well-being of each person. Still, within-day breaks can help knowledge workers proactively keep their resources high and prevent the depletion state, which needs even longer breaks to recharge. Several studies \cite{Largo-Wight.2017, KimS.ParkY.&Niu.2017} have shown that regular breaks can significantly reduce work-related stress. In addition, frequent breaks can reduce physical discomfort \cite{Waongenngarm.2018} and help stay focused \cite{Packer.2021}, thus keeping productivity high throughout the workday.

\textbf{Challenges in Taking Breaks for Knowledge workers.}
To incorporate within-day breaks, knowledge workers need to have a workspace which allows their employees to be free in when to take breaks to better fulfil their individual work patterns. One key factor for a beneficial break environment in the workspace is high job control. Especially in the area of knowledge workers, which are involved in several projects with different agendas and deadlines, it is essential to have flexible schedule breaks and restorative phases \cite{Trougakos.2009}. As discussed in the background section \ref{background}, knowledge workers mostly have high job control, leading to the flexibility to take breaks and the risk of overworking. It can be difficult to balance the work demand and the depletion of personal resources. Therefore, finding opportune moments and beneficial break activities remains often a challenge for many knowledge workers \cite{Trougakos.2009}.


\textbf{The Power of Tiny Positive Habits for Knowledge Workers.}
However, incorporating tiny positive habits is one of the most effective ways to promote physical and mental health into daily life \cite{Taylor.2005}, such as taking the stairs to get more steps in or working at a standing desk to improve posture. Since many knowledge workers spend a large portion of their day at work, it can be helpful to use the time they spend at work to establish tiny positive habits, such as regular breaks. As stated by Fogg, B. J \cite{fogg2019}, tiny habits are especially useful for knowledge workers, as they focus on "tiny" changes, which should only use a small amount of time and enables the person to incorporate new habits easily. If such a tiny change is established, they can grow into big changes after some time \cite{fogg2019, clear2018}.

\textbf{The Importance of Tools for Supporting Healthy Break Routines Among Knowledge Workers.} Having a tool supporting knowledge workers in maintaining their flexibility while engaging in healthy break routines can help. Some studies, such as that of Kaur et al. \cite{Kaur.2020}, use job activity and task data to examine which moments are opportune for knowledge workers to take a break. They can predict the right time with 86\% accuracy. However, other, more commercial tools support break reminders and are already widely used \cite{Alghamdi.2020}, for example, the Pomodoro technique \cite{Cirillo.2006} or the 20-20-20 rule \cite{Min.2019}. Some of the already existing solutions and studies will be discussed in more detail in the next section.




\section{Characteristics of Effective Breaks}
A broad variety of research exists in the area of work breaks. These are categorized as the timing of, duration of, and activity during breaks and discussed as follows.


\subsection{Timing of Breaks}
% - what have other studies already shown
% - what is their approach
% - what can I learn in finding good time slots for breaks
% - Are there different types of people with different timing breaks
% - What are the factors influencing the timing? (input variables?)
Most of the studies done look into the break timing and duration of a break. A solution to find opportune moments for breaks at work was done by Kaur et al. \cite{Kaur.2020}. Using effect, workstation data and task information to conduct whether it is an opportune moment for a break or a task switch. Their accuracy timing is up to 86\%. However, this study only looks at the time when a break should be started but not the duration nor the activity even though many studies have shown that the activities are key for an effective break. They also differ between taking a break and switching tasks. When a person is stuck on a problem but the attention level is still high enough, they propose a task switch instead of a break.

Another approach is a tool called BreakSense \cite{Cambo.2017}. BreakSense is a mobile application which detects through Bluetooth if a person has left their workplace. When it was sensed the person got up, small activities will be suggested which can be done in the office place. The goal is to incorporate small breaks to increase physical movement during work. Instead of suggesting timing, BreakSense detects when the knowledge worker is anyway not working and adds some small physical activities to it.

There are different tools like Flow \cite{flow} or TimeForBreak \cite{luo.2018}, which provide simple timers for focus duration or break duration to follow the Pomodoro technic \cite{Cirillo.2006}. The user can define the duration for which they want to focus on work or take a break, and when the time is finished, the user will be notified. Such a simple timer can have a positive effect on the user as the user does not need to remind taking breaks or need to check the time spent on a task. However, these tools only incorporate the timing of a break and the duration, but not the activity.


\subsection{Duration of Breaks}
\begin{comment} 
- what have other studies already shown
- what is their approach
- what can I learn in finding good time slots for breaks
- Are there different types of people with different timing breaks
- What are the factors influencing the duration? (input variables?)
- Is it correlated with the timing?
\end{comment}

Most of the already existing work does focus on timing. However, as part of their study, important information on the break duration was found. The study of Epstein et al. \cite{epstein.2016t} conducted that the average break duration was 22.9 minutes. They also state the relation between activity and the duration of a break. While breaks outside prolong a break, necessary breaks (such as going to the toilet) mostly shortens the break duration. Additionally, digital breaks most of the time took longer than intended, whereas breaks outside or necessary mostly took less than anticipated.

\subsection{Activity during Breaks} \label{rw_activity}
\begin{comment} 
- what have other studies already shown
- what is their approach
- Are there overall good activities? or is it personal
- Are there different types of people with different needs for activities?
- What are the factors influencing the activities? (input variables?
\end{comment}
As stated in the Background chapter \ref{background}, the activities pursued during the break influences whether the break helps to recharge or further deplete personal resources. Although there are already some studies on the timing and duration of a break, there is surprisingly little work on activities and the recovery process during a break. The study of Sonntag et al. \cite{Sonnentag.2022} looks at two approaches to studying the recovery process: the activity during a break and the psychological experiences.

\textbf{Individual Differences in Ideal Break Activities.}. Whether an activity recharges or consumes resources depends on the person.
As discussed in the Background chapter \ref{background}, two main factors influence a respite break: effort and enjoyablity\cite{Trougakos.2009}. Engaging in activities that the person does enjoy could lead to a positive experience. On the other hand, needing a lot of motivation to engage in an activity which is not enjoyable causes a decrease in resources. Therefore finding enjoyable activities has a high value, as there are barely any resources needed to do them, but they recharge a lot of resources. Which activities are considered enjoyable and which are not is very personal.
\textbf{Individual differences} influence the experience and appraisal of the activities. One individual factor is the extraversion of a person. The definition of extraversion by McCrae and John \cite{McCrae.1992} states that it is the degree to which a person is talkative, assertive and energetic. Highly extroverted individuals may recharge their personal resources while engaging in social activities, such as eating lunch with other people. At the same time, introverted individuals label such an activity as a chore. These activities drain more energy from them than they can recharge. Introverted people may benefit more from having lunch alone, enabling them to recharge their social energy and recharge their personal resources. Another interesting point is that previous research could find that people with a higher extraversion compared to people with a lower extraversion have a more positive appraisal tendency \cite{Gallagher.1990, Hemenover.1996}, leading to a higher coping ability \cite{Penley.2002}. Thus, individual differences play a significant role when looking into the activities.


\textbf{Psychological experiences.} Psychological experiences refer to a person's mental state when being on a break and how they experience the break. Sonntag and Fritz \cite{Sonnentag.2017} use the following terms to determine the recovery experience: psychological detachment from work, relaxation, mastery, and control. Most research attention goes into the psychological detachment from work, which focuses on freeing the mind from work thoughts and gaining distance to the work during free time. Relaxation focuses on activities that calm a person's body and mind. On the other hand, mastery refers to overcoming challenges and learning, whereas control refers to self-agency to decide what to do in non-work times. These four recovery experiences can be shown empirically and conceptional \cite{sonnentag.2007}. Therefore, they can influence beneficial breaks.


\textbf{Approaches for identifying Break Activities.} When looking at break activities, self-reporting has been proven effective in different studies \cite{Bloom.2014, epstein.2016t}. Epstein et al. \cite{epstein.2016t} conducted a study by building a desktop application which gives insights into the break habits of the user in relation to their productivity using self-reports. They can show that the pre-break productivity significantly influenced the selected activity. When productivity was rated low, participants tended to take digital breaks, such as checking emails, whereas when productivity was high, they tended to take necessary breaks, such as going to the bathroom. Their data shows that the length for necessary breaks is around 5min, whereas the break duration for going outside is around 10min. Therefore, the duration is dependent on the activity.



\chapter{Personalized Break Scheduler Approach} \label{design_approach}
%Anschliessend ist es wichtig, nicht gleich mit dem Tool/Implementierung zu beginnen, sondern deinen Approach zu bzw. die Grundkonzepte erklären. D.h. 3.1 und 3.2 könnten wahrscheinlich hier reinpassen; die genauen Konzepte können wir dann aber auch noch besprechen.



As discussed above, a lot of research focuses on how the timing and duration of workers' workdays can be optimized to support better breaks that allow workers to recharge their personal resources. However, the activities that workers pursue during a break also impact the effectiveness of their break, as they could even further deplete their personal resources. What an ideal activity is, depending on the worker's preferences and current work context is individual. Very few approaches exist that try to support workers in identifying ideal activities for their break. Therefore, this thesis focuses on the following key objectives:

\begin{tcolorbox}[colback=white!5!white,colframe=black!75!black]
• \textbf{Q1:} Raise users' awareness on their break habits and their effectiveness

• \textbf{Q2:} Allow users to reflect on their personal resources frequently to identify ideal break activities

• \textbf{Q3:} Empower users to develop a tiny habit of regularly taking breaks that recharge personal resources.
\end{tcolorbox}


Going forward, this approach  will be referred to as the Break Scheduler. The Break Scheduler is developed to enable users to reflect on their personal resources and break habits to find beneficial breaks during work hours. It is essential to state that the main concept is not about dictating the perfect break schedule for the user but rather enabling the user to analyse their break habits and personal resources. In the following sections, it will be discussed how the main objectives of this approach will be tackled.


\section{Users' Awareness on Their Break Habits and Their Effectiveness}
To improve awareness on beneficial break habits and enable the users to rate their effectiveness, the concept of personal resources is very valuable. By the definition of resource depletion, as stated in the Background chapter \ref{background}, Break habits are successful if they can recharge personal resources during the day. 

\textbf{Awareness of Personal Resources during the Day.}
Therefore, one of the first steps for the user is to create awareness of their personal resources in combination with their break habits. After that, the user can evaluate whether resources deplete too much over the day and which break habits can help to reduce resource depletion. As shown in different studies \cite{GRANT20083}, self-reflection is a commonly used strategy to enable the user to raise awareness and make proactive changes. Therefore, different self-reporting measurements are used in the Break Scheduler to improve awareness. First, users will be asked to rate their personal resources each morning and evening. In the morning, they are asked about the state of their personal resources by providing a percentage. Rating personal resources present itself as a challenge. Therefore, a percentage rating should help the user to define their own measurement scale. In the evening, each user rates their used personal resources, including the daily job demand in percentage. The primary value of this self-reflection is to see the relationship between personal resources over the day and how they change over the week. The reported morning and evening values need to be visualised for the user to reflect on their value over the week. This approach motivates the user to report their personal resources and helps them to reflect on their personal resources.

\textbf{Awareness of Break Habits.}
In the second step, the user can use the increased knowledge about their personal resources to reflect on their break habits during the day. Therefore the second measurement is to ask the user about their existing break habit knowledge. In order to encourage the participant to reflect on their break habits and show that taking regular breaks is not a luxury, but a necessity, each user will be asked to state their preferred break interval, which is the time between breaks when a user is focused and working, and also their break duration. The existence of breaks is a given but still enables users to define their preferred interval and duration. This personalisation of the interval and duration allows the Break Scheduler to adapt to its user's specific needs. Another key tool to raise awareness of breaks is the user's personal calendar. Break planning will be done by integrating them into the user's primary calendar so they and others can plan the daily schedule with the breaks in it. Again, the breaks are not scheduled only when there is some time left but, in advance, incorporated into the user's schedule. Scheduling breaks have only a limited advantage, as breaks are often forgotten, as discussed by \cite{McLean.2001}. A reminder prevents the user from forgetting about breaks, a problem that knowledge workers with high job control face, as discussed in the Job Control section \ref{job_control}.

\section{Identify Ideal Break Activities}
The second main objective of the Break Scheduler is to use the increased awareness from the reflection of personal resources to identify beneficial break activities. To support the user in finding beneficial break activities, the Break Scheduler will suggest a break schedule with beneficial break activities, which the user can use as inspiration. The Break Scheduler approach focuses on three key aspects when scheduling a break, which will be discussed in the next section.

\subsection{Definition of the 3 Key Aspects of a Break }
For the approach used in this thesis, a break is defined through three aspects: the timing of a break, the duration of a break and the activity during the break. These aspects are also used by other studies, as stated in chapter \ref{related_work}. However, to create a common understanding and show how these factors are influenced, all three aspects will be elaborated.

\textbf{Timing of the Break}
As shown through several studies \cite{Largo-Wight.2017, KimS.ParkY.&Niu.2017}, it is important to schedule regular breaks, but the preferred break interval can vary from user to user. Thus, the proposed approach starts with a fixed break interval, which the user can define and adjust. This interval will gradually be corrected and personalised based on user preferences over time. Additionally, it is essential that breaks are not scheduled during meetings and thus interfere with a user's work schedule. Therefore, the time of breaks must be adjusted to the user's personal calendar. This feature will be achieved by synchronising the personal calendar and the Break Scheduler. Additionally, all planned breaks will be saved in the personal calendar, which helps a user to remember their planned breaks when planning the rest of their work. A positive side effect is that other company members may have access to the user's calendar and could try not to schedule short-notice meetings into your planned breaks or motivate them to join some of the breaks.

\textbf{Duration of the Break}
Different techniques, such as the Pomodoro and 20-20-20 approaches, suggest different break durations. However, depending on the time of day and activity, the break duration may differ, e.g., a lunch break compared to a coffee break at 9 am. Based on users' self-reports, the approach aims at finding a more optimal duration for each user. Each evening the user will reflect on the break duration and state if it was good, too long or too short. The rule-based system will adjust the duration according to the user's input. As the lunch duration can differ from day to day, the user will also need to enter their preferred lunch duration for the next day. The lunch break can differ, as people can have lunch meetings or a high work demand leading to a shorter lunch break. The Break Scheduler considers the input of the evening report when scheduling breaks for the next day. On the other hand, the duration also depends on the activity. Depending on the duration, the group of possible activities should be adjusted.

\textbf{Activity during the Break}
Finding beneficial breaks, or respite breaks, as stated in the Background chapter \ref{background}, is a main objective of the Break Scheduler. This approach uses different factors of personal resources to state whether a break activity is a respite or a chore. The following section \ref{evaluation_break} will define how the Break Scheduler evaluates if a break activity is a respite or a chore. Users' preferences and self-reports on their experience with the activity should be incorporated to personalise the suggestions for break activity. 


\section{Evaluation of the Break} \label{evaluation_break}
As discussed in the Background chapter \ref{background}, whether a break is a chore or respite depends on the net value of personal resources before and after the break. The Break Scheduler will therefore evaluate whether a break is beneficial or not from the user's input before and after the break. The user reflects on the state before the break starts and on their resources directly after the break. To enforce the evaluation, each participant will receive a reminder when a break should start. Additionally, the report can only be done during the break; afterwards, it is not possible anymore. This design choice prevents the "I can also do it afterwards"-syndrome, which could falsify the input. As it is hard to state the current personal resources, it can be harder to quantify them after some time has passed. Therefore this report can only be done at the time of the break. As the concept of personal resources is hard to quantify, the Break Scheduler uses three central values to state the user's personal resources: Energy, attention and physical well-being. As these quantifiers need to be stated before and after each break, the number of asked values must be as small as possible, giving a broad insight into the personal resources. 

\textbf{Energy}: energy resources are also defined as personal resources at work by Klijn et al. \cite{klijn.2021}. They define energy as a multidimensional construct encompassing physical, emotional, cognitive, and psychological factors. According to Klijn et al., energy is considered a personal resource enabling its holder to deal with the work demand. Additionally, they also found that energy is positively related to well-being and job satisfaction and negatively related to burnout and stressors. Considering these factors, energy is one of the quantifiers for the Break Scheduler.

\textbf{Attention capacity}: attention capacity is defined by Fried and Aricak \cite{Fried2014} as the ability to assign attention resources to complete tasks. They consider attention a personal resource as its owner can use them to cope with work demands. On the other hand, the authors can show that attention capacity can play an essential role in the user's well-being and how to deal with stressors.


\textbf{Physical well-being}: As discussed in the Background chapter \ref{background}. The physiological capacity of each person is limited, and after a certain level, the person is forced to recharge (as the example of eating or sleeping). A deficit in these values mostly leads to serious health issues \cite{Trougakos.2009}. In some definitions, physical capacity will also be added to personal resources and is therefore important for mental well-being. Different studies show that sleep significantly influences a person's performance and stress level \cite{Rosekind.2010, Choi.2018}, proving that physiological resources are important factors to consider. Previous studies showed that physical and mental health are interlinked \cite{Doan.2022}, proving the importance of incorporating physical well-being as a crucial factor when determining whether a break is beneficial. 

These three quantifiers do not claim to complete the whole range of personal and physiological resources. These values are needed to cope with work demands and are part of personal resources. Even though they do not cover the whole range of personal resources, they are practical values in giving insights into the user's current state. 

\textbf{Enjoyability}: Another essential factor when looking at how beneficial a break is, is the \textbf{preferred choice}. As discussed in the Background chapter \ref{background}, the benefit of a break can be measured by two factors: effort and preferred choice. If an activity is enjoyable, it will generate more personal resources and can also balance out the fact that it is high effort, e.g. working out. The fourth value counting into the break evaluation is, therefore: "How enjoyable was the activity?" which will be asked after each break.

\textbf{Experience value:} all these four values will be included in the experience value of the break activity. The experience value states how beneficial the activity was for the user in the past. The higher the value, the better. If the value is positive, it is classified as a respite activity; if it is negative, it is a chore activity. The calculation procedure for the experience value will be shown in the Implementation of the Rule-Based System in section \ref{rule-based-method-implementation}.

\section{Regularly Taking Breaks as a Tiny Habit to Recharge Personal Resources}
The third main objective of this approach is to empower the user to create tiny habits of regularly taking breaks that recharge personal resources. However, users often do not know their preferences from the start. In order to support users in finding beneficial breaks, the Break Scheduler uses a rule-based System which generates a break Schedule to suggest breaks for the next day and enables personalisation. The rule-based System is the heart of the Break Scheduler, which considers personal preferences as the break interval, duration, starting and end time and pre-selected activities but also incorporates current context factors such as meetings or preferences for specific activities. In figure \ref{fig:approach_input_output} in the appendix, different possible input and output values of the Break Scheduler are shown, as for example, the extroversion of a person, their general concentration span or their job control. This figure also visualises different possible input and output values for daily specific factors, such as the current attention level, which could be measured by a window or input tracker, the current physical and mental well-being of the user, or the current workload. However, these are example factors which can influence the personal resources of users and show the expandability of the Break Scheduler. Depending on some of these factors, the Break Scheduler suggests a break characterized by three main components, timing, duration and activity. Again, it is important to mention that the rule-based System will generate a suggestion and does not attempt to dictate the break timing, duration or activity. It should support the user in finding good break habits while adjusting to their needs.


\subsection{Timing and Duration of Breaks}
For the definition of the timing and duration of the break, the Break Scheduler approach focuses on two types of data: static and non-static values. Static values do not drastically change daily and are less influenced by daily circumstances but rather stay the same. These values are fixed when the system starts scheduling breaks for the day. On the other hand, non-static data represent values which are daily specific and can, therefore, change each day. 

\textbf{Static data} are the defined break interval as well as the break duration. They will be defined at the beginning when the Break Scheduler starts for the first time and can 1) be manually adjusted when needed or 2) adjusted over the week as part of the personalisation.


\textbf{Non-static data} will be defined every evening before the schedule is generated. They include the start of the day, end of the day, lunch break duration and, of course, the calendar events for the next day. This data is specific to the respective day, as they can change frequently. Before creating the break schedule, users must confirm or adjust the start and end of the next day and their lunch duration. As the Break Scheduler is synchronised with the personal calendar, a user does not have to enter the calendar data manually. Additional non-static data are values such as the current state of the personal resources, nutritional intake or sleep data, which influence the break habits during the day.


\subsection{Activity during the Break}
In order to detect break activities that help recharge personal resources, each activity has an experience value. This value should sum up how the activity was rated in the past. Activities with a higher experience value should be chosen for future breaks. How the experience value is defined on this approach is shown in the Implementation of the Rule-Based System in section \ref{rule-based-method-implementation}. 

As it depends on the user, which activities are beneficial for them, the Break Scheduler approach provides a set of activities which can be adjusted by the user. A set of activities can be found in the appendix section \ref{table:activities_categories}. Certain activities can be grouped as they have similar characteristics. The Break Scheduler should also consider such beneficial categories of groups. Not only the category but also the daytime can influence whether a break activity is considered beneficial or not. That is why each of the activities has assigned categories. All defined categories are shown in figure \ref{fig:categories}. These are in accordance with the classification proposed by Kim et al. \cite{KimS.ParkY.&Niu.2017}.

\begin{figure}[htp]
    \centering
    \includegraphics[width=4cm]{hasel_thesis/images/categories.png}
    \caption{Break Scheduler Categories adapted from Kim et al. \cite{KimS.ParkY.&Niu.2017} }
    \label{fig:categories}
\end{figure}



These categories will be used to analyse the different aspects of activities, such as body movement or nutrition intake, and also include the daytime for suggesting beneficial activities. Therefore, each of these categories has four assigned values: an overall sum of experience value, a morning sum of experience value, a lunch sum of experience value and an afternoon sum of experience value. When selecting an activity, it is essential to ensure 1) the selection of the past beneficial breaks and 2) the incorporation of the daytime into specific groups of activities. This will be done by combining the categories with the daytime experience values and the activity experience values. At the start of using the Break Schedule, it is also essential to suggest activities that have not been tried to ensure each activity was tested before suggesting the most beneficial one. More details about the implementation will be discussed in the Implementation of the Rule-Based System in section \ref{rule-based-method-implementation}.

When selecting an activity, the experience value and duration are vital, as not each activity can be completed in different break durations. Therefore, the rule-based system should also consider the duration when selecting an activity. The choice of activity also depends on the daytime. At lunch, an activity from the "Nutrition intake" category must be selected, whereas, in the morning or afternoon, activities from the highest-ranked category will be selected. 

Suggesting only one activity can feel restrictive, which could decrease the user's motivation to take breaks. Therefore, a random activity will be chosen from the whole activity list in addition to the suggested activity. This mechanism adds some randomness to the activity selection process and helps to reduce bias which can appear when only the activity with the highest experience value is suggested. Each user will have a suggestion and a random activity to choose from before starting their break. 




\chapter{Implementation} \label{implementation}
\begin{comment}
4.2, 4.3 und 4.4. gehören in 5 – hier geht es ganz konkret darum, wie du die Key-Features/Themes aus dem Approach programmiet/umgesetzt hat. Hier stellst du deinen Break Scheduler vor
\end{comment}

In this chapter, the implementation of a prototype implementing the Break Scheduler approach stated in section \ref{design_approach} will be looked at. Nonetheless, as it is the first version of this prototype, it could be improved and extended, for example, based on users' feedback in the Preliminary Evaluation in chapter \ref{results}). The first section will discuss the system architecture to provide the reader with a comprehensive system overview before diving into the specifics. After that, the main focus is placed on the user interface implementation and the functionalities of the Break Scheduler.

\section{System Architecture}
\begin{comment} 
Describe the overall system architecture, including the components, modules and the relationships between them
Components: User interface, Rule-based system, Google API, Microsoft API and SQLITE DB
\end{comment}
The Break Scheduler desktop application is a standalone electron application mainly written in Typescript. It is built with Electron \cite{electron}, a popular framework for cross-platform desktop applications, to ensure the publication of the desktop app on the three main operating systems: macOS, Windows and Linux. The application has four main components: user interface, rule-based system, calendar integration, and database. In figure \ref{fig:architectur}, the connection of these four components is visualized.


\begin{figure}[htp]
    \centering
    \includegraphics[width=14cm]{hasel_thesis/images/architectur.png}
    \caption{Break Scheduler architecture}
    \label{fig:architectur}
\end{figure}


\textbf{User Interface:} A key aspect of the break scheduler application is the user interface and a self-explanatory design. The user interface's primary goal is visualising the generated data and enabling users to adjust and reflect on them. Different predefined dashboard templates were used when designing the interface to support an intuitive user interface. The User Interface section \ref{interface} will show details of the user interface and its specification. All data from and to the user interface will be transferred through the preload file, as suggested by the electron's inter-process communication (IPC).


\textbf{Rules-Based System:} The rule-based system is the heart of the Break Scheduler. It generates the Break Schedule, which will be added to the users' calendar. Considered input values for the break schedule consist of different self-reports. Additionally, the Break Scheduler rates the experience of the activities by calculating the experience values of each break, which will be added to the activity of this break. Its implementation will be discussed in section \ref{rule-based-method-implementation}.  and performs the 


\textbf{Calendar Integration:} To support the user to plan breaks around their work schedule, calendar integration is crucial. Users have two possibilities, they can either connect the Break Scheduler to their Google calendar or their Microsoft calendar. For this, the Break Scheduler can be connected to the Google API \cite{googleAPI} and the Microsoft Graph API \cite{graphAPI}. Each minute the Break Scheduler synchronises the calendar with the user calendar and checks for changes. Additionally, when the application generates the schedule, the application will synchronize all suggested breaks into the personal account, enabling proactive planning of the breaks. More implementation details are discussed in section \ref{calendar_integration}.


\textbf{Database:} The database used for the Break Scheduler is a simple SQLite database, which is well-suited for storing small amounts of data within a desktop application. This file is stored locally on the user's machine. The advantage here is that the user owns all of their stored data, and no one else can access it without their consent. Within this database, the initial settings, the activity and category values, as well as all self-reports, are saved. This information is needed for the rule-based system to work. For this thesis, additional information is stored in the database, such as logs and scheduled breaks.

In the next section, the user interface design and the key functionalities, including the rule-based system, the calendar integration, and the database, will be discussed.

\section{User Interface Design} \label{interface}

The user interface is an essential factor of the Break Scheduler, as the application should enable users to reflect on their data with the help of different visualisations. When designing the interface, different predefined dashboard templates starAdmin\footnote{\texttt{https://themewagon.com/themes/star-admin-2-free-bootstrap-4-html5-admin-dashboard-template/}}, codyhouse\footnote{\texttt{https://codyhouse.co/gem/schedule-template/}}, radiance\footnote{\texttt{https://www.free-css.com/free-css-templates/page266/radiance}}, and nicepage\footnote{\texttt{https://nicepage.com/website-templates}} for the generation of the self-reports were used to ensure the user experience and human-centred interaction (HCI). However, adjustments and the definition of the user flow needed to be made to fit the user's needs.

\begin{figure}[htp]
    \centering
    \includegraphics[angle=90 , width=8cm]{hasel_thesis/images/userflow_v3.png}
    \caption{Break Scheduler user flow}
    \label{fig:user-flow}
\end{figure}


\textbf{General Application Workflow.} As shown in figure \ref{fig:user-flow}, the user will initially start with the application setup. At this stage, the user will set up the Break Schedule for the first time, so they need to connect their personal calendar and define their break interval and duration. Each user will be asked to select eight activities they want to experiment with. This was defined for the purpose of this thesis as the usage duration is limited to one to two weeks, and all selected activities need to be tried once to activate the primary value of the rule-based system. The selected activities are the activities which the rule-based system will consider. These are the first personalising steps. Adjusting the break interval and duration enables the user to adjust to their timing preferences while selecting the activities enables them to adjust to their activity preferences. These values can be adjusted afterwards but not as part of the primary user flow. 


\textbf{Daily Break Workflow.} When the first break schedule is complete, the primary user flow will start, with an iteration cycle of one day. Each morning the user will be notified to answer the morning report, which is a short energy report. During the day, the user will also be notified, as shown in section \ref{fig:notification}, to take breaks and fill out the before and after break reports, which the rule-based system will use to calculate the experience values. The reminder via notification is an essential factor of the Break Scheduler, as many knowledge workers with high job control struggle to remember to take breaks \cite{McLean.2001}. During the work day, a short notice meeting or a deadline might be approaching, making it essential to reschedule breaks. When working, the user might be on their calendar many times a day. The design choice to enable changes to the break schedule only in the calendar app allows users to work with their known medium and reduce the application used simultaneously. The goal is to help users find their best fit and allow manual adjustments to stay flexible throughout the day by working with a single tool, their personal calendar. During the day, the interaction with the Break Scheduler is limited to the notification and the before and after self-reports, which pop up when clicking on the notification. The goal is to minimise the attention needed to interact with the application and minimise the interruption. 

\begin{figure}[htp]
    \centering
    \includegraphics[width=10cm]{hasel_thesis/images/notification.png}
    \caption{Break Scheduler notification}
    \label{fig:notification}
\end{figure}

\textbf{Break Reminder and Self-Reflection.} When it is time to take a break, the Break Scheduler will send a notification which will open the before Break Report. Before and after each break, there is a brief self-reflection in which the user assesses their energy and attention levels and physical well-being. These insights and an assessment of the activity yield the experience value of that break. This value is added to the activity as well as the overall and daytime category value. The experience value will help the rule-based system find good and effective breaks in the next iterations. However, the suggested breaks are only suggestions. The user does not need to perform this activity. The rule-based system suggests one activity and one randomized activity from the whole list. The user can select one of the two. If both do not fit their needs, the user can always choose an activity from the entire list, enabling them to adjust to their current needs. The selected activity is the one which will be rated and will get the assigned experience value. As discussed in the evaluation of the break section \ref{evaluation_break}, it can be challenging to quantify personal resources such as energy, attention capacity or physical well-being. A slider bar for the evaluation was used to give the participant the possibility to make a detailed rating. For simplification reasons, the step points of the slider are 5 points, which makes it easier for the user to rate their resources. For the enjoyability rating of the activity, shown in figure \ref{fig:rating}, an emoticon rating should motivate the user to place a rating.

\begin{figure}[htp]
    \centering
    \includegraphics[width=10cm]{hasel_thesis/images/rating.png}
    \caption{Break Scheduler emoticon rating}
    \label{fig:rating}
\end{figure}

During the break, the user should be reminded that they are on a break; therefore, a small window will be opened in the lower right corner, showing the remaining time and the suggested activity. This window is, per design, unmovable, not closable and always in front of everything. The only way to close this window is when ending the break by clicking on the "end break" button. This measure should force the user to take a break and prevent thoughts as "only one last mail", which lead to a switch of task and not taking a break. When the break is ended, the end break report will appear.


\textbf{Evening Reflection.} Each evening, the user will fill out the evening report, reflecting on the used energy and rating the interval and duration of the breaks. The rule-based system will consider the rating of the break interval and duration before generating the new schedule. More information on the adjustment of the interval and duration is explained in section \ref{rule-based-method-implementation}. Afterwards, the user will be asked for the start and end of their work day and preferred lunch break, ensuring all daily specific inputs are confirmed. Per default, the values of the last time will be shown, enabling users with similar schedules to continue faster. When the schedule for the next day is generated, the daily cycle will start again.

\textbf{Retrospection \& Insights.} The users are welcome to visit the dashboard or review their activities or reports in between the primary user flow. They can see all critical information at a glance on the home screen, shown in figure \ref{fig:dashboard}. On the left side, the navigation bar helps the user navigate to all pages. The user can find all main functionalities as well as the navigation to all evaluation pages, such as activities or categories, in the middle of the window, accessible with one click. "Plan new Breaks" is needed when the user did not plan the breaks the evening before, and "Start Break" can be used to start a break, which can also be achieved by clicking on the notification. The overview page displays the current break interval and duration, along with the highest-ranked activities and highest-ranked category. It was a design choice not to incorporate the overall time spent in breaks, as this can lead to an unhealthy feeling towards breaks, as taking a break is still often viewed as unfavourable or "time not spent on productive tasks". As shown in the Background chapter \ref{background}, taking breaks is essential to keep performance high and, therefore, productivity. However, such negative feelings towards breaks could be triggered by showing the overall time spent during breaks.

\begin{figure}[htp]
    \centering
    \includegraphics[width=15cm]{hasel_thesis/images/dashboard.png}
    \caption{Break Scheduler landing page}
    \label{fig:dashboard}
\end{figure}

Another analysis feature displayed in figure \ref{fig:dashboard} is the personal resources chart, which shows the energy rating from the morning report and the energy used from the evening report for each day of the week. This line chart helps users to reflect on their use of personal resources during the week. It should also enable them to find particular patterns in their personal resources. For example: If the users spend more energy during the day than they have in the morning, the users could get sick after some days as they continuously overstep their personal boundaries. Another important feature on the overview page is the display of when the next break is planned and what the suggested activity is. This essential information is displayed with a bright blue as an eye-catcher.

\textbf{Interface Test Team.} The first draft of the application was tested with a small test team of two people who were using it for multiple days. This test team was not part of the primary evaluation afterwards. The test team used the application during their regular work day, which was an excellent test to evaluate the visibility during working hours. During the test period, the test participants wrote down all bugs or questions they had, which were discussed at the end of the test in a short evaluation discussion. The participation group could find minor bugs which could be resolved for the primary evaluation. Additionally, it became apparent that an onboarding session in which the user will be supported when installing the application and given a short tour of the application decreases the error rate at the beginning. It also helps the users understand the concept more quickly and builds trust, as they know what to do from the start. These onboarding sessions were then integrated into the primary evaluation.


\section{Functionality Design}
\begin{comment} 
Description of the features and functionality of the Personalized Break Scheduler, such as break reminders, break logs, and statistics
Explanation of how these features support the overall design goals and user needs
Discussion of any technical considerations and constraints in the design process


Explanation of the application's core functionality, such as scheduling breaks, integrating with the user's calendar, and storing data
Overview of the technical architecture and how the different components of the application interact and communicate with each other
Discussion of how the functionality supports the user's goals and meets their needs, such as ease of use, accuracy, and data security
Explanation of any features or functionality that have been added or omitted based on user feedback and testing
Discussion of how the functionality has been tested and validated, including any methods used for testing and debugging
\end{comment}
This section will cover the src section of figure \ref{fig:architectur}. It includes all the main functionalities of the Break Scheduler. All features of the Break Scheduler can be grouped into four subjects: Personalisation, Rule-Based System, Calendar Integration and Database.

\subsection{Personalisation} \label{personalization}
Personalization of the Break Schedule is a crucial factor. All three key features of a break: timing, duration, and activity need, therefore, to be adjustable. That is why the user in the setup phase of the application defines the break interval and duration. Afterwards, the user has to be able to modify these values. Different features of the Break Scheduler enable the user to fine-tune interval and duration to their needs.

One way to adjust the interval and duration is by the rule-based system at the end of each day before scheduling the breaks for the next day. These adjustments depend on the user input in the evening report and are part of the primary user flow during the day. When the interval is rated as "too long", the rule-based system will decrease the interval for the following break schedule. Similar holds for the break duration. Further details will be discussed in Implementing the Rule-Based System section \ref{rule-based-method-implementation}.


Another way to regulate the interval and duration is by hand. The next generation of breaks will consider the new interval and duration. As the generated breaks are synchronised with the calendar of the user. The break timing can also be modified in the calendar, which ensures flexibility during the day and helps users adjust the break planning. More will be discussed in the Calendar Integration section \ref{calendar_integration}.


\subsection{Rule-Based System Implementation} \label{rule-based-method-implementation}

The rule-based system contains three main features: the generation of the breaks, updating the experience values and adjusting the break interval and duration.

\begin{figure}[htp]
    \centering
    \includegraphics[width=15cm]{hasel_thesis/images/rule-based_method.png}
    \caption{Rule-based system process for generating breaks}
    \label{fig:rule-based_method}
\end{figure}

\textbf{Break Schedule.} As shown in chapter \ref{design_approach}, many static and daily specific factors can influence the break schedule. However, for the current implementation, only the input values shown in figure \ref{fig:rule-based_method} are considered, as they are used for the most valuable product. Figure \ref{fig:rule-based_method} additionally visualizes the implementation of the generation process of the breaks. As input, the method needs the interval and duration, activities and categories saved by the application. The user must enter the specific start and end of the day and the duration of the lunch break. The application will automatically gather the meetings for the selected day. When all the input is available, the system will schedule a short break for the morning report at the start of the day. After that, the first break start time is defined as the start of the day plus the interval. Now the method first checks whether the timing with the break duration is in a meeting. If so, it reschedules the timing after the meeting, updates the break duration, as it could already be lunchtime, and checks again if the break is in a meeting. The activity will be selected if the planned break is not in a meeting. Each activity has a minimum and maximum duration and an experience value. The minimum and maximum duration is used when selecting an activity for a certain break duration, as it is, for example, impossible to do outdoor training in a 5min break. The category also has an overall experience value and a morning, lunch and evening experience value. This helps the rule-based system with the selection of the category depending on the daytime. Before the rule-based system selects the activity, the category with the highest experience value for this daytime will be selected. Afterwards, the activity from this category with the highest experience value, which fits the given duration, will be selected. However, if a category or activity was not yet tried out, it will be selected first. When the activity is chosen, one random activity out of the whole list will be supplied. This random activity should add some randomness to the process and reduce possible bias from the rule-based system. When the suggested and the random activities have been selected, the break with all necessary information will be saved, and it will be checked whether the end of the day is already achieved. During the last interval of the day, no breaks are scheduled to enable users to finalize their work. When the last interval of the day is reached, a short break for the evening report will be scheduled. All scheduled breaks will be saved in the database, and most importantly, all breaks will be scheduled in the user's calendar. The process is finished when the user can see all breaks in their personal calendar.

\textbf{Update the Experience Values.} Before and after each break, the user will fill out a short self-report, which asks for the selected break activity and about the current state of their energy (e), attention level (a) and physical well-being (p). After the break, the user also rates the activity (r). In order to reflect how beneficial a certain break activity is perceived, the Break Scheduler uses past experience values. The rule-based system will execute the following calculation to refine the experience value of a break:

$$ rawExpValue = \frac{e_1- e_0}{10} + \frac{a_1- a_0}{10} + \frac{p_1-p_0}{10} + \frac{r}{10} -5$$

$$ expValue = Round(rawExpValue/5)*5$$

Where $e_0$, $a_0$ and $p_0$ are the values before the break and $e_1$, $a_1$ and $p_1$ the values after the break. As $e$, $a$, $p$ and $r$ are values between 0-100, the differences will be divided by 10. As the value $r$ should also be negative if the activity was not perceived well, 5 will be subtracted. Therefore the value r can go from -5 to 5. The $rawExpValue$ is the raw experience value, which needs to be rounded to the next number that is divisible by 5 in order to group more activities together. The $expValue$ will be added to the experience value of the selected activity and the overall and daytime experience value of all related categories of the selected activity.

\textbf{Update Break Interval and Duration.} Each evening, the user will be asked to fill out the evening report where they rate their used energy and the break interval and duration. The user can select whether the interval is suitable, too long or too short; the same holds for the break duration. According to the user input, the following calculation will be done:

"Too Short" for interval (i): $ i = i + Round(i/10) $

"Too Long" for interval (i): $ i = i - Round(i/10) $

"Too Short" for duration (d): $ d = d + Round(d/3) $

"Too Long" for duration (d): $ d = d - Round(d/3) $

If the interval or duration is rated as "good", no changes will be done. This simple calculation was chosen to ensure a relative change, which helps the user find a specific target time, as both values can, step by step, level off to the target. The chosen relative change of the duration is higher (1/3 instead of 1/10)  than the interval, as the duration values usually are smaller and the same relative change (1/10) would not have an impact on the break duration: e.g. 10min break duration would be 11min if the duration was too short. With the greater relative change, the new duration is 15min. This new interval and duration will be used to create the new break schedule.


\subsection{Calendar Integration} \label{calendar_integration}

Synchronising the Break Schedule with the private calendar of the user has multiple advantages. First, the number of simultaneously used applications can be reduced, as all breaks can and should be adjusted directly in the users' calendar. This enables users to have an overview of all their planned work, meetings and breaks, which helps them to quickly adjust to upcoming changes during the day. Additionally, it helps co-workers to know when breaks are planned, so they can schedule meeting around or eventually join certain breaks. In order to integrate the calendar into the Break Scheduler, two APIs were used: Google API \cite{googleAPI} and Microsoft Graph API \cite{graphAPI}. Both APIs ensure the automatic synchronisation between the Break Scheduler and the user's personal calendar. 

\textbf{Google API.} Google provides an API to connect their Google calendar using their OAuth2 Client API. A new Client needs to be opened in the google cloud to use the OAuth Client. As the Break Scheduler is still in the test phase, the user must be entered manually into the Google client, enabling them to connect it with the Google calendar. When selecting the Google calendar, the user will be redirected to the Google login screen, where they can log in and give the application permission to read and adjust calendar data. The API will generate a token saved by the Break Scheduler, ensuring that the user only needs to log in once.

\textbf{MS Graph API.} Similar to the google API, the first step to using the MS Graph API is registering an app in the Azure directory. When users select the Microsoft Calendar, they will be redirected to the Microsoft login page, where they can log in and give the app permission to read and change calendar data. The API gives back a token which is saved by the Break Scheduler. However, in contrast to the Google API, the user needs to log in each time the application is started new.

Each minute the Break Scheduler will send a request to the API and update the user's breaks and meeting data to account for changes during the day. The notification sent by the Break Scheduler will be adjusted accordingly.

\subsection{Database} \label{database}
As already described, the database is a simple SQLite database, which is suitable since all information is saved locally on the user's machine, which reduces data privacy issues. The database considers different tables as shown in figure \ref{fig:db}, which can be found in the appendix. The Break Scheduler actively uses Activities, Categories, and Settings tables to generate new breaks (light blue). In contrast, the tables for the Break Reports and the Daily Reports are stored to show them to the user (dark blue). The Logs and the Break table are currently only used for analytical purposes in the preliminary evaluation (purple).





\begin{comment}

%Nun sollte ein (separates) Implementation-Kapitel folgen (ich glaube das hast du in 4.0.1 vorgesehen). Hier kannst du beschreiben, wie du die Key-Konzepte effektiv implementiert hast.
\textcolor{red}{all not yet finished an do not know what to say here}
- Overview of the Break Scheduler software
- Description of the software's features and functionalities
- Explanation of the software's design and development process

\end{comment}


\chapter{Preliminary Evaluation} \label{primary_evaluation}
%Dann folgt ein Preliminary Evaluation Chapter: hier beschreibst du zunächst die Methode und anschliessend Resultate
This chapter describes the method and procedure used in this preliminary evaluation. The goal of this primary evaluation is to investigate:

\begin{itemize}
    \item The impact of self-reporting on users' awareness on existing break habits and personal resources (RQ1)
    \item How helpful the Break Scheduler is to identify good break activities (RQ2)
    \item The impact on developing tiny habits of taking breaks that recharge personal resources. (RQ2)
    \item Feedback on the experience with the Break Scheduler (RQ3)
\end{itemize}

This section provides a detailed explanation of the sample selection, study design, data collection procedures, and data analysis methods used to answer the research questions stated in the introduction, enabling the reader to reproduce the experiments when needed. The following subchapters provide more information on each aspect of the primary evaluation.

\section{Study Design}
\begin{comment} 
- Explanation of the experimental design, including control and intervention groups, pre- and post-tests, and other relevant details
\end{comment}

This preliminary evaluation tests the Break Scheduler approach for its effectiveness and other things, such as their experience with the tool. To achieve this, the baseline of each participant was gathered by a pre-intervention questionnaire, which asked about demographic data as well as their self-assessment of their personal resources and daily break habits. This within-subjects design ensures each user acts as their own control, enabling a pre-and post-intervention evaluation to answer the research questions. The selection of participants was done by a word-of-mouth approach in combination with a snowball sampling method. Participation in this study was voluntary and could be stopped at any time. Besides the opportunity to gain insights into their own break habits and improve them, no monetary compensation was offered to participants. This study consists of three phases: pre-intervention, intervention and post-intervention phase. 

\textbf{The Pre-Intervention Phase.} This phase provides insights into the participant's demographical data and their awareness of their personal resources and break habits. This phase refers to the control condition to define the pre-intervention state of each participant. The participants were asked to complete the pre-questionnaire in the form of an online survey with different questions about their work environment, their existing break habits and their awareness of personal resources. Additionally, it also focused on their existing knowledge on preferred break activities. This questionnaire represents the baseline of the user's awareness of personal resources and break habits without the influence of the Break Scheduler. The survey also includes demographic questions (e.g., age, gender, occupation, work background) to control for potential confounding variables. The questions of the pre-intervention questionnaire are attached in the appendix \ref{pre-intervention-questionnaire}.

\textbf{The Intervention Phase.} At the start of the intervention phase, a short onboarding session was held with each participant to install the Break Scheduler application and provide a brief introduction to its use. Participants were shown how to connect the Break Scheduler with the calendar for synchronisation and how to fill out the different self-reports and the applications' main features. Subsequently, they were asked to use the Break Scheduler for at least five workdays during their regular work. The Break Scheduler app includes one morning and one evening self-reflection questionnaire and a short pre-and post-break self-reflection questionnaire. All data produced during the intervention is only stored locally on the user's machine and can only be accessed by them. However, as part of the consent form, all participants have signed that they grant access to their database file after the intervention for study purposes.

\textbf{The Post-Intervention Phase.} After completing the intervention phase, participants were requested to answer the post-questionnaire, where they were asked about their experience with the Break Scheduler app, their learning, and potential ideas for improvements. More specifically, they were asked about the impact of the Break Scheduler on their personal resources and break habits, such as interval, duration and break activities and were asked about specific features of the Break Scheduler, such as the notification or self-reports. They were also inquired about the advantages and disadvantages of the implementation. In this post-intervention state, participants rated their awareness of their resources and break habits again and rated the impact of the Break Scheduler. The pre-and post-intervention were compared in an evaluation analysis to answer the research question. The questions of the post-intervention questionnaire are attached in the appendix \ref{post-intervention-questionnaire}.


\section{Participants}
\begin{comment} 
- Description of the sample size, criteria for inclusion/exclusion, and demographics of the participants

\end{comment}
 The study includes a sample of 13 knowledge workers with different backgrounds (students and employees) working in various industries, ranging from IT, chemistry, and mathematics to education, aged 20-31, living and working in the German-speaking part of Switzerland as shown in table \ref{table:demographics}. Participants were recruited through a word-of-mouth approach and a snowball sampling method by asking friends and acquaintances if they would be willing to participate in the study. A total of 16 knowledge workers agreed to participate. Out of these 16, two participants were used for a first test run to identify first bugs and improve the user experience, which eliminated them as participants in the primary evaluation. One of them needed to cancel the study during the intervention phase for personal reasons. The data generated by this user is therefore excluded from the results. It led to a total of 13 participants fulfilling all three mandatory stages.

 \begin{table}[h]
\centering
 \caption{User demographics}
    \begin{tabular}{ |p{3cm}||p{3cm}|  }
     \hline
     \multicolumn{2}{|c|}{User Demographics of N=13 participants} \\
    
     \hline
     Age & 24.8 (min. 20 max. 31) \\
     \hline
     Sex &   7 Male / 6 Female \\
     \hline
     Intervention period (with weekend) & 10.2 days (min. 5, max. 13)\\
     \hline
     Extroversion & 4 extroverts, 9 introverts \\
     \hline            
    \end{tabular}
 \label{table:demographics}
\end{table}
 
 \textbf{Selection Criteria.} Each participant was asked to participate with their main work laptop/computer as the application needed to be incorporated into their daily work. As the Break Scheduler is a desktop application, one selection criteria was for the participants to be able to install this application on their work device. Within this sample group, all three main operating systems ( macOS, Windows and Linux) were represented. 

 
\section{Data Analysis}
In this study, various data were collected and analysed in different formats to gather the findings in the Result chapter \ref{results}. In order to establish a common understanding of the methods used, they will be introduced in this chapter.

\textbf{Data collection.} In this study, a total of 13 pre-intervention and 13 post-intervention questionnaires were considered in the results. Over the intervention period, a total of 154 breaks, 62 morning and 81 evening reports were collected. This collected  data can be split into two types: Quantitative and Qualitative data. Quantitative data was gathered in each user's database file and was generated by the user when using the Break Scheduler in the intervention period. Examples of quantitative data are the experience values of the categories and activities and the break interval and duration of each participant. Additionally, each before and after break Report requires a quantification of the user's current personal resources, which is also saved in the database and used for analytical purposes. The number of how many reports or breaks taken during the intervention period can be concluded with the data in the database. On the other hand, qualitative data is gathered to underline the results from the quantitative data and give more detailed insights into the participant's experience. The qualitative data is provided by the pre-and post questionnaire, which asks about the existing break habits and the impact of the Break Scheduler. Qualitative data will be analysed using thematic analysis \cite{Braun.2006}. 

\textbf{Data cleaning and preparation.} As one of the participants needed to abort the intervention phase, all gathered data of this participant was removed. In the results, only data from participants who completed the whole intervention period were considered.

\textbf{Quantitative Data Analysis Methods.} For the quantitative data analysis, descriptive statistics were used to calculate the average, minimum and maximum values of the data types and provide an overview and trends within the data. Different data visualisations as a violin or an agreement plot, were used to give more insights into the quantitative data. A violin plot is used to display the distribution of a continuous variable. This plot shows the mirrored density plot on either side of the central plot, representing the data's interquartile range (IQR). The data density defines the width at each point along the range of values. The violin's edges represent the data's minimum and maximum values. 

\textbf{Qualitative Data analysis methods.}
In order to find key concepts and find commonalities, the qualitative data were analysed by thematic analysis \cite{Braun.2006}. For each qualitative question, specific codes were defined that matched the participants' answers. Multiple of these codes could be assigned to the quote of each user, which should abstract their response. The codes were discussed with a second researcher to develop the themes that describe the results. The count number of each code states how relevant the statement is in relation to other statements. They are also stated in the Results chapter \ref{results}, which should enable the reader to estimate the weight of each statement.


\begin{comment} 
- violin plot
- agreement plot

- different data input
- Explanation of the statistical methods used to analyze the data, including descriptive statistics, inferential statistics, and any other relevant techniques

The data analysis section of your study is where you describe the methods you used to analyze the data you collected. This section should include the following elements:

Data Cleaning: Describe any steps you took to clean the data, such as removing missing values, checking for outliers, or transforming variables.

Descriptive Statistics: Provide summary statistics for the demographic and pre- and post-intervention data, including means, standard deviations, frequencies, and ranges.

Inferential Statistics: Describe the inferential statistical methods you used to analyze the data and test your hypotheses. For example, if you were comparing the mean level of personal resource recharge between the intervention and control groups, you might use a paired t-test or a repeated measures ANOVA.

\end{comment}

\chapter{Results} \label{results}
\begin{comment} 
Summary of the findings from the study, including any significant effects of the recharge breaks on personal resource recharge
\end{comment}
This chapter describes the key findings that resulted from the preliminary evaluation. They can be grouped into six key themes, summarized in Table \ref{table:key_findings} and detailed in the following subchapters.

\begin{table}[ht] 
 \caption{Summary of the Study Key Findings}
    \begin{tabular}{ |p{1cm} p{10cm} p{1cm} p{2cm}|  }
    \hline
     \# & Findings & Sect. & Related RQ\\
     \hline
     \rowcolor{Gray}
     F1 &   Regular self-reports on personal resources increase users' awareness and mindfulness on how to spend their (limited) personal resources. & \ref{awareness_resources} & RQ1\\
     F2 &   Self-reporting personal resources help users to find beneficial break activities. & \ref{beneficial_breaks} & RQ1\\
     \rowcolor{Gray}
     F3 &   The proactive planning of breaks before starting the next workday helps to be more mindful of them and helps to give more structure to the day. & \ref{planning_breaks} & RQ1\\
     F4 &   Break reminders help users to raise awareness of their break habits and remind users to take breaks. & \ref{notification} & RQ1, RQ3 \\
     \rowcolor{Gray}
     F5 &   Taking more beneficial breaks improves the personal resources levels of the participants. & \ref{resources} & RQ2\\
     F6 &   Break habits are very personal and differ from person to person. A tool supporting the user to create a beneficial break habit should enable personalization. & \ref{personal_break_habits} & RQ2, RQ3\\
      \rowcolor{Gray}
     \hline
     
    \end{tabular}
 \label{table:key_findings}
\end{table}

\section{Awareness of Personal Resources} \label{awareness_resources}
\begin{comment} 
Self-reporting personal resources multiple times a day creates more awareness, helps to be more mindful about how to spend the limited personal resources and helps the user to find beneficial break activities (RQ1, RQ2, RQ3)
\end{comment}

\textbf{Relationship Between Resource Deficits and Reflection Habits Among Participants.}
The analysis of the pre-intervention questionnaire showed that most participants have different existing habits of reflecting on their resources. Generally, participants reflect on their personal resources every other day (5x) or barely (4x), and only a few reflect on them every day (2x) or multiple times per day (2x).  Additionally, most (11x) do not use a tool to reflect on the resources, only one uses an app, and two use an activity tracker.  Most of the participants (9 out of 11) only reflect on their resources when they are low or they have a resource deficit, as underlined in the following statement of one of the participants: \textit{"If they are either really low or really high, then I am more aware of it.  If "normal", then I do not think about it."} -[S06]. During the day, most users are not aware of their personal resources.  Some of them, however, reflect on their resources in the evening: \textit{"I'm generally aware of a lack of personal resources at the end of a given day.  During the day, I'm very unaware of my personal resources"}-[S11].  Also, external factors such as social contact (3x) or high work demand (3x), can motivate the reflection on personal resources: \textit{"On days or during weeks where a lot of work has to be done, I am very aware of my personal resources and their limitations." } -[S03]. As discussed in the Background chapter \ref{background}, a resource deficit can also lead to a higher demand for resources, which can be confirmed by some participants' experience: \textit{"Mostly in situations of high demand or stress.  I get more sensitive to my energy level, physical well-being, and my general ability to socialize when I am under a bit of pressure.  Mostly I feel that I don't have as many resources left, which makes me even more stressed.  If I am in good spirits or have fun activities to do at work, I don't feel my energy levels drain as quickly, and I can work for a long time without taking breaks and don't feel too tired in the evening." }-[S15]. Therefore the baseline of the awareness of the users on personal resources can be increased.

\textbf{Increased Awareness of Personal Resources Through Self-Reporting.} The data collected in the post-intervention questionnaire shows that the participants reflected more on their personal resources during the intervention period, which helped most of them (13x) to reflect on their use of resources and how to act on them (8x). In total, the participants reported a total of 154 breaks during the intervention period. Over 62 evening and over 81 morning reports were collected, in which each participant reflected on their current personal resources and used resources during the day.  

\begin{figure}[htp]
    \centering
    \includegraphics[width=12cm]{hasel_thesis/images/agreement_1.png}
    \caption{Post-questionnaire answers to awareness}
    \label{fig:awareness_agreement}
\end{figure}

 Figure \ref{fig:awareness_agreement} visualizes that most participants did reflect multiple times per day during the intervention period, whereas only a few did before the intervention, as described in the previous section. Most of them (11x) mentioned that the self-reflection helped them to be more aware of their personal resources (13x). Additionally, this helps to learn what influences them (10x): \textit{"These types of questions force me to be mindful about my [personal] resources." }-[S11]. The reflection and the increase of awareness helped some of them (8x) to react when the resources are low and on how they spend their personal resources: \textit{"It works very well, since you report it in the moment."} - [S01]. More than 50\% (9x) agreed on having learned something about their use of personal resources with the help of self-reporting. As shown in figure \ref{fig:feature_ratings_reflecting}, specifically, the morning and evening reports were used by many (7x) users for self-reflection purposes: \textit{"The reports in the morning and evening were really good for self-reflection on how I felt, which I barely ever do otherwise."} -[S06]. 

\begin{figure}[htp]
    \centering
    \includegraphics[width=7cm]{hasel_thesis/images/feature_rating_reflecting.png}
    \caption{Answers of all 13 participants: Which features helped you the most to reflect or learn about your personal resources?}
    \label{fig:feature_ratings_reflecting}
\end{figure}

It can be concluded that the awareness of the personal resources of the user did increase after the intervention week, which shows the positive effect of the Break Scheduler. The morning and evening reports motivate the participants to reflect over the whole day, which helps them to be more aware of the use of personal resources. Therefore, it follows:

\begin{tcolorbox}[colback=white!5!white,colframe=black!75!black]
  \textbf{F1:} Regular self-reports on personal resources increase users' awareness and mindfulness on how to spend their (limited) personal resources.
\end{tcolorbox}


\section{Awareness of Beneficial Breaks} \label{beneficial_breaks}

When asked about the activities chosen in breaks before the intervention week, 9 participants tried to actively engage in activities which are beneficial for them. However, some participants (4x) did not proactively choose beneficial breaks, and 7 out of these mentioned 9 participants only engaged sometimes in beneficial break activities.

\textbf{Self-Reports as a Tool to Reflect on Break Activities.}
The before and after break self-report was perceived positively and useful for reflecting on personal resources by most users (9 out of 13), as shown in figure \ref{fig:feature_ratings_reflecting}. Additionally, the Break Scheduler could help learn about activities which are good as well as activities which do not help to recharge resources. As described in the Approach of the Personalized Break Scheduler in chapter \ref{design_approach}, the short self-report before and after each break was split into the three self-reports on energy, attention and physical well-being, which was well perceived by some participants: \textit{“I very much liked how the personal resources were split up into 3 different dimensions. Usually, I would think of myself in one dimension: "tired" or "energized". During the study, I would map physical aches such as a sore throat, tired eyes, or tension to "body", and mental exhaustion to "attention". The third category, "energy", was the overall feeling at the moment combining the mental and physical aspects.”}- [S15]. The increase in the awareness of personal resources helped some users to act accordingly: \textit{“During the day when I felt tired/exhausted I thought about if it might be more effective to take a break to be more efficient again afterwards.”}-[S06]. Therefore, it can be suggested that the awareness of personal resources had a positive effect on the awareness of the user and on taking beneficial breaks.

In summary, the self-reflection of the current state of the personal resources, such as energy, attention and physical well-being, through the before and after break reports gave the participants insight into which activities recharged their resources and which were not. This awareness helped the participants to find beneficial breaks. From this follows that:

\begin{tcolorbox}[colback=white!5!white,colframe=black!75!black]
  \textbf{F2:} Self-reporting personal resources helps users to find beneficial break activities.
\end{tcolorbox}

\section{Pro-active Planning of Breaks} \label{planning_breaks}
\begin{comment} 
Pro-active planning of breaks helps to be more mindful of them and helps to give more structure to the day (RQ1, RQ2, RQ3)
\end{comment}

In the pre-intervention questionnaire, 11 of the 13 participants stated that they did not plan their breaks in advance, and most of them were not taking regular breaks. \textit{"I had no break habits before.”}-[S10] as one participant stated. All participants have time periods during which they could plan within-day breaks; however, only two participants stated that they plan breaks in advance. These two also mentioned they either have fixed team breaks or arrange breaks with others through chat. In figure \ref{fig:number_breaks_prestudy}, it can be shown that the number of breaks during the day varies. About half of the participants (7x) take two and one lunch breaks a day. Others (3x) take more breaks than that, and a group of participants take only lunch breaks (3x). Still, many (9x) are referring to feeling tired and sleepy at the end of the day. This shows that their resources are depleted and need recharging. As discussed in the Background chapter \ref{background}, one way to do this is by taking regular breaks.

\begin{figure}[htp]
    \centering
    \includegraphics[width=10cm]{hasel_thesis/images/number_breaks_prestudy.png}
    \caption{Overview of how many breaks were done before the intervention}
    \label{fig:number_breaks_prestudy}
\end{figure}

\textbf{Benefits of planning ahead.} During the intervention period, many participants (9x) took more breaks, and some stated that planning the breaks in advance is one of the key features of the Break Scheduler as the breaks got more intentional and helped with \textit{"remembering that breaks as a concept exist.” }-[S11]. One participant mentioned: \textit{“So overall, I took more breaks. But I would say that I took longer breaks before as well. And only through this tool, I realized that this was a good thing.”}-[S15]. Even though some (5x) mentioned that they took fewer or shorter breaks, the breaks which were taken were more intentional: \textit{“My breaks got more intentional with the Break Scheduler, and I actually took some time to do something beneficial and was then more concentrated again afterwards so I did not have to take a mini-break sooner again.”}- [S06], \textit{“I undoubtedly took more breaks. But I believe they stayed around the same length. With being more rested, I do believe I have fewer 2 hours scrolling through social media sessions”}- [S11] or \textit{“I think I took breaks with more awareness: When it said to take a break, I might have talked or walked around or something anyway, but now I actively thought of it as a break.”}-[S17]. The scheduling of the breaks into their calendar also helps some participants to be more disciplined with taking breaks: \textit{“The Break Scheduler helps me staying disciplined when it comes to taking breaks”}- [S04]. Most participants (10x) mentioned that using the Break Scheduler did not impact their work or work environment. One person even mentioned that he gets more things done with the Break Scheduler: \textit{“I got more work done in the morning” }–[S04]. Planning breaks ahead helps many (8x) to be more mindful about taking regular breaks. Additionally, some (4x) appreciated the fact that the scheduling of regular breaks helps to structure their day: \textit{“It sets my whole working day under a clear structure which sometimes can be hard to follow. Because of the tool, I did more breaks.”}-[S13] or \textit{“Definitely improved my break habits. And during the two off days, my breaks were quite messed up. And it's even helpful when you can't stick to the schedule perfectly.”} – [S15]. 

\textbf{Maintaining flexibility throughout the day.} Even though breaks are planned ahead, participants need and want to stay flexible during the day. The feature of rescheduling breaks helps to adjust the breaks during the day: \textit{“It gives me some structure but doesn't enforce it due to the fact that it's possible to move pauses around.”}-[S13]. Nonetheless, planning the exact time and duration of breaks in advance can also lead to inflexibility during the day, as the breaks need to be rescheduled: \textit{“I took more breaks, but often had to skip them or move them around”} –[S00], \textit{“In the afternoon normally I might have taken more breaks but the scheduler told me not to.”}-[S17] or \textit{“The process itself was not, but sometimes it was a little annoying to change the schedule 3 times a day when you haven't had the chance to take the break when originally planned.”}-[S10]. Therefore, the Break Scheduler could be improved by redesigning the rescheduling process. The main factors why participants did not take a break are shown in figure \ref{fig:reasons_why_not}. High workload, the timing did not fit, or they did not see the notification are mentioned most often: \textit{"I think it helped me be mindful of taking breaks and highly increase the rate of breaks I took. Nevertheless, I was too often choosing to ignore the break, in favour of a task or project that was approaching a deadline."} -[S11]. Even though not all participants could take more breaks due to the mentioned reasons above, it helped them be more aware of that: \textit{“It reminded me that I SHOULD take more breaks and take more care of myself.”}-[S02]


\begin{figure}[htp]
    \centering
    \includegraphics[width=8cm]{hasel_thesis/images/reasons_why_not.png}
    \caption{Overview of Participants' self-reported break habits before the intervention}
    \label{fig:reasons_why_not}
\end{figure}

To sum up, the Break Scheduler helped the participants to plan regular breaks into the work day, which increases the awareness of breaks and forces the user to structure their work day in advance, which helps some of them to have an overall better work structure during the day. The following statement can be concluded:

\begin{tcolorbox}[colback=white!5!white,colframe=black!75!black]
 \textbf{F3:} The proactive planning of breaks before starting the next workday helps to be more mindful of them and helps to give more structure to the day.
\end{tcolorbox}


\section{Break Reminder} \label{notification}
\begin{comment} 
Notification helps to raise awareness and remind users to take breaks (RQ1, RQ2, RQ3)

Quantitative:


Qualitative:
- only regular lunch break through most participants (8x) is a lunch break
- Some 
- other than that most take irregular (4x) or only necessary breaks (3x). 
Some do take 

\end{comment}

Some of the participants do not take regular breaks, also since they are not reminded to take any breaks. As stated in the Background chapter \ref{background}, knowledge workers often forget to take breaks and overwork themselves. This is also the case for some of the participants. As already stated in section \ref{planning_breaks}, most participants (11 out of 13) stated not taking any regular breaks except the lunch break.

\begin{figure}[htp]
    \centering
    \includegraphics[width=10cm]{hasel_thesis/images/feature_rating.png}
    \caption{Answers (13 participants): Which features helped you the most to reflect or learn about your break habits?}
    \label{fig:feature_ratings}
\end{figure}

Interestingly, in the post-questionnaire, the notification was rated (8x) as the most valuable feature of the Breaks Scheduler for reflecting on break habits as shown in figure \ref{fig:feature_ratings}. Additionally, many participants (8x) stated that the notification helped them to be reminded about breaks: \textit{“Notifications reminded me that I had not taken a break and that maybe I should.”}-[S17], \textit{“The notifications were really good when working on the computer anyways as I did not have to keep track on time, but actually got a notification.”}-[S06] or \textit{“The notifications reminded me most that I should take a break (they interrupt my workflow otherwise I wouldn't even think of it)”} –[S11]. The reminder notification from the calendar itself was an additional reminder which also helped one participant to push through his tasks in order to complete them before the break: \textit{“I liked that breaks were synced with my calendar and oftentimes, the calendar notifications (event coming up in 30 minutes) helped me to push through a task. I would know that I deserved a break according to calendar/bScheduler if I sticked through.”}-[S15]. On the other hand, the notification also enforces a specific timing, making it inflexible, which was unfortunate for one participant. While one prefers to have fewer reminders, another one likes to have even more reminders to enforce the action for a break when the work demand is high: \textit{“The more notifications I get, the more I can be reminded of taking a consciously taken break. It would have been nice to have more reminders/ notifications on my phone and laptop. One was not enough to get me focused on my break and personal resources.”}-[S02]


To conclude, the notification of the Break Scheduler helped many participants to remind taking breaks which helped them to be more aware of their break habits. It can be stated:

\begin{tcolorbox}[colback=white!5!white,colframe=black!75!black]
  \textbf{F4:} Break reminders help users to raise awareness of their break habits and remind users to take breaks.
\end{tcolorbox}




\section{Improvement of Personal Resources} \label{resources}
\begin{comment} 
Taking more beneficial breaks improves the personal resources levels of the participants (RQ2, RQ3)
\end{comment}

\begin{figure}[htp]
    \centering
    \includegraphics[width=12cm]{hasel_thesis/images/agreement_3.png}
    \caption{Pre- and post-intervention questionnaire answers to sleepiness at the end of the day}
    \label{fig:sleepiness_agreement}
\end{figure}

In the pre-questionnaire many participants mentioned they feel often tired or sleepy after a work day, while not all of their breaks are actively chosen to recharge their personal resources, as stated in \ref{beneficial_breaks}. In figure \ref{fig:sleepiness_agreement}, most participants (9x) stated  they often feel tired or sleepy at the end of the day, whereas only one disagreed with this statement. About half of the participants (6x) can feel how their personal resources deplete over the course of the day: \textit{“Almost without exception they continually decrease until end of workday when I can recover some of them through personal activities. morning activities ameliorate this problem somewhat.}-[S00] or \textit{“I can feel my personal resources depleting during work sessions by noticing that I am very easily distracted or not able to concentrate on a task.”}-[S03]. This depletion can affect their social skills:\textit{“I can feel that if I get tired I feel less the need of talking to people. Additionally my skills to verbally express myself decrease.” } -[S13]. However, many participants (5 of 11) state that the depletion also depends on the tasks they are working on as well as their motivation: \textit{“However, personal resources also depend on the tasks and how motivated I feel to do them.”}-[S06]. \textit{“Certain activities consume resources, others recharge them. A good day incorporates balancing this shift.”}- [S06]. 

\textbf{Beneficial Activities with high Effort.} In the pre-questionnaire, many participants already knew some activities which are beneficial for them but need some motivation to do them. More than half (8x) mentioned body movement or (6x) going outside as activities, that are beneficial but need some initial effort to do them: \textit{“Sometimes I do sports during my lunch break, but that needs some motivation and mostly some other people that join otherwise I normally don‘t do it.”}-[S10] or \textit{“Yes, going outside, since you have to get ready.”}-[S03]. But also relaxation activities such as (5x) meditation or (2x) reading can be beneficial for some of them, but need some initial motivation to do: \textit{“Yes, Meditation often gets lost in the trouble of the day despite it helping me a lot”}-[S00] or \textit{“Reading/Meditating are also activities that require some initial motivation for me and I don't typically do them all that often.”}-[S15]

\begin{figure}[htp]
    \centering
    \includegraphics[width=13cm]{hasel_thesis/images/agreement_2_v2.png}
    \caption{Post-questionnaire answers to the impact of the breaks during the intervention}
    \label{fig:impact_breaks_agreement}
\end{figure}

During the intervention period, when the awareness of beneficial breaks increased, as shown in section \ref{beneficial_breaks}, many participants felt less tired and could see an increase in their energy, attention, and physical and mental well-being. Figure \ref{fig:sleepiness_agreement} shows that only less than half (5x) agreed they feel often sleepy and tired, whereas now 6 participants disagree. But not only an overall effect can be stated, but also a small effect on the measured key aspects: energy, attention level and physical and mental well-being. In figure \ref{fig:impact_breaks_agreement}, more than half of the participants agreed that they can feel an improvement in their attention, energy and physical or mental well-being. This can also be underlined by  figure \ref{fig:change_break}, which shows that there is an average small increase after each break in all of these areas. The measurements during the intervention yield that the biggest impact from the break had the attention levels of the users, which state on average nearly 20 points increment. Shortly after, follows the energy level, which has around a 10 points increment on average. Physical well-being has still a positive average but is close to 0 points net change value. 

\textbf{Activities that Recharge Personal Resources.} Finding new beneficial activities is one of the goals of the Break Scheduler. Some participants (3x) got aware that social breaks or breaks where they are going outside are beneficial breaks for them: \textit{“Social interactions help me recharge my personal resources, I didn't know that before"}-[S04] or \textit{"Making the effort and going outside to get some fresh air is worth it. I feel energized after and my attention is much improved."}-[S03]. 

\textbf{Activities that Deplete Personal Resources.} One person learned that breaks with domestic tasks are not considered as a break, as it feels like another work and does not recharge the personal resources: \textit{"I do not like to do different work for my breaks like doing domestic tasks or socializing, which all takes energy. I want to do something where no energy from my side is needed, otherwise it doesn't feel like a true break for me, but just different work".}-[S01]. Not only is it important to know what activities are beneficial but also what activities are not. Most participants (8x) could learn something about activities, which are not beneficial. The most named activity was spending time on their phone (4x): \textit{"Beeing on my phone (staring into another screen) during my breaks is definitely not increasing personal resources.”}- [S06]. Other unbeneficial activities mentioned are domestic tasks. 

\begin{figure}[htp]
    \centering
    \includegraphics[width=8cm]{hasel_thesis/images/change_break.png}
    \caption{Net experience values change (y-axis) of energy, attention and physical well-being (x-axis) after a break}
    \label{fig:change_break}
\end{figure}

The Break Scheduler could help the participants to find more beneficial breaks, which helped them to recharge their personal resources and increase their energy, attention, physical and mental well-being. To sum up, the following sentence can be stated:

\begin{tcolorbox}[colback=white!5!white,colframe=black!75!black]
 \textbf{F5:} Taking more beneficial breaks improves the personal resources levels of the participants.
\end{tcolorbox}


\section{Personal Break Habits} \label{personal_break_habits}
\begin{comment} 
Self-reporting personal resources multiple times a day creates more awareness, helps to be more mindful about how to spend the limited personal resources and helps the user to find beneficial break activities (RQ1, RQ2, RQ3)
\end{comment}
One key element shown by the collected data is the diversity of preferences of the users. To get a better overview of the personalization of break habits, two aspects will be discussed: timing and duration, as well as activities during the break.


\begin{figure}[htp]
    \centering
    \includegraphics[width=14cm]{hasel_thesis/images/break_Interval_Duration.png}
    \caption{Distribution of the different break intervals and duration according to the database}
    \label{fig:interval_duration}
\end{figure}


\subsection{Timing and Duration}
The collected data can show that the preferences of the participants regarding break interval and break duration differ from person to person and depend on different circumstances. 

\textbf{Initial Timing Preferences.}
Most participants (8x) have high or normal job control, enabling them to plan their tasks and breaks freely. Additionally, most of them (11x) feel supported by their employer to take breaks during working hours, which leads to a solid foundation for creating regular break habits. For the break interval, eight participants had an idea of a rough interval duration, while others (5x) could not state any interval. The range of break intervals stated in the pre-questionnaires goes from 20min to 4 hours without any breaks. About half of the participants (6x) stated that it strongly depends on the circumstances: \textit{"Really depends on a lot of factors. Like how motivated I am, how well rested I am, how close the deadline is, the weather,...” }-[S06] or \textit{“I don‘t have a special focus time I would say. I think it depends more on the task. I have to get into a task, and when I am really working on something, I often want to finish this task if possible.”}-[S10]. Similar behaviour was reported for the duration. Some (8x) of the participants could state a rough duration number which can differ between 10-15 (4x) and 15-30 (4x) and  45-60 (1x) (for a lunch break). One participant stated: \textit{“I guess a ten-minute break is enough. For sure the duration depends on my mood and the tasks I have to do.”}-[S13] while another mentioned: \textit{“The longer the better. Maybe 30 minutes as a minimum.”}-[S15] Around five people could not state any fixed break duration and stated that it depends on the situation or other aspects, as if they take a break alone or with others. \textit{“If I am by myself, breaks are usually shorter as I get bored after a while. But in a social setting, the length depends a lot on the situation.”}-[S06]. 

\textbf{Circumstances Influencing the Timing of a Break.}
Most (10x) participants agreed that the break duration is dependent on certain circumstances, whereas a minority (3x) did not think so. Most of the mentioned factors are the time of the day (2x), mood (2x), tasks (2x), and interruption, for example, open office space (1x), but also the menstrual cycle (1x) or the food and beverage intake (1x).\textit{“It depends on the task I am working on, some tasks require more concentration or attention.”}-[S03]. \textit{“Ideal would be until the task is done, then a break and afterwards continue with the next task, so yes it does depend.”}-[S06]. An important factor which was mentioned (3x) is motivation. If the motivation is high, the break interval is mostly longer: \textit{“Motivation often plays a major part. High motivation can significantly extend attention while the lack thereof can inversely necessitate longer breaks.”}-[S07]. Some participants could also already feel the need for higher breaks after an extended focus period: it depends on \textit{“.. how burnt out I am from working on high focus for longer periods of time (that's when I'm usually in need of holidays)”}-[S02].

\textbf{Break Scheduler Supporting the Different Timing Preferences.}
After the intervention period, most participants (9x and 2-unanswered) could learn something from the Break Scheduler regarding their break habits. Some (2x) could find out that they need to take more breaks, and others (3x) could figure out they need to take more different breaks. 7 out of 13 could learn something about their break duration, while 8 of 13 could learn something about their preferred interval. Figure \ref{fig:interval_duration} shows the collection of the selected break interval and duration of the participant to schedule their breaks during the intervention period. It appears that the distribution of the break interval is large among the participants (between 120min and 43min), but on average, the participants wanted to take a break every 66.4 min. This can also be strengthened by the post-questionnaire data. About half of the participants (6x) could state a specific break interval which does not deplete their resources too much and enables them to work focused. Some mentioned their preferred interval is 50min (2x), others 60min (2x) or 90min (2x). Four participants mentioned that they do not have one preferred interval, as it depends on certain circumstances as the tasks they are doing: \textit{“I feel like this is quite situational for me, mostly depending on the task. A task I really like doing and am totally focussed on, I can do for longer. When doing a task I like less, my time of attention is less, and it might make sense to take breaks more frequently.”}-[S06]. One effect of the break interval is the ability to resume work after a break. Some participants (2x) explicitly mentioned they like to have longer intervals, as they have problems resuming work after a break: \textit{"I don't like taking lots of breaks, because when I am in working mode, taking a break takes me completely out of it and it is hard to get into this mode again after the break."}-[S01]. On the other hand, for some people, it helps to recharge attention for tasks: \textit{“Yes, a duration of 50 minutes has worked well for me. It is useful to take breaks even if I am still on a task, to regain attention.”}- [S03]. 

\textbf{Break Scheduler Supporting the Different Duration Preferences.}
Differences also appear in the duration. From figure \ref{fig:interval_duration}, it can be concluded that the break duration does not diverge as much as the interval. The selected break duration stays between 5 and 20min. The average selected break duration is 11.5 min. When asked about break duration, most participants, except one, could find a preferred break duration. However, this duration does differ. Figure \ref{fig:category_duration} shows the duration and interval per person. This figure shows how different the break interval and duration of each participant are. About half of the users (6x) like to take longer breaks (15min or longer), while only two people like rather short breaks (less than 15min): \textit{“15 min generally shows the best results. Less time can result in not sticking to the schedule.”}-[S07], \textit{“I definitely think breaks can be too long and can also kick you out of focus. Therefore, in my case, I think the chosen break interval of 20min was very good.”}-[S10] or \textit{“Generally, I learned that breaks are more useful when they are longer (at least, on days when the overall energy level is already quite low from the start).”}-[S15]. A bigger group of participants (5x) mentioned they liked to mix their break duration during the day: \textit{“I like shorter breaks of 5 for activities like a coffee break, socializing, surfing on the internet,... because if longer, I feel like I am "too far away" from work after the break and it is more difficult to start again. For breaks like going outside and taking a walk, longer breaks are better for me as I am able to completely get my brain off of work and return with a "fresh" brain.”}-[S06] and others stated that it strongly depends on the activity during the break: \textit{ “Again, I found that the break duration needed varied. Of course, it also varies with the break activity!”}-[S17] or break categories: \textit{“I like long social breaks and short active breaks.”}-[S09]. 

\begin{figure}[htp]
    \centering
    \includegraphics[width=12cm]{hasel_thesis/images/duration_category.png}
    \caption{Each point represents one participant and their last adjusted break interval (in min) in relation to the break duration (in min), measured by the Break Scheduler for scheduling the last day of the intervention period. Each point has the colour of the highest-ranked category for this participant.}
    \label{fig:category_duration}
\end{figure}

\textbf{Personalisation of the Timing as a Key Feature of the Break Scheduler.}
The personalisation of the timing and duration in the Break Scheduler is, therefore, a key factor and needs to have features supporting the participants in doing this. One feature is the adaptation of the interval and duration, which helps the user to optimize their break habits and will be evaluated each evening. 10 out of 11 participants stated that these features helped them to adjust the timing to their needs. The integration of the schedule into the calendar was also mentioned (4x) as helpful in adjusting the timing or duration. However, following the suggested timing and duration was difficult for some participants, as stated here: \textit{“I would need a system that doesn't enforce taking breaks as strictly in terms of the timing.”}-[S15]. Nonetheless, the suggested timing also helped a few users to get more insights into their preferred break habits: \textit{“I tried for the first time doing shorter but more breaks with the Break Scheduler, but this doesn't work for me at all. With the Break Scheduler, I could confirm that few, long breaks are the best option for me.”}- [S01]. For users with an irregular work schedule and low job control, using the Break Scheduler appears to be more challenging due to the advanced planning of the breaks, as mentioned by one of the participants: \textit{"The app gave me good ideas and suggestions on what breaks to take at a certain time. It could not help me with my irregular work schedule."}- [S02]. 

\subsection{Break Activities} \label{activities_difference}

\begin{figure}[htp]
    \centering
    \includegraphics[width=12cm]{hasel_thesis/images/agreement_4.png}
    \caption{Post-questionnaire answers to the impact of the Break Scheduler }
    \label{fig:impact_agreement}
\end{figure}

The types of activities people prefer during their breaks can vary greatly, not just in terms of when and how long they take breaks. The pre-intervention questionnaire shows that preferences diverge when looking at the break activities. Many participants (7x) mentioned that body movement helps them to recharge their personal resources, while others (4x) prefer nutritional intake. All categories used by the Break Scheduler were stated as most beneficial for recharging the personal resources by at least one participant. Interestingly, in the pre-questionnaire, most mentioned break activities were activities done before the start of the day, during the lunch break or after the work day, but barely within-day break activities: \textit{“Body movement followed by meditation in the morning allows me to stay concentrated and alert most of the morning, which makes them my most productive times.”} –[S00],  \textit{“I take the longer path in the morning to get from public transport to the building, which kind of wakes me up.”}-[S06] or \textit{“I usually do some type of sport once a day for about an hour. Either outside or inside. Mostly early in the evening after returning from the office. During home-office, I may take the break earlier."} - [S15]. 

\begin{figure}[htp]
    \centering
    \includegraphics[width=10cm]{hasel_thesis/images/activities.png}
    \caption{Overview experience values of the activities per user}
    \label{fig:activity_ratings}
\end{figure}

\textbf{Personalisation of the Activities as a Key Feature of the Break Scheduler.}
After the intervention, participants learned something about their preference for break activities or categories. Figure \ref{fig:category_duration} shows the most beneficial categories for each participant and their timing preferences. It can be shown that there is no category which is preferred by all or most participants. Even though participants have different preferences, figure \ref{fig:impact_agreement} shows that the Break Scheduler can support the participants' preferences and help them to take more beneficial breaks. More than half of the participants (11x and 2 neutral) agreed that the suggested activities by the Break Scheduler are beneficial to recharge their personal resources and activities they like.  It also helps the participants (12x) to choose beneficial activities: \textit{“Through the app, I was motivated to do more of the good breaks.”}-[S09] or \textit{“I like that I did not have to choose the activity... It made a good decision for me”}- [S09]. The Break Scheduler could also help participants to find new break activities: \textit{“It also gave me new inspiration for activities during breaks and showed me that not every activity is equally recharging my resources.”}-[S06] or \textit{“I (actually) already knew which breaks are good for me (I was still doing them), but I learned about some more good ones like stretching.”}- [S09]. But not only the awareness of beneficial activities did increase, but about half (6x) of the participants could also learn something about bad break activities, which should be avoided. When looking at the experience values of the activities of each user, figure \ref{fig:activity_ratings} shows that there exists no "perfect break activity" fitting for all users. The darker the blue, the higher the experience value, which indicates how beneficial the activity is. If the activity has a shade of red, it indicates a negative experience value, which is a pointer for a chore activity. Interesting are the activities such as socialising or drawing, which are respite activities for some participants but chores for others. 



It can be concluded that the break timing but also the activity did defer from participant to participant. A tool which should help the user to improve their break habits must be able to adapt to the users' needs and enable personalization. Therefore the following statement can be concluded:

\begin{tcolorbox}[colback=white!5!white,colframe=black!75!black]
 \textbf{F6:} Break habits are very personal and differ from person to person. A tool supporting the user to create a beneficial break habit should enable personalization.
\end{tcolorbox}

\section{Exploring the Research Questions}
All of the previous findings F1-F5 of the primary evaluation consider different aspects of the Break Scheduler and how it was perceived by the participants. These insights are the basis for the evaluation and discussion of the research question RQ1-RQ3.

\subsection{Awareness RQ1} \label{RQ1} %findings F1, F3,F2 

RQ1 asks how self-reflection and nudging can increase the awareness of knowledge workers about their break habits and personal resources. In order to analyse this research question, it will be split into self-reflection and nudging. 

\textbf{Self-reflection} does help to raise awareness about personal resources and break habits. F1 states that self-reporting created more awareness about their break habits and personal resources. Additionally, the awareness helped the participants to act and choose beneficial activities (F2). Therefore, there is a trend that self-reflection increases knowledge workers' awareness about their break habits and personal resources. 

\textbf{Nudging} refers to a gentle and gradual push. In the context of break habits, it is about reminding the participants to take regular breaks through different tactics. One tactic of the Break Scheduler is scheduling the breaks in advance into the user's primary calendar (F3). This 1. helps to decrease chances of forgetting breaks as they are scheduled into the calendar, and 2. proactively plans the day with breaks in it, so they are less likely to get lost in the trouble of the day. Every time users look at their calendars, they see the planned breaks and are reminded to take them. This does not force the users but gently reminds them about breaks. Many users mentioned calendar synchronisation as positive, which helped them to be reminded about their breaks. The second tactic is the notifications sent by the Break Scheduler to indicate when a break is due (F4). The notifications, in combination with the planning, helped the participants to be more aware of their break habits and are more intentional with their breaks. 

Therefore, this thesis suggests that self-reporting, planning breaks ahead, and notification can increase the users' awareness about their break habits and personal resources such as energy, attention and physical well-being.

\subsection{Impact RQ2} \label{RQ2} %findings F5, F4
In RQ2, the impact of the Break Scheduler will be looked at; specifically, the impact on identifying good break activities, the impact on the user's break habits, and the ability to recharge personal resources during the breaks. To analyse this question, it will be split into two parts.

\textbf{Personalisation} is a crucial aspect of the Break Scheduler, as shown by F6 in section \ref{personal_break_habits}. There are no such "perfect break habits" that fit everybody; therefore, to have an impact, the Break Scheduler must enable users to personalise their break schedule. The Break Scheduler allows this by changing all three critical aspects of a break: timing, duration and activity. These factors can be adjusted manually, such as the definition of the break interval and the duration and the selection of the eight break activities the user wants to experiment with. Additionally, the interval, duration and activities can be adjusted during the day to maintain flexibility and adjust the momentary circumstances. It is essential to support users by personalising their break schedules. This is done by the review in the evening report, which adjusts the interval and duration. Moreover, the rating of the activities determines which activities will be chosen the next time. This personalisation enables the different users to find good break activities but also activities which are not beneficial, as shown in section \ref{activities_difference}.

\textbf{The Impact on Their Resources} could be shown in F5. As more participants were able to identify beneficial breaks, they achieved an average improvement in energy and attention levels as well as physical well-being after each break. This average improvement could also be shown in the evening as the participants felt less tired and sleepy, indicating they had more resources left. However, to state a clear impact, there is a need for more sophisticated results to reduce the threats. More will be discussed in the section \ref{threats}.

While the evidence found in this thesis is only suggestive, it appears that the Break Scheduler may help support knowledge workers to identify effective break activities and improve their break habits by increasing awareness. Additionally, the Break Scheduler might help recharge the participants' personal resources during the day through within-day breaks.

\subsection{Tool RQ3} \label{RQ3} %findings F3, F5
RQ3 covers the specific implementation of the Break Scheduler. This research question considers the aspects of the Break Scheduler and asks for the most helpful features and opportunities to improve the approach. 

Opportunities of the Break Scheduler mentioned by the participants are features such as the notification (F4), planning of the breaks ahead and incorporating them into the calendar of the user (F3), but also the different self-reporting features (F1). Most users perceived these features well, which helped them gain awareness and act to achieve a positive impact, as discussed in RQ1 and RQ2. Additionally, the user-friendly and simple design helped the users to navigate the application. As mentioned in F6, the personalization feature of the Break Scheduler is a key factor, which already has a solid base but still can be improved. The rule-based system should need to be improved as the system focus quickly on one activity. Similar holds for the rescheduling process, which can appear time-consuming if the user needs to reschedule multiple times a day. Further opportunities for improvements, such as the availability from the phone, are discussed in detail in section \ref{supporting_personalisation}. 

Overall, the Break Scheduler already has many valuable features which helped the participants gain more awareness and improve their break habits. Nonetheless, some features, such as the rule-based system, the rescheduling and adapting during the day, could be improved in future works. Also, the availability of new features, such as a phone application, can help improve the user experience. 

\chapter{Discussion and Future Work} \label{discussion}

\begin{comment}
In table \ref{table:key_findings}, the key findings of the preliminary evaluation are shown. In this chapter, all of these findings will be discussed. Additionally,  the three research questions, shown in table \ref{table:RQ} will be answered with the help of these five findings.


-

\section{Evaluation of the Results}
Before the findings can be used to discuss the research questions, the findings need to be evaluated. In order to validate the findings of the study, each finding will be discussed and compared to related work. 

\textbf{F1:} Nearly all participants perceived self-reporting personal resources as beneficial, as shown in section \ref{self-reporting}. It helped to raise awareness and forced the participant to reflect on the current state of the personal resources. This awareness helped them to choose beneficial breaks proactively. From the post-questionnaire, many users stated that the morning and evening reports helped them reflect on their personal resources. Additionally, the before and after break reports also helped them to reflect on whether the break activity was valuable. The constant reminder to reflect on their resources made them more aware and helped them realise when a break is needed. However, as stated in the Background chapter \ref{background}, personal resources also include more information than energy, attention and physical well-being. For future studies, exploring different factors of personal resources or physiological resources such as sleep or nutrition should be considered. Additionally, as discussed in the section \ref{rw_activity}, not only the activity affects the benefit of the break but also the psychological experience, which refers to the participant's mental state during the break. This factor is currently not considered explicitly in the reflection questions of the Break Scheduler. Some participants mentioned that it is sometimes challenging to emotionally detach from work during breaks. Having more insights into the mental state during a break can be helpful for the awareness of the participant as well as the experience value of the break. However, the before and after break report questions must remain minimal as many questions would hesitate participants to fill them out, which was already mentioned by one participant.


\textbf{F2:} The proactive planning of the breaks helped some participants have more structure and be more mindful of their break habits. The feature of scheduling the breaks into the user's calendar firstly helps to incorporate their schedule but also decreases the chance of forgetting about breaks. In combination with the notification, most participants perceived the planning well, enabling them to take more intentional breaks. As during the day many changes can come up, it is a crucial feature of the Break Scheduler to be able to reschedule breaks during the day. The scheduling of the breaks enforces a definition of timing. This is not favourable to all of the participants. Some like to stay flexible during the day and take breaks depending on different circumstances, such as energy or tasks. Others who have many short-notice changes in their schedule need to stay flexible during the day, too. As stated in the Related Work chapter \ref{related_work}, some studies could already find opportune moments during the day automatically by the software using different input data as shown in the study of Kaur et al. \cite{Kaur.2020}. Finding opportune moments during the day could be improved, as the rescheduling of the Break Scheduler is done manually in the current version. This can make it difficult for users who need to stay flexible and have difficulty defining their breaks the evening before.

\textbf{F3:} As stated in the Background chapter \ref{background}, knowledge workers often have a high job control, enabling them to decide on which task they want to work on and how to structure their workday, which gives them the possibility to take within-day breaks. However, they often overwork themselves and forget to take breaks during the day. During this study, most participants perceived the notifications as a reminder to take breaks during the day and be more aware of their break habits. In order to receive the notification, the participant has to schedule and plan the breaks. A positive effect is that the user does not need to think about taking breaks and will be reminded during the day. On the other hand, a notification often enforces a specific timing. Some participants do not perceive this very well, as they want to maintain flexibility during their work day. Another point mentioned is that the notifications are only visible on the laptop, which requires the user to stay on their laptop. When a knowledge worker does not spend their whole workday in front of the laptop, it is difficult to follow the schedule, as the notifications do not reach the participant. A future feature to solve this issue could be extending the application to be usable from the phone, as most participants have their phones with them most of the time.

\textbf{F4:} As shown in section \ref{awareness_resources}, raising awareness of their break habits and resources helped them better understand which activities are beneficial or not. Different studies show that personal resources can be recharged by taking respite breaks \cite{Trougakos.2009}. This can also be shown by the results gathered in this study. As shown in figure \ref{fig:sleepiness_agreement}, on average, the users were less sleepy and tired at the end of the day during the intervention week. This indicates that the participants could recharge some resources during the day, leaving them with more resources in the evening. Of course, this result can also be affected by factors not considered in this study. The advantage of taking beneficial breaks can be shown at the end of the day and after each break. On average most participants could achieve an improvement in energy, attention level or physical well-being after each break. However, all of this data is gathered by self-reporting, which can be affected by different threats discussed in section \ref{threats}. For future work, it could be interesting to consider data which can be measured during the focus interval, such as completed tasks or concentration.

\textbf{F5:} The break interval and duration differ from person to person and can be supported by the Break Scheduler. The calendar incorporation helps adjust the break schedule around scheduled meetings and helps participants with low job control to maintain their break habits. As the Break Scheduler does schedule the breaks in advance, it is challenging to consider momentary circumstances, such as the current state of the task, the chosen activity, the current resource level or other things. Some participants could feel a difference in their break interval or break duration depending on their motivation, which was also discussed in the Background chapter \ref{background}. Motivation is a potential counteract for depleted resources as stated by Bandura \cite{Bandura.1986} and can help participants to be focused for longer even though their resources are low. This theory can also be underlined by some participants of this study. Additionally, some participants mentioned that depleted resources are needing more resources to continue working and recharging them takes longer breaks. This can also be shown in the study of Trougakos and Hideg \cite{Trougakos.2009}. Depleted resources should also be considered by the Break Scheduler during the day. However, the current version has no automatic adjustment of the Break Scheduler during the day. Although the user can manually adjust the calendar, the system could be improved by checking the current attention level or the state of the tasks and adjusting the break timing accordingly, similar to Kauer et al. \cite{Kaur.2020}. It could also consider other measurements as the current step rate or sleep data, to adjust the timing, duration and activity to the user's needs during the day. Another point mentioned by some participants is that it should be possible to schedule different break durations during the day, depending on the daytime and activities. This could also be added in a future version of the Break Scheduler, enabling more users to adjust the schedule to their needs. Additionally, the rule-based system could be improved, as the method quickly focus's on one activity which will be suggested all the time. Instead, it should focus on a group of activities and ensure the user has some variety in their break schedule, as some participants requested.


\section{Awareness RQ1} \label{RQ1} %findings F1, F3,F2 

RQ1 asks how self-reflection and nudging can increase the awareness of knowledge workers about their break habits and personal resources. In order to analyse this research question, it will be split into self-reflection and nudging. 

\textbf{Self-reflection} does help to raise awareness about personal resources and break habits. F1 states that self-reporting created more awareness about their break habits and personal resources. Additionally, the awareness helped the participants to act and choose beneficial activities. Therefore, there is a trend that self-reflection increases knowledge workers' awareness about their break habits and personal resources. 

\textbf{Nudging} refers to a gentle and gradual push. In the context of break habits, it is about reminding the participants to take regular breaks through different tactics. One tactic of the Break Scheduler is scheduling the breaks in advance into the user's primary calendar. This 1. helps to decrease chances of forgetting breaks as they are scheduled into the calendar, and 2. proactively plans the day with breaks in it, so they are less likely to get lost in the trouble of the day. Every time users look at their calendars, they see the planned breaks and are reminded to take them. This does not force the users but gently reminds them about breaks. Many users mentioned calendar synchronisation as positive, which helped them to be reminded about their breaks. The second tactic is the notifications sent by the Break Scheduler to indicate when a break is due. The notification, in combination with the planning, helped the participants to be more aware of their break habits and are more intentional with their breaks. 

Therefore, this thesis suggests that self-reporting, planning breaks ahead, and notification can increase the users' awareness about their break habits and personal resources such as energy, attention and physical well-being.

\section{Impact RQ2} \label{RQ2} %findings F5, F4
In RQ2, the impact of the Break Scheduler will be looked at; specifically, the impact on identifying good break activities, the impact on the user's break habits, and the ability to recharge personal resources during the breaks. This question will be split into two parts in order to analyse this question.

\textbf{Personalisation} is a crucial aspect of the Break Scheduler, as shown by F5 in section \ref{personal_break_habits}. There are no such "perfect break habits" that fit everybody; therefore, to have an impact, the Break Scheduler must enable users to personalise their break schedule. The Break Scheduler allows this by changing all three critical aspects of a break: timing, duration and activity. These factors can be adjusted manually, such as the definition of the break interval and the duration and the selection of the eight break activities the user wants to experiment with. Additionally, the interval, duration and activities can be adjusted during the day to maintain flexibility and adjust the momentary circumstances. It is essential to support users by personalising their break schedules. This is done by the Break Scheduler with the adjustment of the interval and duration with the review of the evening report and the rating of the activities, determining which activities will be chosen the next time. This personalisation enables the different users to find good break activities but also activities which are not beneficial, as shown in section \ref{activities_difference}.

\textbf{The impact on their resources} could be shown in F4. As more participants were able to identify beneficial breaks, they achieved an average improvement in energy and attention levels as well as physical well-being after each break. This average improvement could also be shown in the evening as the participants felt less tired and sleepy, indicating they had more resources left. However, to state a clear impact, there is a need for more sophisticated results to reduce the threats. More will be discussed in the section \ref{threats}.

While the evidence found in this thesis is only suggestive, it appears that the Break Scheduler may help support knowledge workers to identify effective break activities and improve their break habits by increasing awareness. Additionally, the Break Scheduler might help recharge the participants' personal resources during the day through within-day breaks.

\section{Tool RQ3} \label{RQ3} %findings F3, F5
RQ3 covers the specific implementation of the Break Scheduler. This research question considers the aspects of the Break Scheduler and asks for the most helpful features and opportunities to improve the approach .

\textbf{Opportunities of the Break Scheduler} mentioned by the participants are features such as the notification (F3), planning of the breaks ahead and incorporating them into the calendar of the user (F2), but also the different self-reporting features (F1). Most users perceived these features well, which helped them gain awareness and act to achieve a positive impact, as discussed in RQ1 and RQ2. Additionally, the user-friendly and simple design helped the users to navigate the application. As mentioned in F5, the personalization feature of the Break Scheduler is a key factor, which already has a solid base but still can be improved.

\textbf{Opportunities for improvements} are features which are not existing or are not yet optimal in the current version of the Break Scheduler. Besides some minor bugs, some participants mentioned the importance of adjusting to different break duration over a day. Currently, only one break duration is supported for the scheduling. For future work, it would be beneficial for users to get different break duration suggested by the Break Scheduler depending on the activities and current state of personal resources, which also enables them to have more variety in the breaks and their activities. The break duration currently decides which group of activities can be chosen, namely the activities that fit this duration. However, in a future version, the Break Scheduler could choose the activity and adjust the duration accordingly, enabling more variety in the break duration and activities. Another improvement needs to be made with the rule-based system. The system focuses quickly on one activity rated with a high experience value. Focusing on a group of activities rather than one activity is essential. The current version already considered this by rounding the experience values. However, this rounding mechanism did not fully satisfy all the participants and needs improvement. Rescheduling the breaks did the job for most participants; however, if the user needs to reschedule many breaks, it appears to be time-consuming. Supporting the user during the day in adjusting to their personal needs, such as rescheduling when new meetings are coming up, or resending reminders after some time, would help reduce manual rescheduling. The last mentioned point is that the Break Scheduler is currently limited to the laptop. However, not all knowledge workers are always on their laptops, which is needed to use this application. Therefore, using the Break Scheduler from the phone would enable the users to use the tool more often and improve the notification as they could be sent to the phone.

Overall, the Break Scheduler already has many valuable features which helped the participants gain more awareness and improve their break habits. Nonetheless, some features, such as the rule-based system, the rescheduling and adapting during the day, could be improved in future works. Also, the availability of new features, such as a phone application, can help improve the user experience. 
    
\end{comment}
The Break Scheduler approach provides insights into supporting the user in finding more beneficial break activities to recharge their personal resources. Many knowledge workers with a high job control and high job demand forget to take breaks or decide to skip them in favour of a project. This can lead to depleted resources, leading to various negative aspects, such as increased stress levels or mental exhaustion \cite{Sonnentag.2001,Trougakos.2009}. Incorporating tiny positive habits into the workday is crucial to recharging personal resources and maintaining performance.

\section{Importance of Incorporating Tiny Positive Habits into the Workday} 
As knowledge workers spend a significant amount of time at work, it is essential to also consider the time spent at work to incorporate small positive habits. For most knowledge workers, the recovery process of their personal resources starts only after working hours \cite{Trougakos.2009}, which was also observed in the pre-intervention questionnaire of this study.

\textbf{Planning Breaks Ahead.}  Only a few participants were already incorporating regular breaks during working hours, although most agreed they would have the job control to take breaks whenever they want. By the theory of resource depletion \cite{BaumeisterR.F.BratslavskyE.MuravenM.&TiceD.M..1998}, an abundance of depleted resources lead to an even bigger need for resources. If knowledge workers only start recharging their resources after working hours, it may not be enough time to recharge them for the next day. On the other hand, resources can be recharged faster when the depletion is not yet too low. This strengthens the process of incorporating tiny habits into the workday. Examples of such tiny habits are breaks. The feature of the Break Scheduler to schedule the breaks into the user's calendar firstly helps to incorporate their schedule but also decreases the chance of forgetting about breaks. In combination with the notification, most participants perceived the planning well, enabling them to take more intentional breaks. On the other hand, as many changes can come up during the day, it is a crucial feature of the Break Scheduler to reschedule breaks during the day. 

\textbf{Maintain Flexibility during the Day.} Scheduling of the breaks enforces a definition of timing. This is not favourable to all of the participants. Some like to stay flexible during the day and take breaks depending on different circumstances, such as energy or tasks. Others who have many short-notice changes in their schedule need to stay flexible during the day, too. Therefore, finding opportune moments for breaks depending on the momentary circumstances remains a challenge. As stated in the Related Work chapter \ref{related_work}, some studies could already find opportune moments during the day automatically by the software using different input data as shown in the study of Kaur et al. \cite{Kaur.2020}. Finding opportune moments during the day could be improved in this approach, as the rescheduling of the Break Scheduler is done manually in the current version. This can make it difficult for users who need to stay flexible and have difficulty defining their breaks the evening before.
That tiny positive habits, such as breaks, can influence the personal resources of the users can be shown by different studies \cite{KimS.ParkY.&Niu.2017,Largo-Wight.2017} and will be strengthened by this study, too. Many participants could feel a positive difference in their personal resources, during the intervention week, as stated in section \ref{resources}, since they took more intentional breaks with beneficial activities. As shown in figure \ref{fig:sleepiness_agreement}, on average, the users were less sleepy and tired at the end of the day during the intervention week. This indicates that the participants could recharge some resources during the day, leaving them with more in the evening. However, all of this data is gathered by self-reporting, which can be affected by different threats discussed in section \ref{threats}. For future work, it could be interesting to consider data which can be measured during the focus interval, such as completed tasks or concentration.


\section{Self-Reflection as a Tool to Create Action}
Many participants confirmed their unawareness of personal resources and break habits before the intervention period. As stated in other studies \cite{GRANT20083}, self-reflecting is a common tool to create awareness and can also lead to action. This study suggested that the created awareness through self-reporting can help users to improve their break habits and find beneficial activities. On the other hand, the constant reminder to reflect on their resources made them more aware and helped them realise when a break is needed. Nonetheless, many participants also stated that they would benefit from using the Break Scheduler over an extended period by finding more patterns in their break habits and improving their knowledge of beneficial breaks. 

\textbf{Factors of Personal Resources.} As stated in the Background chapter \ref{background}, personal resources also include more information than energy, attention and physical well-being. For example, the amount of sleep and the need for nutritional intake were also mentioned by some participants as influential factors in their break habits. As shown in several studies \cite{Rosekind.2010,Gingerich.2017,Choi.2018}, a good night's sleep can help to stay focused longer, which means taking fewer breaks. For future studies, exploring different factors of personal resources or physiological resources such as sleep or nutrition should be considered. 

\textbf{Psychological Experience during the Break.} Additionally, as discussed in the section \ref{rw_activity}, the activity affects the benefit of the break and the psychological experience, which refers to the participant's mental state during the break. This factor is currently not considered explicitly in the reflection questions of the Break Scheduler. Some participants mentioned that it is sometimes challenging to emotionally detach from work during breaks. Having more insights into the mental state during a break can be helpful for the awareness of the participant as well as the experience value of the break. However, the before and after break report questions must remain minimal as too many questions would hesitate participants to fill out the break reports.


\section{Supporting Personalisation} \label{supporting_personalisation}
From the primary evaluation of this thesis, it became apparent that in different aspects of the Break Scheduler, personalisation is a crucial aspect which needs to be supported through different factors. As discussed in the Related Work chapter \ref{related_work}, how beneficial a break activity is, depends on the person, as the preferences influence how the activity is perceived. This is also highlighted by this thesis, as activities such as socializing or drawing were beneficial for some participants or chores for others. However, not only do preferences of activities need to be considered, but also break habits can differ.

\textbf{Internal and External Factors for Break Habits} Some participants like to take fewer breaks with a longer duration, while others prefer more regular breaks but shorter duration. This thesis suggests that internal and external factors can influence the user's preference for break intervals and duration. Internal factors, such as how hard it is for the user to resume work after a break or how highly motivated the user is, are mentioned to impact the break interval and duration. Nevertheless, different external circumstances can also influence these preferences, such as the work tasks, the needed attention level and interruptions, which can happen in open office spaces. It can be assumed that many more factors influence the preference for break habits and the break interval and duration. Nonetheless, internal and external factors should be considered in a future version of the Break Scheduler approach, enabling users to adjust even better to their current needs.

\textbf{Opportunities for Improvements.} Besides some minor bugs, some participants mentioned the importance of adjusting the break duration during the day. Currently, only one break duration is supported for the scheduling. For future work, it would be beneficial for users to get different break duration suggested by the Break Scheduler depending on the activities and current state of personal resources, which also enables them to have more variety in the breaks and their activities. The break duration currently decides which group of activities can be chosen, namely the activities that fit this duration. However, in a future version, the Break Scheduler could choose the activity and adjust the duration accordingly, enabling more variety in the break duration and activities. Another improvement needs to be made with the rule-based system. The system focuses quickly on one activity rated with a high experience value. Focusing on a group of activities rather than one activity is essential. The current version already considered this by rounding the experience values. However, this rounding mechanism did not fully satisfy all the participants and needs improvement. Rescheduling the breaks did the job for most participants; however, if the user needs to reschedule many breaks, it appears to be time-consuming. Supporting the user during the day in adjusting to their personal needs, such as rescheduling when new meetings are coming up, or resending reminders after some time, would help reduce manual rescheduling. The last mentioned point is that the Break Scheduler is currently limited to the laptop. However, not all knowledge workers are always on their laptops, which is needed to use this application. Therefore, using the Break Scheduler from the phone would enable the users to use the tool more often and improve the notification as they could be sent to the phone. 


\section{Threats to Validity} \label{threats}

As already mentioned throughout the Discussion, the preliminary evaluation has a few limitations. 

\textbf{External Validity.} First and foremost, the number of participants (N=13) limits the generalizability of the results of this thesis significantly. In order to make a conclusive statement for a wider group of people, more participants with different backgrounds should be gathered. Additionally, the selection bias from the word-to-mouth and snowball sampling methods used to gather the participants can threaten the validity. Most of the participants have similar demographic backgrounds, which makes the participant group biased. On the other hand, the self-reflection bias during the use of the Break Scheduler and in the pre-and post-intervention questionnaire can influence the outcome of the results. This could be decreased by incorporating more measured data as finished tasks or measured attention levels. Additionally, further research is needed to explore the long-term impact of using a Break Scheduler on well-being and productivity.

\textbf{Internal Validity.} Another threat is that most of the data collected rely on self-reporting as the pre- and post-intervention data and the personal resources self-reporting during the intervention period. Additionally, the novelty effect could affect the positivity towards self-reflection, which could fade after some time. In further studies, more measured data should be gathered, and a long-term study should be evaluated to mitigate this threat.

\textbf{Construct Validity.} As part of the qualitative data analysis in the thematic analysis, participants' responses from the pre- and post-intervention questionnaire were encoded. One potential threat can be that this open coding step was done by only one person and reviewed by a second person. This step could be discussed in teams to reduce bias in future work.



\chapter{Conclusion} \label{conclusion}

%
%- short summary
%- conclusion
%   - show limitations
This thesis explores how the Break Scheduler approach  can improve the awareness of personal resources and the user's break habits and how it can support the user in finding beneficial break activities to recharge their resources. Through the investigation of the research questions, several insights could be found.

Firstly, self-reporting personal resources create more awareness, helps to be more mindful about how to spend the resources and helps to find beneficial break activities. Secondly, proactive planning of breaks helps to be more mindful of them and can structure the day. Combined with notifications, this can raise awareness and remind users to take breaks. Thirdly, taking more beneficial breaks improves the participants' personal resources, which already existing studies can also underline \cite{Trougakos.2009}. However, break habits are very personal and differ greatly. Therefore, a tool supporting the user in creating beneficial break habits should enable personalisation.

Overall, the findings suggest that the Break Scheduler can be a valuable tool to promote beneficial break habits for knowledge workers by increasing the awareness of personal resources and break habits through self-reporting and nudging. The personalisation and other features of the Break Scheduler have the potential to help users improve their break habits and recharge their energy and attention levels.

Despite the promising findings of this thesis, there are a few limitations which should be addressed in future studies. For example, the small number of participants and the selection bias through the snowballing selection method. Also, the self-reflection bias must be evaluated, as most data is generated by self-reporting. Additionally, further research is needed to explore the long-term impact of using a Break Scheduler on well-being and productivity.

Overall, this thesis can offer essential insights into the potential of the Break Scheduler approach  in supporting knowledge workers to improve their break habits and create awareness about their resources. However, future research is needed in the area of beneficial break activities and break habits, hopefully, inspired by the insight of this thesis, contribute to the growing body of knowledge on the importance of breaks and well-being in the workplace.






% 
% \subsubsection{Subsubsection}
% \fig[.5\textwidth]{logos/logo_hasel}{Our logo}{logo}

% \subsection{Subsection}
% %
% \paragraph{Paragraph.} Always with a point.

% \begin{lstlisting}[caption=An example code snippet]
% /**
%  * Javadoc comment
%  */
% public class Foo {
% 	// line comment
% 	public void bar(int number) {
% 		if (number < 0) {
% 			return; /* block comment */
% 		}
% 	}
% }
% \end{lstlisting}


\backmatter
\bibliographystyle{alpha}
\bibliography{hasel_thesis/references}

\appendix
\chapter{Collected Data}

\begin{figure}[htp]
    \centering
    \includegraphics[width=8cm]{hasel_thesis/images/overall_reports_v2.png}
    \caption{Graph of the total reports collected during the intervention}
    \label{fig:overall_reports}
\end{figure}


\begin{figure}[htp]
    \centering
    \includegraphics[angle=90, width=10cm]{hasel_thesis/images/input_output_values.png}
    \caption{Approach Static and daily specific input and output values}
    \label{fig:approach_input_output}
\end{figure}

\chapter{Provided Activities and Categories from the Break Scheduler}

\begin{table}[ht] 
\centering
    \begin{tabular}{ |p{7cm} p{7cm}|  }
    \hline
     Activity & Categories\\
     \hline
     \rowcolor{Gray}
     Reading &   Relaxation, Cognitive\\
     Walking &   Body Movement, Exposure to nature\\
     \rowcolor{Gray}
     Drink water &   Beverages-intake\\
     Stretching &   Body Movement\\
     \rowcolor{Gray}
     Snacking &   Relaxation\\
     Drawing &   Relaxation, Cognitive\\
     \rowcolor{Gray}
     Exercising indoor &   Body Movement\\
     Exercising outdoor &   Body Movement, Exposure to nature\\
     \rowcolor{Gray}
     Get lunch with others &   Nutrition-intake, Social\\
     Eat lunch by yourself &   Nutrition-intake\\
     \rowcolor{Gray}
     Journaling &   Relaxation, Cognitive\\
     Power nap &   Relaxation\\
     \rowcolor{Gray}
     Domestic tasks &   Cognitive\\
     Socializing &   Social\\
     \rowcolor{Gray}
     Cooking + eating &   Nutrition-intake\\
     Meditation &   Relaxation\\
     \rowcolor{Gray}
     Drink coffee &   Beverages-intake\\
     Go outside and breathe fresh air &   Exposure to nature\\
     \rowcolor{Gray}
     Social Media &   Relaxation, Social\\
     Gaming &   Relaxation, Cognitive\\
     \rowcolor{Gray}
     Texting, Chatting via phone or phone call &   Social\\
     Surfing the web for non-work purposes &   Relaxation, Cognitive\\
     \hline
    \end{tabular}
    \caption{Set of activities provided by the Break Scheduler associated with the categories}
 \label{table:activities_categories}
\end{table}

\chapter{Pre-Intervention-Questionnaire} \label{pre-intervention-questionnaire}
\includepdf[pages=-]{pdf/pre-intervention-questionnaire.pdf}

\chapter{Post-Intervention-Questionnaire} \label{post-intervention-questionnaire}
\includepdf[pages=-]{pdf/post-intervention-questionnaire.pdf}

\chapter{Database Overview} \label{fig:database}

\begin{figure}[htp]
    \centering
    \includegraphics[width=15cm]{hasel_thesis/images/db.png}
    \caption{Database ER-Diagram}
    \label{fig:db}
\end{figure}


\end{document}
